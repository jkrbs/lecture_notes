\documentclass{../tudscript}
\title{Mathe VL 9}
\author{Jakob Krebs}

\begin{document}
    \ssect{Beispiel: Taylorpolynome}
        \ilmath{f: \bR \rightarrow \bR: x \mapsto x^2 -1}
        gesucht: \underline{Taylorpolynom ($P_n (x)$)} von f(x).
        \ilmath{f(x) &= x^2 -1 \\
                f'(x)&= 2x \\
                f''(x) &= 2 \\
                f'''(x) &= 0 \\
                \vdots}

        \ilmath{x_0 &= 0 \\
                f(x_0) &= -1 \\
                f'(x_0) &= 0 \\
                f''(x_0) &= 2 \\
                f'''(x_0) &= 0 \\
                \vdots}

        \ilmath{p_n (x) &= f(0) + f'(0) (x-0) + \frac{f''(0)}{2!} (x-0)^2 + \frac{f'''(0)}{3!} (x-0)^3 + \ldots \\
                        &= -1 + 0 \cdot x + \frac{2}{2!} x^2 + 0 = -1 + x^2 = f(x)}

        an der Entwicklungsstelle
        
        \ilmath{x_0 &= 1 \\
                f(x_0) &= 0 \\
                f'(x_0) &= 2 \\
                f''(x_0) &= 2 \\
                f'''(x_0) &= 0 \\
                \vdots}

        \ilmath{p_n (x) &= f(1) + f'(1) (x-1) + \frac{f''(1)}{2!} (x-1)^2 + \frac{f'''(1)}{3!} (x-1)^3 + \ldots \\
                        &= 0 + 2 \cdot (x-1) + \frac{2}{2!} (x-1)^2 + 0 = 2x -2 + x^2 - 2x + 1 = f(x)}

        nun $e^x$
        \ilmath{f(x) = \frac{e^x}{cos x}}
        gesucht: $p_x(x)$
    \ssect{Methode des implizieten Differenzierens}
        \ilmath{f(x) \cdot cos x = e^x \\
                f'(x) cos x + f(x) (-sin x) = e^x \\
                f''(x) cos x + f'(x) (-sin x) + f'(x) (-sin x) + f(x) (-cos x) = e^x}
        (ges. $f'(0), f(0), f''(0)$)
        \ilmath{f(0) \cdot cos 0 = e^0 \implies f(0) = 1}
        \ilmath{f'(0) 1 + \frac{1}{f(0)} \cdot 0 = 1 \implies f'(0) = 1}
        \ilmath{f'(0) 1 + f(0) (-1) = 1 \implies f''(0) = 2}
        \ilmath{p_2(x) = 1 + 1x + \frac{2}{2!} x^2 = 1 + x + x^2}
        
    \ssect{Beispiel: sonderfälle bei Taylorpolynomen}
        \ilmath{f(x) = \frac{1}{x+1}\text{ rot}} 
        \ilmath{p_1(x) &= 1-x\text{ blau}\\
                p_2(x) &= 1 - x + x^2\text{ grün}\\
                p_3(x) &= 1 - x + x^2 -x^3\text{ orange}}\\

		Die immer bessere Annäherungen an die echte Funktion.

        \begin{tikzpicture}
	        \draw[->] (-1.2,0) -- (4.2,0) node[right] {$x$};
	        \draw[->] (0,-0.2) -- (0,3.2) node[above] {$y$};
	        \draw[scale=1,domain=-0.75:4,smooth,variable=\x,red] plot ({\x},{1/(\x+1)});
	        \draw[scale=1,domain=-0.75:3,smooth,variable=\x,blue] plot ({\x},{1-\x});
	        \draw[scale=1,domain=-0.75:2,smooth,variable=\x,green] plot ({\x},{1-\x+\x^2});
	        \draw[scale=1,domain=-0.75:1.5,smooth,variable=\x,orange] plot ({\x},{1-\x+\x^2-\x^3});
        \end{tikzpicture}

    \ssect{Taylor-Formel}
        \ilmath{f(x) = p_n(x) + \underbrace{R_n(x, x_0)}_{\text{n-tes Restglied}}}
        $R_n(x, x_0)$: Fehler bei der Approximation
        
        \sssect{Satz:}
            Darstellung von $R_n(x, x_0)$ nach Lagrange.

            Sei
            \ilmath{f: (a, b) \rightarrow \bR}
            eine $(n+1)$-mal stetig differenzierbare Funktion. $x_0 \in (a, b)$
 
            Dann gilt:
            \ilmath{f(x) = p_n (x) + R_n(x, x_0)}
            und
            \ilmath{\forall x \in (a, b): \exists z \in \bR: \text{zwischen x und $x_0$}: \\
            R_n (x, x_0) = \frac{f^{(n+1)}(z)}{(n+1)!} (x-x_0)^{n+1}}
    \sssect{Beispiel}
        \ilmath{f(x) = e^x, x_0 = 0}
        \ilmath{p_n(x) = \sum_{k = 0}^\infty \frac{f^{(k)}(0)}{k!} x^k = \sum_{k = 0}^\infty \frac{x^k}{k!}}
        \ilmath{f(x) = p_n (x) + R_n (x, x_0)}
        und
        \ilmath{R_n (x, x_0) = \frac{e^z}{(n+1)!} x^{n+1}}
        für ein geeignetes $z \in (0,x)$oder $z \in (x, 0)$

        Wir betrachten
        \ilmath{f(x) = e^x \text{ für } |x| \leq 1}
        \ilmath{|R_n(x, 0)| = |\frac{e^z}{(n+1)!} x^{n+1}| \leq \frac{e^z}{(n+1)!} \cdot 1^{n+1}}
        Näherungsformel für $e^x$
        \ilmath{p_5 (x) = 1  + x+ \frac{x^2}{2!} + \frac{x^3}{3!} + \frac{x^4}{4!} + \frac{x^5}{5!}}

        \ssect{Definition: Taylorreihe}
            Sei
            \ilmath{f: (a, b) \rightarrow \bR}
            beliebig oft stetig differenzierbar und $x_0 \in (a, b)$
            Die Reihe
            \ilmath{\sum_{k = 0}^\infty \frac{f^{(k)} (x_0)}{k!} (x-x_0)^k}
            heißt \underline{Taylorreihe} von f an der Stelle $x_0$.
        \ssect{Bemerkungen zu Taylorreihen}
        \begin{enumerate}
            \item Nicht für jede Funktion f(x) ist die Taylorreihe konvergent.
            \item Ist die Taylorreihe konvergent, dann muss der Grenzwert nicht die Funktion f sein.
            \item Ist die Taylorreihe konvergent gegen f, dann heißt
                \ilmath{f(x) = \sum_{k = 0}^\infty \frac{f^{(k)} (x_0)}{k!} (x-x_0)^k}
                reel analytisch.
                %%TODO the nth example
        \end{enumerate} 
   \ssect{Rechnen mit Potenzreihen}
        Sei
        \ilmath{ a(x) = \sum a_k (x-x_0)^k, b(x) = \sum b_k (x-x_0)^k}
        mit Konvergenzradius $r_1$ für $a(x)$ und $r_2$ für $b(x)$.
        Sei 
        \ilmath{r := min \Set{r_1, r_2}}
        dann gilt:
        \ilmath{a(x) \pm b(x) = \sum (a_k \pm b_k) (x-x_0)^k \text{ für } x \in (x_0 -r, x_0 +r)}
        \ilmath{x \cdot a(x) = \sum c a_k (x-x_0)^k \text{ für } x \in (x_0 -r, x_0 + r)}

        %%TODO 
        \ilmath{a(x) \cdot b(x) = \sum (a_0 b_k + }
        \ilmath{\frac{1}{a(x)} \text{ für } f(x) \neq 0}
        kann mit der %%TODO berechnet werden.
\end{document}
