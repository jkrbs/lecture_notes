\documentclass{../../meta/tudscript}
\begin{document}
\sect{Aussagenlogik}

\ssect{Definition: Aussage}

Eine Aussage ist eine endliche Zeichenfolge, der genau einer der
Wahrheitswerte ``wahr'' (1) oder ``falsch''(0) zugeordnet ist.

\ssect{Beispiele}
\begin{itemize}
\item Dresden liegt an der Elbe
\item Die Erde ist eine Scheibe
\item 4 ist eine Primzahl
\end{itemize}
\ssect{Definition}

Zwei Aussagen A und B können durch Junktoren zu neuen Aussagen verknüpft
werden:


\begin{enumerate}
\def\labelenumi{\arabic{enumi}.}
\item
$\neg A$ :  \''nicht A\'' ist wahr, wenn A falsch is. (Negation)
\item
$A \vee B $: \''A oder B\''  ist wahr, wenn  A oder B oder beide wahr sind. (Disjunktion)
\item
$A \wedge B$ : \''A und B\'' ist wahr, wenn A und B wahr sind. (Konjunktion)
\item
\(A \Rightarrow B\) : \''A impliziert B\'' wenn A, dann B. (Implikation)
\item
\(A \Leftrightarrow B\) : \''genau dann A, wenn B  \'' ist wahr, wenn A und B wahr sind oder beide falsch sind . (ist wahr, wenn A und B wahr sind oder beide falsch sind)

\end{enumerate}

\ssect{Darstellung als Wahrheitstabelle}

\begin{tabular}{l c c c c c c r}
           A & B & \neg A & A\vee B &A\wedge B &A \Rightarrow & A \Leftrightarrow B \\
            0 & 0 & 1 & 0 & 0 & 1 & 1 \\
            0 & 1 & 1 & 1 & 0 & 1 & 0\\
            1 & 0 & 0 & 1 & 0 & 0 & 0 \\
            1 & 1 & 0 & 1 & 1 & 1 & 1 \\
\end{tabular}

\ssect{Proposition: Grundlegende Gesetze}
Folgende Aussagen sind wahr für beliebige Aussagen A,B,C:

\begin{enumerate}
\def\labelenumi{\arabic{enumi}.}
\item
  \(A \vee (B \vee C) \Leftrightarrow (A \vee B) \vee C\)

  \(A\wedge (B \wedge C) \Leftrightarrow (A \wedge B) \wedge C\)
\item
  \(A \vee \urcorner A\)
\item
  \(A \vee B \leftrightarrow B \vee A\)

  \(A \wedge B \leftrightarrow B \wedge A\)
\item
  \(A \vee (B \wedge C) \Leftrightarrow (A \vee B) \wedge (A \vee C)\)

  \(A \wedge (B \vee C) \Leftrightarrow (A \wedge B) \vee (A \wedge C)\)
\item
  \(\urcorner (A \vee B) \leftrightarrow \urcorner A \wedge \urcorner B\)

  \(\urcorner(A \wedge B) \leftrightarrow \urcorner A \vee \urcorner B\)
\item
  \(\urcorner \urcorner A \Leftrightarrow A\)
\item
  \((A \leftrightarrow B) \leftrightarrow ((A \rightarrow B) \wedge (B \rightarrow A))\)
\item
  \(A \rightarrow B \leftrightarrow (\urcorner A \vee B)\)
\end{enumerate}

Beweis durch Wahrscheinlichkeitstabellen oder durch Benutzung bereits
bekannter Aussagen (am Beispiel von 8)

\begin{tablular}{l c c c c r}
A  & B  & \(\urcorner A\)  &\(\urcorner A \vee B\)  &\(A \rightarrow B\)  &\((A \rightarrow B) \leftrightarrow (\urcorner A \vee B)\)  \\
0  & 0  & 1  & 1  &1  &1 \\
0  & 1  & 1  & 1  &1  &1  \\
1  & 0  & 0  & 0  &0  &1  \\
1  & 1  & 0  & 1  &1  &1  \\
\end{tablular}

Prädikatenlogik Wir betrachten im Folgenden Variablen x,y,z,\ldots aus
einem festzulegenden Universum, z,B alle Menschen, alle natürlichen
Zahlen, etc.

\ssect{Definition}

Ein Prädikat ist eine endliche Zeichenfolge, die Variablen beinhalten
kann und der für jede Variablenbelegung aus dem Universum ein
Wahrheitswert zugeordnet ist. Ein Prädikat, das für alle
Variablenbelegungen wahr ist heißt Tautologie.

\ssect{Beispiele}

\begin{tablular}{}
Prädikat  & Universum  \\

A(x)=``x ist eine Katze''  & alle Tiere  \\
B(x)=``x ist eine Primzahl''  & alle natürlichen Zahlen  \\
C(x,y)=``x ist Teiler von y''  & alle natürlichen Zahlen  \\
D(x,y,z)=``\(x^2 + y^2 = z\)''  & alle natürlichen Zahlen \\
E(x)=``x ist Primzahl \(\rightarrow\) x ist ungerade''  & alle natürlichen Zahlen  \\
F(x, y)=``(x gerade) \(\wedge\) (y gerade) \(\rightarrow\) (x + y) gerade''  &
alle natürlichen Zahlen  \\
\end{tablular}

Wie Aussagen lassen sich Prädikate durch Junktoren (Negation,
Disjunktion, Konjunktion, Implikation, Äquivalenz, \ldots{}) zu neuen
Prädikaten verbinden. Wir führen nun den Allquantor \(\forall\) und den
Existenzquantor \(\exists\) ein.

\ssect{Definition}

Sei P(x) ein Prädikat. Dann sind \(\forall x: P(x)\) und
\(\exists x: P(x)\) Aussagen mit folgenden Wahrheitswerten:

\begin{itemize}
\item
\(\forall \> x: P(x)\) ~``für alle x gilt P(x)''  ist wahr, wenn P(x) für alle x aus dem Universum wahr ist, ansonsten
falsch
\item
\(\exists \> x: P(x)\) ~``es existiert x mit P(x)''ist wahr, wenn P(x) für mindestens x aus dem Universum wahr ist,
ansonsten falsch.
\item
Weiter definieren wir $\exists ! \> x: P(x) : \iff  \exists \> x: P(x) \wegde (\forall x,y : P(x) \wedge P(y) \rightarrow x=y$
\end{itemize}

\ssect{Bemerkung}

Seien P(x) und Q(x) Prädikate, so gilt:

\begin{itemize}
\item \(\forall \> x: P(x)) \leftrightarrow \exists \>x: \urcorner P(x)\)
\item \(\urcorner (\exists \> x: P(x)) \leftrightarrow \forall \> x: \urcorner P(x)\)
\item \((\forall \> x: P(x)) \wedge (\forall \> x: Q(x)) \leftrightarrow \forall \> x: P(x) \wedge Q(x)\)
\item \((\exists \> x: P(x)) \vee (\exists \> x: Q(x)) \leftrightarrow \exists \> x: P(x) \vee Q(x)\)
\end{itemize}

\sect{Mengen}

\ssect{Naive Definition (Georg Cantor,
1895)}

Unter einer Menge verstehen wir jede Zusammenfassung M von bestimmten
wohlunterschiedenen Objekten unserern Anschauung oder unsers Denkens
(welche die Elemente von M genannt werden) zu einem Ganzen

Beispiele für Mengen sind

\begin{itemize}
\item \{rot, grün, blau\}, \{0, 1\}
\item \(\mathbb{N}\) = \{0, 1, 2, 3, \ldots{}\}
\item \(\mathbb{Z}\) = \{\ldots{}, -2, -1, 0, 1, 2, \ldots{}\}
\item \(\mathbb{Q}, \> \mathbb{R}, \> \mathbb{C}\)
\end{itemize}

\ssect{Notation}

\begin{itemize}
\item \(x \in M\) ``x ist Element der Menge M''
\item \(x \in M\) ``x ist nicht Element der Menge M''
\end{itemize}


\end{document}
