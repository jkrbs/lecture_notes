\documentclass{../tudscript}
\begin{document}
\hypertarget{einfuxfchrung-in-die-mathematik-fuxfcr-informatikerinnen}{%
\sect{Einführung in die Mathematik für
InformatikerInnen}\label{einfuxfchrung-in-die-mathematik-fuxfcr-informatikerinnen}}

\hypertarget{vl}{%
\sect{1. VL}\label{vl}}

\hypertarget{personen}{%
\ssect{Personen}\label{personen}}

\begin{itemize}
\tightlist
\item
  Prof.~Ulrike Baumann

  \begin{itemize}
  \tightlist
  \item
    ulrike.baumann@tu-dresden.de
  \end{itemize}
\item
  Kursassistenz: Antje Noack

  \begin{itemize}
  \tightlist
  \item
    antje.noack@tu-dresden.de
  \end{itemize}
\end{itemize}

\hypertarget{klausuren}{%
\ssect{Klausuren}\label{klausuren}}

\begin{enumerate}
\def\labelenumi{\arabic{enumi}.}
\tightlist
\item
  Klausur 90min

  \begin{itemize}
  \tightlist
  \item
    Anfang Dezember
  \item
    Wdh. Termine zu Beginn des SoSe19
  \end{itemize}
\item
  Klausur 120min

  \begin{itemize}
  \tightlist
  \item
    Prüfungszeitraum des WiSe18/19
  \item
    Wdh. Termine im Prüfungszeitraum des SoSe19
  \end{itemize}
\end{enumerate}

\hypertarget{hausaufgabe}{%
\ssect{Hausaufgabe}\label{hausaufgabe}}

\begin{itemize}
\tightlist
\item
  Nachbereitung der Vorlesung
\item
  Zettel liegen auf Seite der VL

  \begin{itemize}
  \tightlist
  \item
    getrennt zzwischen dis und lin
  \end{itemize}
\item
  regeölmäßiges Lösen empfohlen

  \begin{itemize}
  \tightlist
  \item
    um die VL weiterhin zu verstehen
  \end{itemize}
\item
  HA kann zur Bewertung abgegeben werden

  \begin{itemize}
  \tightlist
  \item
    auf den Zetteln mit A gekennzeichnet
  \end{itemize}
\item
  Bonuspunkte in Klausur

  \begin{itemize}
  \tightlist
  \item
    wenn Klausur knapp nicht bestanden, kann HA dich retten
  \end{itemize}
\end{itemize}

\hypertarget{hilfsmittel-bei-klausur}{%
\ssect{Hilfsmittel bei Klausur}\label{hilfsmittel-bei-klausur}}

\begin{itemize}
\tightlist
\item
  kein Taschenrechner
\item
  keine elektronischen Hilfsmittel
\item
  eine A4 Seite beidseitig beschrieben

  \begin{itemize}
  \tightlist
  \item
    keine Kopie
  \end{itemize}
\end{itemize}

\hypertarget{literaturempfehlung}{%
\ssect{Literaturempfehlung}\label{literaturempfehlung}}

\begin{itemize}
\tightlist
\item
  Formeln + Hilfen zur höheren Mathematik

  \begin{itemize}
  \tightlist
  \item
    ISBN-10: 392392335X
  \item
    ISBN-13: 978-3923923359
  \end{itemize}
\item
  weitere Lit. auf der Website
\item
  Gerd Fischer, Lineare Algebra

  \begin{itemize}
  \tightlist
  \item
    DOI 10.1007/978-3-658-03945-5
  \end{itemize}
\item
  Grundwissen Mathematik, Springer

  \begin{itemize}
  \tightlist
  \item
    ISSN: 1431-4215
  \end{itemize}
\end{itemize}

\hypertarget{inhalt-der-vl}{%
\ssect{Inhalt der VL}\label{inhalt-der-vl}}

\begin{itemize}
\tightlist
\item
  Körper der komplexen Zahlen
\item
  Matrizen
\item
  Linera Gleichungssysteme
\item
  Vektorräume über Körper
\item
  Lineare Abbildung
\item
  Determinanten
\item
  Euklidische vektorräume
\item
  Best Approximation
\end{itemize}

\hypertarget{vl-einfuxfchrung-kuxf6rper-der-komplexen-zahlen}{%
\ssect{1.VL Einführung Körper der komplexen
Zahlen}\label{vl-einfuxfchrung-kuxf6rper-der-komplexen-zahlen}}

\begin{itemize}
\tightlist
\item
  zahlenbereich:

  \begin{itemize}
  \tightlist
  \item
    Menge von Zahlen
  \item
    Struktur von Zahlen

    \begin{itemize}
    \tightlist
    \item
      Es wird eine Struktur, wenn die Rechenoperationen darauf Anwendbar
      sind und die Eigenschaften auf die Struktur zutreffen
    \end{itemize}
  \end{itemize}
\end{itemize}

\hypertarget{begriffe}{%
\sssect{Begriffe}\label{begriffe}}

\begin{itemize}
\tightlist
\item
  Definitionen
\item
  Sätze

  \begin{itemize}
  \tightlist
  \item
    Aussagen, die richtig sind und beweisbar sind
  \end{itemize}
\item
  Beweis
\item
  Lemma

  \begin{itemize}
  \tightlist
  \item
    Hilfssätze
  \end{itemize}
\item
  Folgerungen
\item
  im Prinzip auch Sätze mit offentsichtlichem Beweis
\item
  Bemerkungen

  \begin{itemize}
  \tightlist
  \item
    andere richtige Aussagen
  \item
    einfacher und ohne Beweis
  \end{itemize}
\item
  Beispiele

  \begin{itemize}
  \tightlist
  \item
    sollten nach jeder Definition und nach jedem Satz auftachen
  \item
    dienen zur Verständlichkeit
  \end{itemize}
\end{itemize}

\hypertarget{zahlenbereiche}{%
\sssect{Zahlenbereiche}\label{zahlenbereiche}}

\begin{itemize}
\tightlist
\item
  Natürliche Zahlen
\item
  Ganze Zahlen

  \begin{itemize}
  \tightlist
  \item
    natürlich Zahlen sind teilmenge der ganzen Zahlen
  \end{itemize}
\item
  Rationale Zahlen

  \begin{itemize}
  \tightlist
  \item
    Gnaze Zahlen sind teilmenge der rationalen Zahlen
  \end{itemize}
\item
  Reele Zahlen

  \begin{itemize}
  \tightlist
  \item
    in \(\Re\) ist eine Gleichung nicht lösbar: \(x^{2} = -1\)
  \end{itemize}
\end{itemize}

\hypertarget{zahlenbereich-konstrurieren-mit}{%
\sssect{Zahlenbereich konstrurieren
mit:}\label{zahlenbereich-konstrurieren-mit}}

\begin{itemize}
\tightlist
\item
  enthält alle reelen Zahlen
\item
  \(x^{2} = -1\) soll lösbar sein
\item
  Rechengesetze aus der Menge der Reelen Zahlen sollen auch gelten
\item
  neue Berreich soll so klein wie möglich sein
\end{itemize}

\hypertarget{komplexe-zahlen}{%
\sssect{Komplexe Zahlen}\label{komplexe-zahlen}}

\hypertarget{def.}{%
\paragraph{Def.}\label{def.}}

\begin{itemize}
\tightlist
\item
  i mit der Eigenschaft \(i^{2} = i \cdot i := -1\) heißt imaginäre
  Einheit.
\end{itemize}

\hypertarget{bem.}{%
\paragraph{Bem.}\label{bem.}}

\begin{itemize}
\tightlist
\item
  \(i \in \mathbb{C}\)
\item
  \(\Re \in \mathbb{C}\)
\item
  \({a,b \in \Re => a+bi \in \mathbb{C}}\)
\end{itemize}

\hypertarget{def.-1}{%
\paragraph{Def.}\label{def.-1}}

\begin{itemize}
\tightlist
\item
  \(\mathbb{C} := {a+bi | a,b \in \Re}\) heißt Menge der komplexen
  Zahlen
\item
  Für \(\lbrace Z = a + bi \in \mathbb{C} \rbrace\) heißt
\item
  a: Realteil Re(Z)
\item
  b: Imaginärteil Im(Z)
\item
  Man nennt eine \(\overline{Z} := a - bi\) die zu Z konjugierte
  komplexe Zahl
\item
  reinimmaginäre Zahlen haben den Realteil 0
\item
  Komplexe Zahlen haben real und immaginärteil
\end{itemize}

\hypertarget{rechenoperation-mit-komplexen-zahlen}{%
\ssect{Rechenoperation mit komplexen
Zahlen}\label{rechenoperation-mit-komplexen-zahlen}}

\hypertarget{addition}{%
\sssect{Addition}\label{addition}}

\hypertarget{beispiel}{%
\paragraph{Beispiel}\label{beispiel}}

\((1+2i) + (3+4i) = 1 + 2i + 3+ 4i = = 4 + 6i\)

\hypertarget{mit-formelzeichen}{%
\paragraph{mit Formelzeichen}\label{mit-formelzeichen}}

\((a+bi) + (c+di) = (a+c) + (b+d)i\)

\hypertarget{allgemein}{%
\paragraph{Allgemein}\label{allgemein}}

\(a+b = (Re(a) + Re(b)) + (Im(a) + Im(b))i \forall \> a,b \in \mathbb{C}\)

\hypertarget{subtraktion}{%
\sssect{Subtraktion}\label{subtraktion}}

\hypertarget{mit-formelzeichen-1}{%
\paragraph{mit Formelzeichen}\label{mit-formelzeichen-1}}

\((a+bi) - (c+di) = (a-c) + (b-d)i\)

\hypertarget{allgemein-1}{%
\paragraph{Allgemein}\label{allgemein-1}}

\(a+b = (Re(a) - Re(b)) + (Im(a) - Im(b))i \forall \> a,b \in \mathbb{C}\)

\hypertarget{multiplikation}{%
\sssect{Multiplikation}\label{multiplikation}}

\hypertarget{mit-formelzeichen-2}{%
\paragraph{mit Formelzeichen}\label{mit-formelzeichen-2}}

\((a + bi) \cdot (c + di) =\) \$ac + adi + bci + bdi\^{}2 = \$
\((ac - bd) + (ad + bc)i\)

\hypertarget{division}{%
\sssect{Division}\label{division}}

\hypertarget{beispiel-1}{%
\paragraph{Beispiel:}\label{beispiel-1}}

\(\frac{(1+2i)}{(3+4i)} = \frac{1+2i}{3+4i} \cdot \frac{3-4i}{3-4i} = \frac{3+8+i(6-4)}{9+16} = \frac{11+2i}{25} = \frac{11}{25} + \frac{2i}{25}\)

\hypertarget{mit-formelzeichen-3}{%
\paragraph{mit Formelzeichen:}\label{mit-formelzeichen-3}}

\(\frac{a + bi}{c + di} \cdot \frac{c - di}{c - di} = \frac{ac + d + (ad + bc)i}{c^{2} + d^{2}}\)

\hypertarget{bemerkung}{%
\paragraph{Bemerkung}\label{bemerkung}}

\((\mathbb{C};+;\cdot)\) heißt Körper der konmplexen Zahlen

\hypertarget{gauuxdfsche-zahlenebene}{%
\ssect{Gauß'sche Zahlenebene}\label{gauuxdfsche-zahlenebene}}

\footnote{Quelle:
  \url{https://de.wikipedia.org/wiki/Gau\%C3\%9Fsche_Zahlenebene\#/media/File:Gau\%C3\%9Fsche_Zahlenebene.svg}}
\includegraphics{gausssche_zahlenebene.png}

Jeder Punkt der Gauß'schen Zahlenebene stellt eine komplexe Zahl dar.

\hypertarget{darstellungsformen-komplexer-zahlen}{%
\ssect{Darstellungsformen komplexer
Zahlen}\label{darstellungsformen-komplexer-zahlen}}

\(Z \in \mathbb{C}\)

\hypertarget{arithmetischekartesiche-darstellung}{%
\sssect{arithmetische/kartesiche
Darstellung}\label{arithmetischekartesiche-darstellung}}

\(Z = a + bi\)

\hypertarget{trigonometrische-darstellungpolarkoordinaten}{%
\sssect{trigonometrische
Darstellung/Polarkoordinaten}\label{trigonometrische-darstellungpolarkoordinaten}}

\(Z = r \cdot (cos(\phi) + i \cdot sin(\phi))\)

\hypertarget{konversion}{%
\paragraph{Konversion}\label{konversion}}

\(r = \sqrt{a^{2}+b^{b}}\) \(sin \phi = \frac{b}{r}\)
\(cos \phi = \frac{a}{r}\)

\hypertarget{eulerischeexponentielle-darstellung}{%
\sssect{Eulerische/exponentielle
Darstellung}\label{eulerischeexponentielle-darstellung}}

\(Z = r \cdot e^{i \cdot \phi}\)

\hypertarget{multiplikation-in-der-eulerischen-dartstellung}{%
\paragraph{Multiplikation in der Eulerischen
Dartstellung}\label{multiplikation-in-der-eulerischen-dartstellung}}

\((r_{1} \cdot e^{i \phi_{1}}) \cdot (r_{2} \cdot e^{i \phi_{2}}) = r_{1} \cdot r_{2} \cdot e^{i \cdot (\phi_{1} + \phi_{2})}\)

\hypertarget{division-in-der-eulerischen-darstellung}{%
\paragraph{Division in der Eulerischen
Darstellung}\label{division-in-der-eulerischen-darstellung}}

\(\frac{r_{1} \cdot e^{i \phi_{1}}}{r_{2} \cdot e^{i \phi_{2}}}\)
\(= \frac{r_{1}}{r_{2}} \cdot e^{i \cdot (\phi_{1} - \phi_{2})}\)
\end{document}
