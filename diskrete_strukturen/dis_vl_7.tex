\documentclass{../../meta/tudscript}
\begin{document}

\ssect{Definition}

Die Relation \(\leq \quad \subseteq \mathbb{N}\times \mathbb{N}\) ist
definiert durch \(m \leq n :\iff \exists k \in \mathbb{N}: n = m + k\)
Auch: \(m < n :\iff m \leq n \wedge m \neq n\)

\ssect{Satz}

Die Relation \(\leq\) ist eine totale Ordnungsrelation auf
\(\mathbb{N}\)

\sssect{Beweis}

\paragraph{Reflexivität:}

Für jedes \(n \in \mathbb{N}\) ist \(n = n+0\) und daher \(n \leq n\)

\paragraph{Transitivität}

Seien \(k,m,n \in \mathbb{N}\) mit \(k \leq m\) und \(m \leq n\). Dann
gibt es \(p,q \in \mathbb{N}\) mit \(m = k + p\) und \(n = m + q\). Es
folgt \(n = m + q = (k + p) + q = k + (p + q)\) und somit \(k \leq n\)

\paragraph{Antisymmetrie}

Seien \(m,n \in \mathbb{N}\) mit \(m \leq n\) und \(n \leq m\). Dann
gibt es \(p,q \in \mathbb{N}\) mit \(n = m + p\) und \(m = n + q\), also
\(n = n + p + q\). Nach Bemerkung 5.9.1 folgt \(0 = p + q\) und wegen
Bemerkung 5.9.2 dann \(p = q = 0\). Daher m = n.

Damit ist \(\leq\) eine Ordnungsrelation auf \(\mathbb{N}\). Noch zu
zeigen: \(\leq\) ist total. Wir zeigen für jedes \(n \in \mathbb{N}\)
gilt \(\forall m \in sN: m \leq n \vee n \leq m\).

\paragraph{\texorpdfstring{Beweis durch vollständige Induktion über
\(\mathbb{N}\)}}

\boxed{IA}: \(n = 0\). Für jedes \(m \in \mathbb{N}\) ist \(m = 0 + n\)
(Proposition 5.8.2) und daher \(0 \leq m\)

\boxed{IS}: Sei \(n \in \mathbb{N}\), sodass

Sei \(m \in sN\). Fallunterscheidung: Fall 1: \(m \leq n\). Dann ist
\(m \leq n + 1\) (wegen Transitiviät und \(n \leq n + 1\) )

Fall 2: \(n \leq m\). Dann gilt \(k \in \mathbb{N}\) mit \(m = n + k\).
Ist \(k=0\), so folgt \(m = n \leq n + 1\). Ist \(k \neq 0\), so gibt es
\(l \in sN\) mit \(k = l + 1\) (Üb!.). Somit ist
\(m = n + l + 1 = n + 1 + l\) also \(n + 1 \leq m\).

In jedem Fall ist \(m \leq n + 1 \vee n + 1 \leq m\) eine wahre Aussage.

\ssect{Definition}

Für natürliche Zahlen \(k,n,x_n,...x_{n+k} \in \mathbb{N}\) definieren
wir die Summe \(\displaystyle\sum_{i=n}^{n} x_i\) mittels Rekursion über
\(k\) wie folgt:

Für \(m,n \in \mathbb{N}\) mit \(m < n\) definieren wir:

\sect{Endliche Mengen}}

\ssect{Definition: endliche Menge}

Für \(n \in \mathbb{N}\) definieren wir
\([n] := \{ m \in \mathbb{N}\mid 0 < m \leq n \}\) (d.h.
\([n] = \{1,...,n\}\) und insbesondere \([0] = \emptyset\))

\ssect{Lemma: Abbildungen in endliche Mengen}

Sei \(n \in \mathbb{N}\). Jede injektive Abbildung
\(f:[n] \rightarrow [n]\) ist surjektiv.

\sssect{Beweis}

Vollständige Induktion über \(n \in \mathbb{N}\)

\boxed{IA}: \(n = 0\). Dann \([n] = [0] = \emptyset\). Es gibt genau
eine Abbildung \(f : \emptyset \rightarrow \emptyset\), und diese ist
surjektiv.

\boxed{IS}: Annahme: Die Aussage ist wahr für \(n \in \mathbb{N}\). Zu
zeigen: Die Aussage ist wahr für \(n + 1\). Sei dazu
\(f: [n+1] \rightarrow [n + 1]\). Wir zeigen \([n+1] = im(f)\).

\paragraph{\texorpdfstring{Hilfsaussage:
\(n + 1 \in im(f)\)}}

\paragraph{Beweis (durch Widerspruch)}

Wäre \(n + 1 \not\in im(f)\), also \(im(f) \subseteq [n]\) dann wäre die
Einschränkung \(f\mid_{[n]}: [n] \rightarrow [n], k \mapsto f(k)\)
injektiv, also surjektiv nach \boxed{IV}. Wegen \(f(n+1) \in [n]\) gäbe
es ein \(k \in [n]\) mit \(f(k) = f(n+1)\). Widerspruch zur Injektivirät
von f (da \(k \neq n+1\)). \(\hfill\square\)

Also \(n+1 \in im(f)\), d.h. es gibt \(k \in [n+1]\) mit
\(n+1 = f(k)\quad\). Fallunterscheidung:

Fall 1: \(k = n+1\) Dann \(f\mid_[n]: [n] \rightarrow [n]\)
(wohldefiniert und injektiv) also surjektiv nach \boxed{IV}, somit
\([n] \subseteq im(f)\)

Fall 2: \(k \neq n + 1\). Dann ist \(g:[n] \rightarrow [n]\) mit
\(g_i:= \begin{cases} f(i) & \text{ falls } i \neq k \\ f(n+1) & \text{ falls } i = k \end{cases}\)
injektiv, also surjektiv nach \boxed{IV}. Somit
\([n] \subseteq im(g) \subseteq im(f)\). In jedem Fall ist also
\([n \subseteq im(f)]\). Da auch \(n + 1 \in im(f)\) ist \(f\)
surjektiv.


\includegraphics[scale=0.3]{assets/point_diagrams.png}
\caption{Mengendiagramme}


\ssect{Folgerung}

Seien \(m,n \in \mathbb{N}\). Dann gilt:

\begin{enumerate}
\def\labelenumi{\arabic{enumi})}

\item
  \(n \leq m \iff \exists f: [n] \rightarrow [n]\) injektiv
\item
  \(n = m \iff \exists f: [n] \rightarrow [n]\) bijektiv
\end{enumerate}

\sssect{Beweis}

\paragraph{1)}

``\(\implies\)'': Klar, denn ist \(n \leq m\), dann
\([n] \subseteq [m]\) und \([n] \rightarrow[m] k \mapsto k\) injektiv
nach Beispiel 4.4, 1)

``\(\impliedby\)'': Sei \(f[n] \rightarrow[m]\) injektiv. Annahme:
\(n \not\leq m\), d.h. m \textless{} n nach Satz 5.14. Damit ist
\(g:[n] \rightarrow[n], k \mapsto f(k)\) injektiv, aber wegen
\(im(g) = im(f) \subseteq [m] \subsetneq [n]\) nicht surjektiv.
Widerspruch zum Lemma 6.2. Also \(n \leq m\)
\begin{tikz}
\begin{tikzcd}
X \arrow[r, "f"]  & X \arrow[d, xshift=0.5ex, "\text{$g^{-1}$}"] \\
\text{$[n]$} \arrow[u, "g"] \arrow[r, dashrightarrow, "h"] &
\text{$[n]$} \arrow[u, xshift=-0.5ex, "g"]
\end{tikzcd}
\end{tikz}

\end{document}
