\documentclass{../../meta/tudscript}
\begin{document}

\ssect{Axiom (Unendlichkeit)}

Es existiert eine Menge M mit \(\emptyset \in M\) und
\(\forall x \in M: x^+ \in M\)

\ssect{Definition: Menge der natürlichen Zahlen}

Sei M eine Menge, wie in Axiom 5.2. Die Menge der natürlichen Zahlen ist
definiert als:

\begin{flalign*}\mathbb{N} = \cap \{S \subseteq M | \emptyset \in S \wedge \forall x \in S: x^+ \in S\}\end{flalign*}

Diese Definition von \(\mathbb{N}\) ist unabhängig von der Wahl von M.


\ssect{Satz ``Peano Axiome''}

Für die Menge \(\mathbb{N}\) gelten folgende Aussagen:

\begin{enumerate}
\def\labelenumi{\arabic{enumi}.}
\item
  \(0 \in \mathbb{N}\)
\item
  \(\forall i \in \mathbb{N}: i^+ \in \mathbb{N}\)
\item
  \(\forall i \in \mathbb{N}: i^+ \neq 0\)
\item
  \(\forall m,n \in \mathbb{N}: m^+ = n^+ \rightarrow m=n\)
\item
  Ist \(S \subseteq \mathbb{N}\) und \(0 \in S\) und
  \(\forall n \in S: n^+ \in S\), dann \(S = \mathbb{N}\) (Prinzip der
  vollständigen Induktion)
\end{enumerate}


\sssect{Bemerkung zu 5.4.1 bis
5.4.4:}

Die Abbildung \(f: \mathbb{N} \rightarrow \mathbb{N}: n \mapsto n^+\)
ist injektiv aber nicht surjektiv.

\sssect{Beweis von 5.4}

Sei M Menge, wie in 5.2 und
\(\mathscr{C}= \{S \leq M | \emptyset \in S \wedge \forall x \in S, x^+ \in S\}\),
dann \(\mathbb{N} = \cap \mathscr{C}\)

1,2) Klar, denn diese Aussagen gelten für alle Mengen in \(\mathscr{C}\)

\begin{enumerate}
\def\labelenumi{\arabic{enumi})}
\setcounter{enumi}{2}
\item
  Für alle \(n \in \mathbb{N}\) ist \(n \in n^+\), daher
  \(n^+ + \emptyset = 0\)
\item
  Sei \(S \subseteq M\) mit \(\emptyset \in S\) und
  \(\forall n \in S: n^+ \in S\). Dann ist \(S \in \mathscr{C}\) und
  daher
  \(\mathbb{N} = \cap \mathscr{C}\subseteq S \subseteq \mathbb{N}\),
  also \(S = \mathbb{N}\)
\item
  Wir benötigen folgende Hilfsaussagen: Für alle
  \(x,y,z \in \mathbb{N}\) gilt:
\end{enumerate}

\begin{enumerate}
\def\labelenumi{\arabic{enumi}.}
\item
  \(x \in y \rightarrow y \nsubseteq x\)
\item
  \(x \in y \wedge y \in z \rightarrow x \in z\)
\end{enumerate}

\paragraph{Beweis}

\begin{enumerate}
\def\labelenumi{\roman{enumi})}

\item
  Betrachte
  \(S := \{y \in \mathbb{N} | \forall x \in y y \nsubseteq x \}\)
  Offenbar \(0 \in S\)
\end{enumerate}

Zu zeigen:

Ist \(y \in S\), dann auch \(y^+ \in S\). Sei also \(y \in S\)

\begin{itemize}
\item
  Dann: für jedes \(x \in y^+ = y \cup \{y\}\) ist \(x \in y\) und daher
  \(y \nsubseteq x\), da \(y \in S\)
\item
  oder: x = y und daher \(y \nsubseteq x\) da \(y \in S\), in beiden
  Fällen \(y^+ \nsubseteq x\). Somit \(S = \mathbb{N}\) nach 5..
\end{itemize}

\begin{enumerate}
\def\labelenumi{\roman{enumi})}
\setcounter{enumi}{1}

\item
  Betrachte
  \(T := \{z \in \mathbb{N} | \forall y \in z \forall x \in y: x \in z\}\)
\end{enumerate}

Wieder \(0 \in T\)

Behauptung: Ist \(z \in T\), dann auch \(z^+ \in T\). Sei also
\(z \in T\), dann für jedes

\begin{enumerate}
\def\labelenumi{\arabic{enumi}.}
\item
  \(y \in z\) (Fall 1) oder
\item
  \(y=z\) (Fall 2)
\end{enumerate}

Fall 1: Dann gilt für jedes \(x \in y\) eben \(x \in z\) (weil
\(x \in T\)) und somit \(x \in z^+\)

Fall 2: Hier ist für jedes \(x \in y\) eben
\(x \in y \in z \subseteq z^+\)


\paragraph{Zum Beweis von 4}

Seien \(m,n \in \mathbb{N}\) mit \(m^+ = n^+\). Wegen
\(n \in n^+ = m^+ = m \cup \{m\}\) gilt \(n \in m\) oder \(n = m\).
Wegen \(m \in m^+ = n^+ = n \cup \{n\}\) gilt \(m \in n\) oder \(n = m\)

Annahme \(n \neq m\)

Dann folgt \(n \in m\) und \(m \in n\). Nach i) folgt \(n \in n\) und
nach ii) \(n \nsubseteq n\).

Widerspruch! Also n = m


\ssect{Bemerkung: vollständige
Induktion}

Das Prinzip der vollständigen Induktion stellt eine Beweismethode dar.
Wie man eine Aussage der Form \(\forall n \in \mathbb{N}: P(n)\)
beweisen, so zeigt man:

\begin{itemize}
\item
  Induktionsanfang IA: P(0) ist wahr
\item
  Induktionsschritt IS:
\end{itemize}

\begin{flalign*}\forall n \in \mathbb{N} P(n) \rightarrow P(n^+)\end{flalign*}

Hierbei bezeichnet man P(n) als Induktionsvoraussetzung IV bzw. als
Induktionshypothese.

Zugehörige Definitionsmethode: vollständige Rekursion


\ssect{Definition: Addition}

Sei \(m \in \mathbb{N}\). Dann

\begin{itemize}
\item
  \(m + 0 = m\)
\item
  \(m + n^+ = (m + n)^+\) für jedes \(n \in \mathbb{N}\)
\end{itemize}


\ssect{Bemerkung}

Wegen \(1 = 0^+\) ist \(m+1= m + 0^+ = (m + 0)^+ = m^+\) für alle
\(n \in \mathbb{N}\)


\ssect{Proposition}

Für alle \(k,m,n \in \mathbb{N}\) gilt

\begin{enumerate}
\def\labelenumi{\arabic{enumi}.}
\item
  Assioziativität: \((k + m) + n = k + (m + n)\)
\item
  Neutralität der Null: \(0 + n = n = n + 0\)
\item
  Kommutativität: \(m + n = n + m\)
\end{enumerate}


\sssect{Beweis}

\begin{enumerate}
\def\labelenumi{\arabic{enumi}.}

\item
  Vollständige Induktion über \(n \in \mathbb{N}\)
\end{enumerate}

IA: \(n = 0\). Dann \((k + m) + n = k + m = k + (m + 0)\) für alle
\(k,m \in \mathbb{N}\)

IS:Sei \(n \in \mathbb{N}\), sodass

Dann folgt:

\((k + m) + n^+ = ((k + m) + n)^+ = (k + (m + n))^+ = k + (m + n)^+ = k + (m + n^+)\)
für alle \(k,m \in \mathbb{N}\)

\begin{enumerate}
\def\labelenumi{\arabic{enumi}.}
\setcounter{enumi}{1}

\item
  Nach Definition \(n+ 0 = n\) für \(n \in \mathbb{N}\) Rest durch
  vollst. Induktion über \(n \in \mathbb{N}\)
\end{enumerate}

IA: n = 0. Offenbar gilt 0 + 0 = 0

IS: Sei \(n \in \mathbb{N}\), sodass \(0+n=n\)

Dann folgt \(0+n^+ = (0+n)^+ = n^+\)

\begin{enumerate}
\def\labelenumi{\arabic{enumi}.}
\setcounter{enumi}{2}

\item
  Übung. (es ist sinnvoll zuerst \(m^+ + n = (m+n)^+\))
\end{enumerate}


\ssect{Bemerkung}

Für alle \(k,m,n \in \mathbb{N}\) gilt:

\begin{enumerate}
\def\labelenumi{\arabic{enumi}.}
\item
  Kürzbarkeit: \(m + k = n + k \rightarrow m = n\)
\item
  Nullsummenfreiheit: \(m + n = 0 \rightarrow m = n = 0\)
\end{enumerate}

Beweis mittels vollständiger Induktion möglich.


\ssect{Definition:
Multiplikation}

Sei \(m \in \mathbb{N}\). Dann:


\ssect{Proposition}

Für alle \(k,w,n \in \mathbb{N}\) gilt.

\begin{enumerate}
\def\labelenumi{\arabic{enumi}.}
\item
  Assoziativität: \((k \cdot m) \cdot n = k \cdot (m \cdot n\)
\item
  Neutralität der Eins: \(1 \cdot n = n = n \cdot 1\)
\item
  Absorbtion der Null: \(0 \cdot n = 0 = n \cdot 0\)
\item
  Kommutativität: \(m \cdot n = n \cdot m\)
\item
  Distributivität: \(k \cdot (m + n) = k \cdot m + k \cdot n\)
\end{enumerate}

Beweis mittel vollst. Induktion möglich.


\ssect{Definition}

Sei \(m \in \mathbb{N}\). Dann:

\begin{flalign*}m^0 := 1\end{flalign*}
\begin{flalign*}m^{n+1} := m^n \cdot m\end{flalign*}

für alle \(n \in \mathbb{N}\)


\end{document}
