\documentclass{../../meta/tudscript}
\begin{document}
\sssect{Ergänzung zu Bemerkung(7.4)}
Konvention zur Schreibweise: Ist \((M,+)\) eine Gruppe, dann bezeichnet
man das zu einem \(x \in M\) inverse Element in \((M,+)\) üblicherweise
mit \(-x\), und man schreibt \(x-y:=x+(-y)\) für \(x,y \in M\)

\ssect{Definition}

Eine Halbgruppe \((M,*)\) heißt \underline{kommutativ (abelsch)} falls
gilt: \(\forall x,y \in M: x*y=y*x\)

\ssect{Beispiele}

\begin{enumerate}

\item
  \((\mathbb{N},+)\) und \((\mathbb{N},\cdot )\) sind kommutative
  Monoide. (Propositionen 5.8 und 5.11)
\item
  Sei \(X\) Menge. Die Menge aller Abbildungen von \(X\) nach \(X\)
  zusammen mit der Verkettung ist \((X^X,\circ )\) ist ein Monoid.
  (Lemma 4.9) Das neutrale Element von \((X^X ,\circ )\) ist
  \(id_x :X \rightarrow X\)
\item
  Sei \(X\) Menge und \(Sym(X):=\{ f \in X^X | f \text{ bijektiv} \}\).
  Dann ist \((Sym(X),\circ )\) eine Gruppe, die
  \underline{(volle) Symmetrische Gruppe} über der Menge \(X\).
  (Proposition 4.12)
\end{enumerate}

\ssect{Definition}

Ein \underline{Semiring} (Halbring) ist ein Tripel \((R,+,\cdot )\)
bestehend aus einer Menge \(R\) sowie den Abbildungen

\(+:R \times R \rightarrow R, (x,y) \mapsto x+y\),
\(\cdot : R \times\ R \rightarrow R, (x,y) \mapsto x\cdot y\),

so dass gilt:

\begin{enumerate}

\item
  \((R,+)\) ist kommutatives Monoid (dessen neutrales Element wir als
  \underline{Nullelement} \(0=0_R\) bezeichnen),
\item
  \((R,*)\) ist Monoid (dessen neutrales Element wir als Einselement
  \(1=1_R\) bezeichnen),
\item
  \(\forall x \in R: 0 \cdot x = 0 = x \cdot 0\) (Null ist
  \underline{Absorbierend})
\item
  \(\forall x,y,z \in R: x \cdot(y+z)=x \cdot y + x \cdot z \land (x+y) \cdot z = x \cdot z + y \cdot z\)
  (\underline{Distributivität})
\end{enumerate}

Ein Semiring \((R,+,*)\) heißt:

\underline{Ring} \(:\iff (R,+)\) ist Gruppe \underline{kommutativ}
\(:\iff (R,\cdot )\) ist kommutativ

Ein \underline{Körper} ist ein kommutativer Ring \((R,+,\cdot )\), so
dass * \(0_K \neq 1_K\) * \(\forall x \in K \setminus \{0\}\)
\(\exists y \in K: 1=y \cdot x (=xy)\) (\(K\) ohne \(0\) ist mit
\(\cdot\) eine Gruppe)

\ssect{Beispiele}

\begin{enumerate}

\item
  \((\mathbb{N},+,\cdot )\) ist kommutativer Semiring (weil keine
  negativen Zahlen)
\item
  Sei \(X\) Menge. Dann \((\wp(X),\cup ,\cap )\) kommutativer Semiring
  (Proposition 2.14) mit Nullelement \(\varnothing\) und Einselement
  \(X\)
\item
  Ausblick: \((\mathbb{Z},+,\cdot )\) ist kommutativer Ring,
  \((\mathbb{Q},+,\cdot )\) ist Körper
\end{enumerate}

\sect{Ganze und Rationale Zahlen}

\ssect{Lemma}
Sei \(X:= \mathbb{N}\times \mathbb{N}\) Dann ist die Relation
\(\sim \subseteq X \times X\) mit \((a,b) \sim (a',b'):\iff a+b'=a'+b\)
eine Äquivalenzrelation auf \(X\).

\sssect{Beweis}

\paragraph{Reflexivität + Symmetrie: Trivial siehe Übung!}

\paragraph{Transitivität:}

Seien \((a,b),(a',b'),(a'',b'') \in X\) mit \((a,b) \sim (a',b')\) und
\((a',b') \sim (a'',b'')\) d.h. \(a+b'=a'+b\) und \(a'+b''=a''+b'\) Dann
\(a+b''+b'=a'+b''+b=a''+b'+b\) und daher \(a+b''=a''+b\) wegen
Kürzbarkeit (Bemerkung 5.9,1) Also \((a,b) \sim (a'',b'')\)

\ssect{Definition}

Wir bezeichnen
\(\mathbb{Z}:= ( \mathbb{N}\times \mathbb{N})_{/\sim} = \{ [ (a,b) ]_\sim | a,b \in \mathbb{N}\}\)
als die Menge der \underline{ganzen Zahlen}
(\(\mathbb{N}\neq \mathbb{Z}\) weil Datentyp anders)

\ssect{Bemerkung}

Die Abbildung
\(\mathbb{N}\rightarrow \mathbb{Z}, a \mapsto [(a,o)]_\sim\) ist
injektiv. Wir identifizieren eine natürliche Zahl \(a \in \mathbb{N}\)
mit ihrem Bild \([(a,0)]_\sim \in \mathbb{Z}\) (d.h. wir schreiben
\(a=[(a,0)]_\sim\) ) und fassen so \(\mathbb{N}\) als Teilmenge von
\(\mathbb{Z}\) auf. \(\mathbb{N}\) und \(\mathbb{Z}\) werden
\underline{nicht} durch Größe unterschieden.

\ssect{Satz: $\bZ$ als kommutative Gruppe}

Die Menge \(\mathbb{Z}\) bildet mit der \underline{Addition}
\(+ : \mathbb{Z}\times \mathbb{Z}\rightarrow \mathbb{Z}\), definiert
durch \([(a,b)]_\sim + [(c,d)]_\sim := [(a+c, b+d)]_\sim\) für alle
\(a,b,c,d \in \mathbb{N}\) eine kommutative Gruppe.

\sssect{Beweis}

Wohldefiniert (d.h. Unabhängig der Definition von der Wahl der
Repräsentanten):

Seien \((a,b),(a',b'),(c,d),(c',d') \in \mathbb{N}\times \mathbb{N}\)
Ist \((a,b) \sim (a',b')\) und \((c,d) \sim (c',d')\), d.h.
\(a+b'=a'+b\) und \(c+d'=c'+d\), dann folgt
\(a+c+b'+d'=a'+b+c'+d=a'+c'+b+d\), und somit
\((a+c,b+d) \sim (a'+c',b'+d')\), d.h.
\([(a+c.b+d)]_\sim = [(a'+c',b'+d')]_\sim\)

Nach Proposition 5.8. ist \((\mathbb{N},+)\) kommutatives Monoid, mit
neutralem Element \(0\)

Es folgt: \(\mathbb{Z},+)\) ist kommutatives Monoid mit neutralem
Element \(0_\mathbb{Z}= [(0,0)]_\sim\).

\paragraph{Beispiel}

Assoziativität: Für alle \(a,b,c,d,e,f \in \mathbb{N}\) gilt

\([(a,b)] + ( [(c,d)] + [(e,f)] ) = [(a,b)] + [(c+e,d+f)]\)
\(= [(a+(c+e),b+(d+f))] = [((a+c)+e,(b+d)+f)]\)
\(= [(a+c,b+d)] + [(e,f)] = ( [(a,b)] + [(c,d)] ) + [(e,f)]\)

\sssect{Weiter zu Beweisen}

Existenz inverser Elemente: Zu jedem \([(a,b)] \in \mathbb{Z}\) ist
\([(b,a)]\) ein inverses Element, denn:
\textit{\textbf{\textexclamdown EinfacherTrick\textsuperscript{TM}!}}

\([(b,a)] + [(a,b)] = [(a+b,a+b)] = [(0,0)] = 0_\mathbb{Z}\) (Denn jedes
identische Paar \([(x,x)] \in \mathbb{Z}= 0_\mathbb{Z}\))

\(\hfill \square\)


\ssect{Bemerkung (vgl. Bem.
7.4)}

\begin{enumerate}

\item
  Der obige Beweis zeigt: \(-[(a,b)] = [(b,a)]\) für alle
  \(a,b \in \mathbb{N}\) Es folgt: \(-a = -[(a,0)] = [(0,a)]\) für alle
  \(a \in \mathbb{N}\) Für jedes \(z=[(a,b)] \in \mathbb{Z}\) erha\dots
\end{enumerate}

\sssect{Nachtrag zur Bemerkung 8.5}
\begin{enumerate}

\item
  für jede ganze Zahl \(z = [(a,b)] \in \mathbb{Z}\) erhalten wir eine
  Darstellung der Form
\end{enumerate}

\( z = [(a,b)] = [(a,0)] + {[}(0,b){]} = a - b\)

und es gilt \((a,b) ~ (a', b') \iff a-b = a' - b'\)

\begin{enumerate}

\item
  Für jedes \(z \in \mathbb{Z}\) gibt es ein \(n \in \mathbb{N}\) mit
  \(z = n\) oder \(z = -n\)
\end{enumerate}

\sssect{Beweis}

Sei \(z =p [(a,b)] \in \mathbb{Z}\). Nach Satz 5.14 ist
\(a \leq b aber b \geq a\), aber existiert \(n \in \mathbb{N}\) mit
\(b = a + n\) oder \(a = b + n\), und somit

\begin{flalign*}z = [(a,b)] = [(a, a+n)] = [(0,n)] = -n\end{flalign*}

oder

\begin{flalign*}z = [(a,b)] = [(b+n,b)] = [(n,0)] = n\end{flalign*}

\ssect{\texorpdfstring{Satz Multiplikation in \(\mathbb{Z}\)}}
Die Menge \(\mathbb{Z}\) bildet mit der Addition
\(+: \mathbb{Z}\times \mathbb{Z}\rightarrow \mathbb{Z}\) aus Satz 8.4
und der Multiplikation
\(\cdot : \mathbb{Z}\times \mathbb{Z}\rightarrow \mathbb{Z}\), definiert
durch

\begin{flalign*}[(a,b)] \cdot [(c,d)] := [(ac+bf)],ad+bc]\end{flalign*}

für alle \(a,b,c,d \in \mathbb{N}\), einen \textbf{kommutativen Ring}.


\ssect{Bemerkung}

Nach obiger Definition gilt:

\begin{flalign*}(a-b) \cdot (c-d) = (ac + bd) -  (ad - bc)\end{flalign*}

für alle \(a,b,c,d \in \mathbb{N}\).


\sssect{Beweis von 8.6}

Wohldefiniert (d.h Unabhängigkeit der Definition von der Wahl der
Repräsentanten):

Für \(a,b,c,d \in \mathbb{N}\) sei
\((a,b) \cdot (c,d) := (ac+bd, ad+bc)\)

Seien \(a,b,c,d,a',b',c',d' \in \mathbb{N}\) mit \((a,b) ~ (a',b')\) und
\((c,d) \ (c',d')\). Zu zeigen:
\( (a,b) \cdot (c,d) \sim (a',b') \cdot (c',d') \)

Wir beweisen: (1): \((a,b) \cdot (c,d) ~ (a',b') \cdot (c',d')\)

(2): \((a',b') \cdot ~ (a',b') \cdot (c',d')\)

Dann folgt aus (1) und (2) wegen Transitivität:

\begin{flalign*}(a,b) \cdot (c,d) ~ (a',b') \cdot (c',d')\end{flalign*}

\textbf{Beweis zu (1)}

Wegen \((a,b) ~ (a',b')\) ist \(a + b' = a' + b\) und daher

\begin{flalign*}ac+ bd + a'd + 'c = (a+b')c + (a'+b)d\end{flalign*}

\begin{flalign*}= (a' + b)c  + (a+b')d = a'c + b'd + ad + bc\end{flalign*}

d.h

\begin{flalign*}(a,b) \cdot (c,d) ~ (a',b') \cdot (c,d)\end{flalign*}

\textbf{Beweis zu (2)}

Analog: Üb!

Mit Satz 8.4 und Proposition 5.11 folgt nun leicht, dass
\((\mathbb{Z}, +, \cdot)\) ein kommutativer Ring ist


\ssect{Bemerkung:
``Nullteilerfreiheit''}

Der Ring \((\mathbb{Z}, +, \cdot)\) ist ``nullteilerfrei'', d.h für alle
\(x,y \in \mathbb{Z}\) gilt:

\begin{flalign*}x \cdot y \Rightarrow x = 0 \lor y = 0\end{flalign*}


\ssect{Lemma}

Sei \(X := \mathbb{Z}\times (\mathbb{Z}\\ \{0\})\). Dann ist die
Relation: eine Äquivalenzrelation auf X.


\sssect{Beweis}

Reflexivität + Symmetrie: Üb!

Transitivität:

Seien \((x,y),(x'y'),(x'',y'') \in X\) mit \((x,y) ~ (x',y')\) und
\((x',y') ~ (x'',y'')\), d.h.\(xy' = x'y\) und \(x'y'' = x''y'\). Dann
\(xy''y' = x'yy'' = x''y'y = x''yy'\), also \((xy'' - x''y) y' = 0\)
Wegen \(y' \neq 0\) und Nullteilerfreiheit von
\((\mathbb{Z}, +, \cdot)\) folgt \(xy'' = x''y\).

Somit \((x,y) ~ (x'',y'')\).


\ssect{Definition: Rationale
Zahlen}

Wir bezeichnen:

als die Menge der rationalen Zahlen.


\ssect{\texorpdfstring{Bemerkung: \(\mathbb{Z}\) als Teilmenge
von
\(\mathbb{Q}\)}}

Die Abbildung \(\mathbb{Z}\rightarrow \mathbb{Q}, x \mapsto [(x,1)]\)
ist injektiv. Wir identifizieren im Folgenden eine ganze Zahl
\(\mathbb{Z}\) mit ihrem Bild \([(x,1)] \in \mathbb{Q}\) wir schreiben:

\begin{flalign*}x =  [(x,1)]\end{flalign*}

und fassen so \(\mathbb{Z}\) als Teilmenge von \(\mathbb{Q}\) auf.


\ssect{Satz}

Die Menge \(\mathbb{Q}\) bildet mit der Addition
\(+: \mathbb{Q}\times \mathbb{Q}\rightarrow \mathbb{Q}\) und der
Multiplikation
\(\cdot : \mathbb{Q}\times \mathbb{Q}\rightarrow \mathbb{Q}\) definiert
durch \([(x,y)] + [(z,w)] := [(xq+yz,yw)]\)

\([(x,y)] \times [(z,w)] := [(xz,yw)]\)

für \(x,y,z,w \in \mathbb{Z}\) mit \(y \neq 0\) und \(w \neq 0\), einen
Körper.

\textbf{Beweis}

Wohldefiniertheit (d.h Repräsentantenunabhängigkeit)

Sind
\((x,y),(x',y'),(z,w),(z',w') \in \mathbb{Z}\times (\mathbb{Z} \setminus \{0\})\)
mit \((x,y) ~ (x',y')\) und \((z,w) ~ (z',w')\), also \(xy' =x'y\) und
\(zw' = z'w\), dann folgt \((xw+yz)y'w' = (x'w'+y'z')yw\),
\((xz)(y'w') = (x'z')(yw)\).

Somit \(((xw+yz,yw) ~ (x'w'+y'z',y'w')\) und \((xz,yw) ~ (x'z',y'w')\)

(Beatchte auch, dass \(yw + 0\) für alle \(yw \in \mathbb{Z} \setminus \{0\}\)
nach Bemerkung 8.8) Mithilfe von Satz 8.6 folgert man leicht, dass
\((\mathbb{Q},+, \cdot)\) ein kommutativer Ring mit Einselement
\([(1,1)]_{~}\) ist.

Existenz inverser Elemente für die Multiplikation: Für alle
\(x,y \in \mathbb{Z} \setminus \{0\}\) ist

\begin{flalign*}[(x,y)] \cdot [(y,x)] = [(xy,xy)] = [(1,1)]\end{flalign*}

q.e.d.


\ssect{Bemerkung (vgl. Bemerkung
8.11)}

Für eine ganze Zahl \(x \in \mathbb{Z} \setminus \{0\}\) schrieben wir

\begin{flalign*}\frac{1}{x} := x^{-1} = [(x,1)] = [(1,x)]\end{flalign*}

Für jede rationale Zahl \(q = [(x,y)] \in \mathbb{Q}\) erhalten wir eine
Darstellung der Form:

\begin{flalign*}q = [(x,1)] \cdot [(1,y)] = x \cdot \frac{1}{y} := \frac{x}{y}\end{flalign*}

mit dieser Notation gilt für die Operation des Körpers
\((\mathbb{Q}, +, \cdot)\):

\begin{flalign*}\frac{x}{y} + \frac{z}{w} = \frac{x+w+yz}{yw} und \frac{x}{y} \cdot \frac{z}{w} = \frac{xz}{yw}\end{flalign*}


\sect{§9 Elementare Zahlentheorie}


\ssect{Definition}

Wir definieren die Relation : \(|: \mathbb{Z}\times \mathbb{Z}\) wie
folgt: (gerader senkrechter Strich (it's not a pipe)) (a teilt b)


\ssect{Proposition}

\begin{enumerate}
\def\labelenumi{\arabic{enumi}.}

\item
  Die Relation \textbar{} ist reflexiv und transitiv (auf
  \(\mathbb{Z}\))
\item
  Für alle \(a,b,d,x,y \in \mathbb{Z}\) gilt: 
\end{enumerate}

\textbf{Beweis} (1) Üb!

\begin{enumerate}
\def\labelenumi{(\arabic{enumi})}
\setcounter{enumi}{1}

\item
  Gelte d\textbar{}a und d\textbar{}b, d.h es gilt
  \(k,l \in \mathbb{Z}\) mit \(a = k \cdot d\) und \(b = l \cdot d\).
  Dann folgt:
\end{enumerate}

\begin{flalign*}xa + yb = xkd + yld = (xk+yl)d)\end{flalign*}

somit d\textbar{}(xa+yb)


\ssect{Bemerkung}

\begin{enumerate}
\def\labelenumi{\arabic{enumi}.}

\item
  \(\{(a,b) \in \mathbb{N}\times \mathbb{N}| a | b \}\) ist eine
  Ordnungsrelation auf \(\mathbb{N}\)
\item
  Für alle \(a,b \in \mathbb{N}\) gilt:
  \(a|b \iff \exists c \in \mathbb{N}: b = a \cdot c\)
\end{enumerate}


\ssect{Definition: Teiler natürlicher
Zahlen}

Für eine \(n \in \mathbb{N}\) heißt
\(T(n) := \{d \in \mathbb{N}| d|n\}\) die Menge der \textbf{Teiler} von
n.


\ssect{Bemerkung}

\begin{enumerate}
\def\labelenumi{\arabic{enumi}.}

\item
  Für jedes \(n\in \mathbb{N}\\ \{0\}\) ist
  \(T(n) \subseteq \{1,...,n\}\) und daher T(n) endlich
\item
  \(T(0) = \mathbb{N}\)
\end{enumerate}


\ssect{Definition}

Sei (M, \(\leq\)) eine geordnete Menge, d.h ein Paar bestehend auseiner
Menge M und einer Ordnungsrelation \(\leq \subseteq M \times M\)

Sei \(A \subseteq M\). Dann heißt \(x \in M\)

\begin{itemize}
\item
  \textbf{untere Schranke} von A (bzgl. \(\leq\))
  \(: \iff \forall a \in A: x \leq A\)
\item
  \textbf{obere Schranke} von A (bzgl. \(\leq\))
  \(: \iff \forall a \in A: a \leq x\)
\item
  \textbf{Infimum} von A \(: \iff\)
\end{itemize}

\begin{enumerate}
\def\labelenumi{\arabic{enumi}.}

\item
  x ist untere Schranke von A
\item
  \(\forall y \in M:\) untere Schranke von A \(\Rightarrow y \leq x\)
\end{enumerate}

\begin{itemize}

\item
  \textbf{Supremum} von A
\end{itemize}

\begin{enumerate}
\def\labelenumi{\arabic{enumi}.}

\item
  x ist obere Schranke von A
\item
  \(\forall y \in M:\) obere Schranke von A \(\Rightarrow x \leq y\)
\end{enumerate}

\begin{itemize}
\item
  \textbf{kleinstes Element} von A \(: \iff x \in A \land x\) untere
  Schranke von A.
\item
  \textbf{größtes Element} von A \(: \iff x \in A \land\) obere Schranke
  von A.
\end{itemize}

Wir nennen \(\leq\) eine \textbf{Wohlordnung} auf M und nennen
\((M, \leq)\) eine \textbf{wohlgeordnete Menege}, falls gilt: 1.
\(\leq \subseteq M \times M\) ist total 2. jede nicht-leere Teilmenge
von M besitzt ein kleinstes Element bzgl \(\leq\).

Anmerkung des Chronisten: Ist nur für unendliche Mengen relevant.


\ssect{Bemerkung}

Sei \((M, \leq)\) geordnete Menge \(A \subseteq M\).

\begin{enumerate}
\def\labelenumi{\arabic{enumi}.}
\item
  A besitzt höchstens ein kleinstes Element
\item
  A besitzt höchstens ein größtes Element
\item
  A besitzt höchstens ein Infimum
\item
  A besitzt höchstens ein Supremum
\end{enumerate}

bzgl \(\leq\)

(Beweis: Sind \(x, x' \in M\) Infima von A, dann \(x \leq x'\) und
\(x' \leq x\), also \(x = x'\))

\begin{enumerate}
\def\labelenumi{\arabic{enumi}.}
\setcounter{enumi}{1}

\item
  Ist \(x \in A\) kleinstes (größtes) Element von A bzgl. \(\leq\), dann
  x Infimum (Supremum) von A bzgl. \(\leq\)
\item
  Ist \(\leq\) total, dann besitzt jede \textbf{endliche} nichtleere
  Teilmenge von M ein kleinstes (größtes) Element bzgl. \(\leq\)
\end{enumerate}

(Beweis durch vollständige Induktion über die Kardinalität einer
endlichen nichtleeren Teilmenge von M.)


\ssect{Satz: Wohlordnungssatz}

\((\mathbb{N}, \leq)\) ist wohlgeordnet.


\sssect{Beweis}

Nach Satz 5.14: \(\leq\) ist totale Ordnungsrelation auf \(\mathbb{N}\)

Sei \(\emptyset \neq A \subseteq \mathbb{N}\) Wähle \(n \in A\). Die
Menge \(B := \{a \in A | a \leq n\}\) ist endliche und nicht leer,
besitzt daher ein kleinstes Element \(b \in B\) (Bemerkung 9.7,3) . Es
folgt b kleinstes Element von A. \(\hfil \quad \linebreak\)


\ssect{Lemma: Division mit
Rest}

Für alle \(a,b \in \mathbb{N}\) und \(b \neq 0\) existieren
\(q,r \in \mathbb{N}\), sodass \(a = q \cdot b + r\) und \(r < b\).


\sssect{Beweis}

Sei \(b \in \mathbb{N} \setminus \{0\}\). wir beweisen die Aussage:

\begin{flalign*}\forall a \in \mathbb{N}\exists q,r \in \mathbb{N}: a= q \cdot b + r, r < b\end{flalign*}

Beweis mittels vollständiger Induktion über \(a \in \mathbb{N}\)

\(\boxed{IA} a = 0\) Klar mit \(a = 0 \cdot b + 0\).

 \(\boxed{IS}\) Sei
\(a \in \mathbb{N}\), sodass

Ist \(r+1 < b\), dann folgt IS mit:

\( a+1=b+(r+1)  \)

Ist \(r+1 = b \), dann folgt IS mit:
\( a+1 = (q+1) b + 0  \)

\(\hfil \quad \linebreak\)


\ssect{Definition}

Seien \(a,b \in \mathbb{N}\). Dann heißt \(d \in \mathbb{N}\)
\textbf{größter gemeinsamer Teiler (ggT)} von a und b \(: \iff\) 1.
\(d|a \land d|b\) 2.
\(\forall d' \in \mathbb{N}: d'|a \land d'|b \Rightarrow d'|d\)

Weiter heißt \(m \in \mathbb{N}\) \textbf{kleinstes gemeinsames
Vielfaches (kgV)} von a und b \(: \iff\) 1. \(a|m \land b|m\) 2.
\(\forall m' \in \mathbb{N}: (a|m' \land b|m') \Rightarrow m|m'\)


\ssect{Satz}

Seien \(a,b \in \mathbb{N}\). Dann gelten: 1. Es existiert genau ein ggT
von a und b, den wir mit ggT(a,b) bezeichnen. Es gilt: 2. Es existiert
genau ein kgV von a und b, den wir mit kgV(a,b) bezeichnen.


\sssect{Beweis}

\begin{enumerate}
\def\labelenumi{\arabic{enumi}.}

\item
  Eindeutigkeit: Sind d und d' ggT von a und b, dann d\textbar{}d' und
  d'\textbar{}d, also \(d = d'\) (Bemerkung 9.3.1) Existenz: Ist
  \(a = b = 0\), dann \(ggT(a,b) = 0\) (Klar ;) )
\end{enumerate}

Wir nehmen daher an (o.B.d.A = Ohne Beschränkung der Allgemeinheit, o.E.
= ``ohne Einschränkung'', w.l.o.g = without loss of generality), dass
m\textbar{}
\(a \neq 0\) oder \(b \neq 0\). Dann ist die Menge nicht leer. Nach Satz
9.8 besitzt A kleinstes Element \(d \in A\) (bzgl. \(\leq\)). Nach
Definition von A gibt es \(s,t \in \mathbb{Z}\) mit \(d = sa+bt\).

Behauptung: \(d|a \land d|b\). Dazu (nach Lemma 9.9) gibt es
\(q,r \in \mathbb{N}\) mit \(a =dq+r, r < d\). Dann

\begin{flalign*}(\star):r = a - qd = \underbrace{(1-qs)}{\in\mathbb{Z}} a + \underbrace{(-qt)}{\in \mathbb{Z}}b\end{flalign*}

Wäre \(r \neq 0\), dann also \(r \in A\) (wegen (\(\star\))) und daher
Widerspruch wegen \(r < d\) Somit also \(r = 0\), d\textbar{}a. Also
zeigt man d\textbar{}b. Ist nun \(d' \in \mathbb{N}\) mit
\(d'|a \land d'|b\), dann \(d'| (sa+tb)\) (nach Proposition 9.2,2).

Somit ggt(a,b) = d = sa + tb

\begin{enumerate}
\def\labelenumi{\arabic{enumi})}
\setcounter{enumi}{1}

\item
  Eindeutigkeit: wie zuvor mit (Bemerkung 9.3,1). Existenz: Ist a = 0
  oder b = 0, dann kgV(a,b) = 0
\end{enumerate}
\end{document}
