\documentclass{../../meta/tudscript}
\begin{document}
    \sect{Modulare Arithmetik}
        \ssect{Lemma: Modulares Rechnen}
            Sei $m \in \bN \ \Set{0}$. Die Relation $\equiv_m \subseteq \bZ \times \bZ$ mit $x \equiv_m y :\iff m|(x-y) (\iff \exists k \in \bZ: km=x-y)$ ist
            eine Äquivalenzrelation
            
            \sssect{Beweis}
                Reflexivität und Symmetrie: Klar: Üb!
		\paragraph{}
                Transitivität: Seien $x,y,z \in \bZ$ mit $x \equiv_m y$ und $y \equiv_m z$, d.h. $m|(x-y)$ und $m|(y-z)$.
		Dann $m|((x-y)+(y-z)) = x-z$ nach Proposition 9.2\ref{9.2,2)}, also $x \equiv_m z$.
                $\hfill\square$

        \ssect{Lemma: Eindeutigkeit}
            Sei $m \in \bN \setminus \Set{0}$ und $x \in \bZ$. Dann existiert genau ein Paar $(q,r) \in \bZ \times \Set{0, \ldots, m-1}$ mit $x = qm -r$
            
            \ssect{Beweis}
                Existens folgt mit Lemma 9.9\ref{9.9}. Eindeutigkeit: $q, q' \in \bZ$ und $r, r' \in \Set{0, \ldots, m-q}$ mit $qm+r = q'm +r'$
                
                O.E. (Ohne Einschränkung): $r \geq r'$. Dann $(q'-q)m = r -r' \in \Set{0, \ldots, m-1}$ und daher $r-r' = 0$ und $q'-q = 0$ und $r = r'$ und $q = q'$
                
        \ssect{Definition: Schreibweisen für modulares rechnen}
            Sei $m \in \bN \setminus \Set{0}$. Für $x \in \bZ$ bezeichnet man das eindeutig bestimmte $r \in \Set{0, \ldots, m-1}$ mit $m|(x-r)$ durch

	    \ilmath{x \text{mod} m:= r: \text{genannt: \underline{Rest} von x \underline{modulo} m}}

            und nennt die Menge $[x]_m := [x]_{\equiv_m} = \Set{y \in \bZ \mid x \equiv_m y}$ die \underline{Restklasse} von x modulo m.

	    \paragraph{}
            Zahlen $x,y \in \bZ$ heißen \underline{kongruent modulo m} $:\iff x \equiv_m y$

            (Schreibweise $x \equiv y (\mod m)$)

            Definiere $\bZ_m := \bZ \ \equiv_m = \Set{[x]_m \mid x \in \bZ}$

       \ssect{Satz}
            Sei $m \in \bN \setminus \Set{0}$. Die Menge $\bZ_m$ bildet mit der Addition $+: \bZ_m \times \bZ_m \rightarrow \bZ_m$ und der Multiplikation 
            $\cdot: \bZ_m \times \bZ_m \rightarrow \bZ_m$, definiert durch $[x]_m + [y]_m = [x+y]_m$, $[x]_m \cdot [y]_m = [x \cdot y]_m$ für alle $x,y \in \bZ$, 
            einen kommutativen Ring.

            \sssect{Beweis}
                Wohldefiniertheit: $x, x', y, y' \in \bZ$ mit $x \equiv_m x'$ und $y \equiv_m y'$, also $m|(x-x')$ und $m|(y-y')$, dann mit Proposition 9.2\ref{9.2} auch:
    
                \ilmath{m| (x-x'+y-y') &= (x+y)- (x'+y') \\
                        m| ((x-x')y + x' (y-y')) &= xy-x'y'}
                
                daher $x+y \equiv_m  x'+y'$ und $xy \equiv_m x'y'$.
                Mit 8.6\ref{8.6} folgt nun, dass $(\bZ_m, +, \cdot)$ ein kommutativer Ring mit Nullelement $[0]_m$ und Einselement $[1]_m$.

        \ssect{Bemerkung}
            Sei $m \in \bN \setminus \Set{0}$ die Abbildung $\Set{0, \ldots, m-1} \rightarrow \bZ_m, x \mapsto [x]_m$ ist eine Bijektion. 
            (Umkehrabbildung: $\bZ_m \rightarrow \Set{0, \ldots, m-1}, [x]_m \mapsto x \mod m$).

            Man identifiziert häufig $x \in \Set{0, \ldots, m-1}$ mit $[x]_m \in \bZ_m$ und die Menge $\Set{0, \ldots, m-1}$ mit $\bZ_m$
            
        \ssect{Beispiel: Verknüpfungstafel}
            Verknüpfungstafel für den Ring $\underbrace{\bZ_3}_{\Set{0, 1, 2}}, +, \cdot$

		

				
					\begin{tabular}{|l|l|l|l|}
						\hline
						+ & 0 & 1 & 2 \\ \hline
						0 & 0 & 1 & 2 \\
						1 & 1 & 2 & 0 \\ 
						2 & 2 & 0 & 1 \\
					\end{tabular}
			

            	Symmetrie an der Diagonalen.
			
					\begin{tabular}{|l|l|l|l|}
		        		       	\hline
					$\cdot$  & 0 & 1 & 2 \\ \hline
                				0 & 0 & 0 & 0 \\
	                	   		1 & 0 & 1 & 2 \\
       		        		  	2 & 0 & 2 & 1 \\
            				\end{tabular}
             
        		Für den Ring $(\bZ_4, + \cdot)$

			
            			\begin{tabular}{|l|l|l|l|l|}
                			+ & 0 & 1 & 2 & 3 \\ \hline	
        		        		0 & 0 & 1 & 2 & 3 \\
               			1 & 1 & 2 & 3 & 0 \\
		         	      	2 & 2 & 3 & 0 & 1 \\
        		        		3 & 3 & 0 & 1 & 2 \\
		           	 \end{tabular}
        	    		$$
			
            			\begin{tabular}{|l|l|l|l|l|}
                			$\cdot$  & 0 & 1 & 2 & 3 \\ \hline
        			        	0 & 0 & 0 & 0 & 0 \\
                   			1 & 0 & 1 & 2 & 3 \\
		                     	2 & 0 & 2 & 0 & 2 \\
        		             	3 & 0 & 3 & 2 & 1 \\
            			\end{tabular}
	            	
            
            
            
        \ssect{Definition}
            Sei $(R, +, \cdot)$ Ring. Die Elemente von $R* := \Set{x \in R \mid \exists y \in R: xy = yx = 1_R}$ heißen \underline{Einheiten} des Rings $(R, +, \cdot)$.

        \ssect{Bemerkung}
            Ist $(M, \cdot)$ ein Monoid, dann ist $(\Set{x \in M \mid \exists y \in M: x \circ y = y \circ x = e}, \circ)$ eine Gruppe. 
            
            Insbesondere gilt:

            Ist $(R, +, \cdot)$ ein Ring, dann ist $(R*, \cdot)$ eine Gruppe.

        \ssect{Proposition}
            Sei $m \in \bN \ \Set{0}$. Für $a \in \Set{0, \ldots, m-1}$ gilt:

            \ilmath{[x]_m \in \bZ*_m \iff ggT (a, m) = 1}

            \sssect{Beweis}
                Sei $a \in \Set{0, \ldots, m-1}$. Vorüberlegeung:
                \ilmath{ggT (a, m) = 1 \iff \exists s,t \in \bZ:sa + tm = 1}
                
                Beweis dazu: "$\implies$": gilt nach Satz 9.11\ref{9.11} (Lemma von Bezont)

                "$\Leftarrow$": Seien $s,t \in \bZ$ mit $sa + tm = 1$. als $ggT(a,m)|a$ und $ggT(a,m)|m$ folgt mit Proposition 9.2\ref{9.2} auch $ggT(a.m)|(sa+tm) = 1$, 
                also $ggT(a,m) = 1$.
                $\hfill\square$

                Zum Beweis der Proposition
                \ilmath{[a]_m \in \bZ*_m &\iff \exists s \in \bZ:[s][a] = 1\\
                &\iff \exists s \in \bZ: m|(1-sa) \\
                &\iff \exists t \in \bZ: 1 - sa = tm \\
                &\iff \exists s,t \in \bZ: 1 = sa + tm \\
                &\iff ggT(a,m) = 1}
                
                $\hfill\square$
                
        \ssect{Folgerung}
            Sei $m \in \bN \ \Set{0}$. Dann gilt: $(\bZ_m, +, \cdot)$ ist ein Körper, genau dann wenn $m \in \bP$.

            \sssect{Beweis}
                $(\bZ_m, +, \cdot)$ Körper $\iff [0]_m + [1]_m$ und $\bZ_m \ \Set{0} \subseteq \bZ*_m$
                $\iff m \geq 2$ und $\forall a \in \Set{1, \ldots, m-1}: ggT(a,m) = 1$ $\iff m \in \bP$
                
                $\hfill\square$

        \ssect{Folgerung}
            Für $m \in \bN \ \Set{0}$ ist $\varphi(m) = |\bZ*_m|$. 

            \sssect{Beweis}
                Folgt direkt aus Proposition 10.9\ref{10.9}

                $\hfill\square$


\end{document}
