\documentclass{../../meta/tudscript}
\begin{document}

\ssect{Axiom: Extensionalität}

Zwei Mengen A und B sind genau dann gleich, wenn sie dieselben Elemente
besitzen, d.h. es gilt \(A = B \iff (\forall x: x \in A \iff x \in B)\)

\sssect{Randbemerkung}

Insbesondere \(\{a, b\} = \{b, a\}\) für beliebige Elemente a,b sowie
auch \(\{a, a\} = \{a\}\)

\ssect{Definition}

Sind A, B Mengen, dann gilt:

\begin{itemize}

\item
  \(A \neq B :\iff \neg(A = B)\)
\item
  \(A \subseteq B :\iff \forall x: (x \in A \Rightarrow x \in B)\) {[}A
  ist Teilmenge von B{]}
\item
  \(A \subsetneq B :\iff (A \subseteq B) \cap (A \neq B)\) {[}A ist
  echte Teilmenge von B{]}
\item
  \(A \nsubseteq :\iff \neg (A \subseteq B)\)
\end{itemize}

\ssect{Lemma (Hilfssatz)}

Für beliebige Mengen A, B, C gilt:

\begin{enumerate}
\def\labelenumi{\arabic{enumi}.}

\item
  \(A \subseteq A\)
\item
  \(A = B \iff (A \subseteq B) \cap (B \subseteq A)\)
\item
  \(A \subseteq B \cap B \subseteq C \Rightarrow A \subseteq C\)
\end{enumerate}

\sssect{Beweis}

\begin{enumerate}
\def\labelenumi{\arabic{enumi}.}

\item
  Offenbar ist \(\forall x: x \in A \iff x \in A\) eine wahre Aussage
\item
	\(A = B \overset{Extens.}{\iff} \forall x : x \in A \iff x \in B\)
\end{enumerate}

\(\hspace{37pt} \qquad \Leftrightarrow{1.4, 7)} \forall x: (x \in A \Leftarrow x \in B) \cap (x \in B \Rightarrow x \in A)\)

\(\hspace{37pt} \qquad \Leftrightarrow{1.8} (\forall x: x\in A \Rightarrow x \in B) \cap (\forall x: x \in B \Rightarrow x \in A)\)

\begin{enumerate}
\def\labelenumi{\arabic{enumi}.}
\setcounter{enumi}{2}
\item
  Annahme: \(A \subseteq B\) und \(B \subseteq C\)

  Zu zeigen: \(A \subseteq C\) d.h.
  \(\forall x: x \in A \Rightarrow x \in C\)

  Sei also \(x \in A\). Dann ist \(x \in B\) (wegen \(A \in B\)) und
  weiter \(x \in C\) (wegen \(B \subseteq C\) )

  qed.
\end{enumerate}

\ssect{Axiom: Leere Menge}

Es existiert eine Menge, die keine Elemente besitzt, d.h.
\(\exists M: \forall x: \neg (x \in M)\)

Wegen Extensionalität ist diese Menge eindeutig bestimmt, und wir
bezeichnen diese als die leere Menge \(\emptyset\).
\ssect{Bemerkung}

Für jede Menge M gilt:

\(\emptyset \in M\), denn die Aussage
\(\forall x: x \in \emptyset \Rightarrow x \in M\) ist offenbar war.

\ssect{Axion (Aussonderung)}

Sei M Menge und P(x) Prädikat. Dann existiert die Menge
\(\{x \in M \vert P(x)\}\) all jener Elemente x von M, für die P(x) wahr
ist.

\ssect{Definitionen}

Seien A,B Mengen, so gilt

\begin{itemize}

\item
  \(A \cap B := \{x \in A \vert x \ in B\}\) {[}Schnittmenge ``A und
  B''{]}
\item
  \(A \diagdown B := \{x \in A \vert x \ in B\}\) {[}Differenzmenge ``A
  ohne B''{]}
\item
  Ist \(A \subseteq U\) für eine Menge U, so wird \(U \diagdown A\) auch
  das Komplement von A in U gennant und mit \({A}^\complement\) oder
  \(\overline{A}\) bezeichnet, falls U klar ist.
\item
  Ist \(\mathscr{C}\) eine nichtleere Menge von Mengen, dann enthält
  \(\mathscr{C}\) mindestens eine Menge \(C_0 \in \mathscr{C}\) Wir
  definieren die Schnittmenge
  \(\bigcup\mathscr{C} := \{x \in C_0 \vert \forall C \in \mathscr{C}: x \in C\}\)
\end{itemize}

\ssect{Darstellung durch Venn Diagramme}
\begin{center}
\begin{tikzpicture}[scale=0.25]
\fill [pattern=north west lines, pattern color=blue](0,0) circle (2);
\fill [pattern=north west lines, pattern color=blue](2,0) circle (2);
\draw (0,0) circle (2)
      (2,0) circle (2)
      (-3,-3) rectangle (5, 4.5)
      (1, 2) node [text=black, above] {$A \cup B$};
\end{tikzpicture}
\begin{tikzpicture}[scale=0.25]
    \scope
        \clip (0,0) circle (2);
        \fill [pattern=north west lines, pattern color=blue](2,0) circle (2);
    \endscope
    \scope
        \clip (2,0) circle (2);
        \fill [pattern=north west lines, pattern color=blue](0,0) circle (2);
    \endscope
    \draw (0,0) circle (2)
          (2,0) circle (2)
          (-3,-3) rectangle (5, 4.5)
          (1, 2) node [text=black, above] {$A \cap B$};
\end{tikzpicture}
\begin{tikzpicture}[scale=0.25]
    \scope[even odd rule]
        \clip (2,0) circle (2)
              (0,0) circle (2);
        \fill [pattern=north west lines, pattern color=blue](0,0) circle (2);
    \endscope
    \draw (0,0) circle (2)
          (2,0) circle (2)
          (-3,-3) rectangle (5, 4.5)
          (1, 2) node [text=black, above] {$\overline{A \diagdown B}$};
\end{tikzpicture}
\begin{tikzpicture}[scale=0.25]
        \fill [pattern=north west lines, pattern color=blue] (-3,-3)rectangle(5,4.5);
        \fill [color=white] (2,0) circle (2) (0,0) circle (2);
    \draw (0,0) circle (2)
          (2,0) circle (2)
          (-3,-3) rectangle (5, 4.5)
          (5, 4.5)  node [text=black, above left] {$U$}
          (1, 2) node [text=black, above] {$\overline{A \cup B}$};
\end{tikzpicture}
\end{center}

\ssect{Proposition}

\begin{enumerate}
\def\labelenumi{\arabic{enumi}.}

\item
  Assoziativität

  \begin{itemize}
  
  \item
    \(A \cup (B \cup C) = (A \cup B) \cup C\)
  \item
    \(A \cap (B \cap C) = (A \cap B) \cap C\)
  \end{itemize}
\item
  Komplementanität

  \begin{itemize}
  
  \item
    \(A \cap \overline{A} = \emptyset\)
  \end{itemize}
\item
  Kommutativität

  \begin{itemize}
  
  \item
    \(A \cup B = B \cup A\)
  \item
    \(A \cap B = B \cap A\)
  \end{itemize}
\item
  Distributivität

  \begin{itemize}
  
  \item
    \(A \cup (B \cap C) = (A \cup B) \cap (A \cup C)\)
  \item
    \(A \cap (B \cup C) = (A \cap B) \cup (A \cap C)\)
  \end{itemize}
\item
  De Morgansche Regeln

  \begin{itemize}
  
  \item
    \(\overline{(A \cup B)} = \overline{A} \cap \overline{B}\)
  \item
    \(\overline{(A \cap B)} = \overline{A} \cup \overline{B}\)
  \end{itemize}
\item
  \(\overline{\overline{A}} = A\)
\item
  \(A = B \iff (A \subseteq B) \ cap (B \subseteq A)\)
\end{enumerate}

\sssect{Beweis (Hier nur für 5)}

\(x \in \overline{A \cup B} \iff \neg(x \in A \cup B) \iff \neg(x \in A \cup x \in B)\)

\(\iff \neg(x \in A) \cap \neg (x \in B) \iff x \in \overline{A} \cap x \in \overline{B}\)

\(\iff x \in \overline{A} \cap \overline{B}\)

\ssect{Axion (Potenzmenge)}

Für jede Menge M existiert die Potenzmenge \(\wp(M)\), welche als
Elemente genau die Teilmengen von M enthält

\ssect{Proposition}

Sei \(\mathscr{C} \subseteq \wp(U)\) für eine Menge U, dann gilt:

\(\overline{\underset{A\in \mathscr{C}}{\bigcup A}} = \underset{A \in \mathscr{C}}{\bigcup \overline{A}} \quad,\quad \overline{\underset{A\in \mathscr{C}}{\bigcap A}} = \underset{A \in \mathscr{C}}{\bigcap \overline{A}}\),

\sssect{Beweis: Hier nur erste Gleichung. Für jedes $x \in U$ gilt:}

\({x \in \underset{\overline{A}\in \mathscr{C}}{\bigcap A}} \iff {\neg \left(x \in \underset{A\in \mathscr{C}}{\bigcap A}\right)}  \iff {\neg \left(\exists A \in \mathscr{C}: x \in A\right)} \iff {\forall A \in \mathscr{C}: \neg(x \in A)} \iff \forall A \in \mathscr{C}: x \in \overline{A} \iff x \in \underset{A\in \mathscr{C}}{\bigcap \overline{A}}\)

Wegen Extensionalität folgt
\(\overline{\underset{A\in \mathscr{C}}{\bigcup A}} = \underset{A\in \mathscr{C}}{\bigcap \overline{A}} \quad qed.\)

\ssect{Definition}

Seien A,B Mengen. Für \(a \in A\) und \(b \in B\) definieren wie das
(geordnete) Paar. \((a, b) := \{\{a\}, \{a,b\}\}\)

Das kartesische Produkt \(A \times B\) ist die Menge aller geordneten
Paare (a,b) mit \(a \in A\) und \(b \in B\), d.h.
\(A x B:=\{(a,b) \vert a \in A, b \in B\}\)

Mengentheoretisch genauer:

\(A \times B:=\{x \in \wp(\wp(A \cup B)) \vert \exists a \in A \>\exists b \in B: x=(a,b)\}\)


\end{document}
