\documentclass{../tudscript}
\author{Jakob Krebs}
\title{Mathe VL 7}

\begin{document}
\sect{Bemerkung}
Im Restklassenring $\bZ_n$ gilt:
\ilmath{a \in \bZ_n \setminus \Set{0} \implies \text{ a ist entweder Nullteiler oder Einheit}}
\paragraph{Beweis}
Sei $a \in \bZ_n \setminus \Ste{0}$/
\begin{enumerate}
\item $ggT(a, n) = 1 \implies A^{-1} ex. \implies$ a Einheit
\item $ggT(a, n) = d > 1 \implies an = n' d $ und $a = a' d$ mit $a', n' \in \bZ_n$
  d.h. a ist ein Nullteiler in $\bZ_n$
\item a kann nicht geleichzeitig Einheit und Nullteiler sin. $\hfill\square$
\end{enumerate}

Sei $(R, +, \cdot)$ ein kommutativer Ring mit Einselemnt.

Annahme: $a \in R, a \neq 0$, a Nullteiler, a Einheit.

$a a^{-1} = 1 \implies (a a^{-1})b = 1 b \implies (ab) a^{-1} = 1 b \implies b = 0$

Widerspruch. Also kein a nicht Nullteiler und einheit gleichzeit sein.

\ssect{Bemerkung}
$\bZ = \Set{0, 1, -1}$ enthält weder Nullteiler noch Einheiten

%TODO Lücken füllen

\sect{Polynomringe}
Sei $(z, +, \cdot)$ ein Ring

$R(x)$ R adjungiert x bezeichnet die Menge der Polynome in (der unbestimmten) x über R.

\ilmath{R(x) := \underbrace{\Set{_0 + a_1 x + a_2 x^2 + \ldots \mid a_-, \ldots a_n \in R(x)}}_{Nur endlich viele Summanden sind von 0 verschieden.}}

\ssect{Bezeichnung}
\ilmath{\underbrace{a_0 + a_1 x + \ldots + a_n x^n}_{a(x)} = (a_0 a_1 \ldots a_n)}
heißt Polunom in x über R, $a_0, \ldots$ heißt Koeefizienten

\ssect{Bemerkung}
x ist ein formales Rechensymbol
\ssect{Rechnen in POlynomringen}

Addition:

\ilmath{a(x) \oplus b(x) = (a_0 + b_0) + (a_1 + b_1)x + \ldots \\
(a_0 a_1 a_2 ) \oplus (b_0 b_1 b_2 ) = (a_0b_0 a_1b_1 a_2b_2)}

Multiplikation:
\ilmath{a(x) \odot b(x) = \sum_{m = 0}^n ((\sum_{i = 0}^{m} a_i b_{m-1})x^m})

\ssect{Satz}
$(R, +, \cdot)$ Ring $\implies (R(x), \oplus, \odot)$ Ring, \underline{Polynomring} über R in der unbestimmten x.
\ssect{Satz}

$(R, +, \cdot)$ Integritäatsring $\implies (R(x), \oplus, \odot)$ Integritätsring.

\ssect{Bemerkung}
$(K, +, \cdot)$ Körper $\implies (K(x), \oplus, \odot)$ kein Körper, denn x hat kein multiplikatives Inverses
\ilmath{a(x) \odot x + 1 \text{ für alle } a(x) \in K(x)}
%TODO Beweis 
\ssect{Satz}

$(K, +, \cdot)$ Körper $\implies (K(x), \oplus, \odot)$ euklidischer Ring.

\sect{Bemerkung: euklidische Ringe}
Ein Ring R ist genau dann ein \underline{euklidischer Ring}, wenn man jeden $ggT(a, b)$ mit $a, b \in R, (a, b) \neq (0,0)$ mit dem
euklidischen Algorithmus berechnen kann.

\ssect{Grad eines Plynoms}
%%TODO
wie bei Polynomfunktionen.

Das Nullpolynom 0 hat keinen Grad.

\ssect{Satz: Polynomdivision}
Sei K ein Körper.
Gilt $a(x), b(x) \in K(x)$ mit $a(x) \neq 0$, dann gibt es eindeutig bestimmte POlynome $q(x), r(x) \in K(x)$ mit:

\ilmath{b(x) &= q(x) \odot a(x) \oplus r(x)}
und
\ilmath{r(x) = 0 \lor grad(r(x)) < grad(a(x))}

\ssect{Bemerkung}
%%TODO kann ich nicht lesen. wir sitzten echt zuweit hinten.

%%TODO vielleicht hat jemand Lust das Beisiel zu texen

\sect{Definition: ggT in Integritätsringen}
Sei $(R, +, ]cdot)$ ein Integritätsring, $a, b \in R, (a, b) \neq (0, 0)$
$d \in R$ heißt \underline{ein größter gemeinsamer Teiler}, von a und b, wenn gilt:
\begin{enumerate}
  \item $d|a$ und $d|b$
  \item $t|a$ und $t|b \implies t|d$ für alle $t \in R$
\end{enumerate}
\ssect{Bezeichnung}
  \ilmath{ggT(a, b) \cong d}
  assoziiert zu.
  \ssect{Beispiel}
  %%TODO kann jakob wieder nicht lesen

\sect{endliche Körper GF(q)}
\ssect{Satz}
Ein endlicher Körper mit q Elementen existiert genau dann,
wenn q eine Primzahl ist. Gilt $q = p^k$ (p Prim, $k \in \bN, k \geq 1$), dann gibt bis auf Isomorphie genau einen endliche Körper mit q Elementen.

\ssect{Beispiel}
\ilmath{GF(p) \cong \bZ_p, P prim}
Aber: $Z_n$ n kein Prim, ist kein Körper.
\end{document}
