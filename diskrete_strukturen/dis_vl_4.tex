\documentclass{../../meta/tudscript}
\begin{document}

\ssect{Definition: Partition}

Sei X Menge. Eine Partition von X ist eine Menge \(\mathscr{C}\) von
nicht-leeren Teilmengen von X derart, dass

\begin{enumerate}
\def\labelenumi{\arabic{enumi})}

\item
  \(X = \bigcup \mathscr{C}\)
\item
  \(\forall A, B \in \mathscr{C}: A\neq B \Rightarrow A \cap B = \emptyset\)
\end{enumerate}

Einschub: 2 Mengen A und B heißen disjunkt, wenn
\(A \cap B = \emptyset\)

Ist \(\mathscr{C}\) eine Partition von X, so definieren wir die Relation
\(R_\mathscr{C}\subseteq X x X\) durch

\ssect{Satz}

Sei X Menge, dann

\begin{enumerate}

\item
  Ist R Äquivalenzrelation auf X, dann ist \(X/R\) eine Partition von X
  und \(R_{X/R} = R\)
\item
  Ist \(\mathscr{C}\) eine Partition von X, dann ist \(R_\mathscr{C}\)
  eine Äquivalenzrelation auf X und \(X/R_\mathscr{C}= \mathscr{C}\)
\end{enumerate}

\sssect{Beweis}

\paragraph{(1)}

Sei R Äquivalenzrelation auf X. Wegen Reflexivität:

\begin{itemize}

\item
  \(\emptyset \not\in X/R\), denn
  \(\forall x \in X: (x \in [x]_R \Rightarrow[x]_R \neq 0)\){]}
\item
  \(X = \bigcup X/R\)
\end{itemize}

Nach Lemma 3.7 auch

Also \(X/R\) Partition von X. Außerdem:

\(\forall x,y \in X: [ (x,y) \in R_{X/R} \iff \exists z \in X: x,y \in [z]_R \overset{3.7}{\iff} (x,y) \in R ]\)

Somit  \(R\_\{X/R\} = R\)

\paragraph{(2)}

Sei \(\mathscr{C}\) Partition von X. Zu zeigen: \(R_\mathscr{C}\) ist
Äquivalenzerlation auf X 
\begin{itemize}
	\item
		Reflektiv: Für jedes \(\exists x\) existiert \(A \in \mathscr{C}\) mit \(x \in A\), und daher \((x,x) \in R_\mathscr{C}\)
	\item
		Symmetrie: Klar 
	\item 
		Transitivität: Seien \((x,y) \in R_\mathscr{C}\) und \((y,z)\in R_\mathscr{C}\). Dann gibt es
		\(A,B \in \mathscr{C}\) mit \( x,y \in A\) und \(y,z \in B\). Da \(y \in A \cup B\), ist 
		\(A \cup B \neq \emptyset\) und somit \(A = B\) (weil \(\mathscr{C}\) Partition, siehe 3.8.2). Folglich
		\(x,z \in A = B\) also \((x,z) \in R_\mathscr{C}\)
\end{itemize}
Noch zu zeigen: \(X/R = \mathscr{C}\)

\paragraph{Beobachtung:}

\paragraph{\texorpdfstring{Beweis von \((\text{*})\):}}

Sei \(A \in \mathscr{C}\) und \(x \in A\). Für alle \(y \in X\) gilt:

TODO

Mit \((*)\) folgt:
TODO
qed.

\ssect{Beispiele: Partitionen}

Es gibt fünf Partitionen auf der Menge \{1, 2, 3\}:

\begin{itemize}

\item
  \{\{1, 2, 3\}\}
\item
  \{\{1\}, \{2\}, \{3\}\}
\item
  \{\{1, 2\}, \{3\}\}
\item
  \{\{1, 3\}, \{2\}\}
\item
  \{\{2, 3\}, \{1\}\}
\end{itemize}

Darstellung der zugehörigen Äquivalenzrelationen:

\begin{tabular}{|c|ccc|ccc|ccc|ccc|ccc|}
\hline
&1&2&3&1&2&3&1&2&3&1&2&3&1&2&3\\
\hline
1&    X& & &    X&X& &    X& &X&    X& & &    X&X&X\\
1&     &X& &    X&X& &     &X& &     &X&X&    X&X&X\\
1&     & &X&     & &X&    X& &X&     &X&X&    X&X&X\\
\hline
\end{tabular}

\ssect{Bemerkung}

Sei X Menge. Betrachte die Menge

Dann ist eine Ordnungsrelation \(\preceq\) auf \(Part(X)\) gegeben durch
``\(\mathscr{B}\) wird verfeinert durch \(\mathscr{C}\)''

\sect{Abbildungen}

\ssect{Definition: Abbildung}

Seien X,Y Mengen. Eine Abbildung (oder Funktion) f von X nach Y, in
Zeichen \(f: X -> Y\), ist eine Relation \(f \subseteq X \times Y\)
derart, dass \(\forall x \in X \;\exists! y \in Y: (x,y) \in f\) Für das
zu einem \(x \in X\) eindeutig bestimmte \(y \in Y\) mit
\((x, y) \in f\) schreiben wir \(f(x) := y\)

Sprechweise: f bildet das Argument x auf den Funktionswert y ab

Notation: \(f: X \rightarrow Y,\quad x \mapsto f(x)\)

Die Menge X heißt Definitionsbereich und die Menge Y Zielbereich der
Funktion f.
\ssect{Beispiel}

Die Abbildung
\(f: \mathbb{R} \rightarrow \mathbb{R}, \quad x \mapsto x^2\) ordnet
jeder reellen Zahl \(x \in \mathbb{R}\) ihr Quadrat
\(x^2 \in \mathbb{R}\) zu.

\ssect{Definition}

Seien X, Y Mengen mit \(f: X \rightarrow Y\), so heißt für
\(A \subseteq X\) und \(B \subseteq Y\)

\begin{flalign*}f(A) := \{f(a) \mid a\in A \} = \{y \in Y \mid \exists a \in A: y = f(a)\}&&\end{flalign*}

das Bild von A unter f, sowie

\begin{flalign*}f^{-1}(B) := \{x \in X \mid f(x) \in B\}&&\end{flalign*}

das Urbild von B unter f.

Die Einschränkung von f auf eine Teilmenge \(A \in X\) ist die Abbildung
\({f \mid _A: A \rightarrow Y, \quad a \mapsto f(a)}\), d.h
\(f \mid _A = f \cap (A \times Y)\)

Die Abbildung \(f: X \rightarrow Y\) heißt

\begin{itemize}

\item
  \(\underset{\text{(Injektion)}}{\text{injektiv}}\)
  \(:\iff \forall x,y \in X: f(x) = f(y) \Rightarrow x=y\)
\item
  \(\underset{\text{(Sujektion)}}{\text{sujektiv}}\)
  \(:\iff \forall y \in Y \enspace\exists x \in X: f(x) = y \Rightarrow x=y\)
\item
  \(\underset{\text{(Bijektion)}}{\text{bijektiv}}\) \(:\iff\) f ist
  injektiv und sujektiv
\end{itemize}

Die Menge \(im(f) := f(X)\) heißt auch das Bild von f

\ssect{Beispiele}

\begin{enumerate}
\def\labelenumi{\arabic{enumi})}

\item
  Sei X Menge und \(A \subseteq X\). Dann ist
  \({c_{A}^X: A \rightarrow X, \enspace a \mapsto a}\) bijektiv
\item
  Sei R Äquivalenzrelation auf Menge X. Dann ist
  \({\pi_R: X \rightarrow X/R,\enspace x \mapsto [X]_R}\)
\item
  Sei X Menge. Die identische Abbildung
  \(Id_X : X \rightarrow X,\enspace x \rightarrow x\) ist bijektiv
  (Beachte: \(\bigtriangleup_X = Id_x\))
\item
  Die Abbildung
  \(f: \mathbb{R} \rightarrow \mathbb{R},\enspace X \mapsto x^2\) ist
  weder injektiv noch sujektiv
\end{enumerate}

\ssect{Definition}

Der Kern einer Abbildung \(f: X \rightarrow Y\) ist definiert als

TODO

\ssect{Proposition}

Sei X Menge

\begin{enumerate}
\def\labelenumi{\arabic{enumi})}

\item
  Ist \(f: X \rightarrow Y\) eine Abbildung, dann ist \(Ker(f)\)
  Äquivalenzrelation auf X
\item
  Ist R eine Äquivalenzrelation auf X, dann gilt \(R = Ker(\pi_R)\)
\end{enumerate}

\sssect{Beweis:}

\paragraph{1)}

\begin{itemize}

\item
  Reflexivität und Symmetrie: Übung
\item
  Transitivität: Für \(x,y,z \in X\) ist 
\end{itemize}

\paragraph{2)}

Sei R Äquivalenzrelation auf X, so gilt für alle \(x, y \in X\)

\ssect{Bemerkung}

Eine Abbildung \(f: X \rightarrow Y\) ist genau dann injektiv, wenn
\(Ker(f) = \bigtriangleup_X\) ist.

\ssect{Definition}

Seien X, Y, Z Mengen. Für Abbildungen \(f: X \rightarrow Y\) und
\(g: Y \rightarrow Z\) heißt die Abbildung
\(g \circ f: X \rightarrow Z, \enspace x \mapsto g(f(x))\) Komposition
oder Verkettung (``g nach f'').


\end{document}
