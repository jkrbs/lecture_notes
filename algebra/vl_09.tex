\documentclass{../tudscript}
\author{Jakob Krebs}
\title{Mathe VL 9}

\begin{document}
\sect{Satz vom primitiven Element}
\ssect{Satz}
Für jeden endlichen Körper GF(q) ist die multiplikative Gruppe zyklisch.
\ssect{Bemerkung}
In GF(q) gibt es ein Element $\alpha$ mit
\ilmath{GF(q) &= \Set{0} \cup \Set{\alpha, \alpha^2, \ldots, \alpha^{q-1}} \text{ mit} \alpha^{q-1} =1 \\
  &= \Set{0} \cup \langle \alpha \rangle}
$\alpha$ wir ein \underline{primitives} Elemenet genannt.

\ssect{Bemerkung}
x ist ein primitives Element in $GF(p)[x] / f(x) \iff f(x)$ ist ein primitives polynom, mit $f(x)$ irreduzibel.
\ssect{Beispiel}
\begin{enumerate}
  \item $x^4 + x^3 + 1$ ist primitiv über GF(2) $\implies x$ ist ein primitives Element im Körper %GF(2)[x]/ x^4 + x^3 + 1 \implies$
    $GF(2)[x]/x^4+x^3 + 1$ hat die Elemente $0$ und $x^i mod x^4 + x^3 + 1 (i = 0, 1, \ldots, 2^4 -2)$ also $0, a, \alpha, \alpha^2, \ldots, \alpha^{14}$
    Es gilt $GF(2)[x]/x^4 + x^3 + 1 = 0 \cup \langle \alpha \rangle$ mit $\alpha:= x \mod x^4 + x^3 + 1$

    siehe Logarithmentafel
  \item $x^4 + x^3 + x^2 + x +1$ ist nicht primitiv über $GF(2)$ $\implies x$ ist kein primitives Element im Körper $GF(2)[x] / x^4 + x^3 + x^2 + x + 1$
    Aber: $GF(2)/x^4 + x^3 + x^2 + x + 1= 0 \cup \langle \alpha \rangle$ mit $\alpha := x+1 mod x^4 + x^3 + x^2 + x + 1$, d.h. $x + 1$ ist ein primitives Element.
  \item Auch endliche Körper $GF(p), p prim$ enthalten primitive Elmenete
    \begin{itemize}
    \item $\bZ_3 = 0 \cup \langle 2 \rangle$
    \item $\bZ_5 = 0 \cup \langle 2 \rangle = 0 \cup \langle 3 \rangle$
    \item $\bZ_7 = 0 \cup \langle 3 \rangle = 0 \cup \langle 5 \rangle$
    \end{itemize}
\end{enumerate}
\ssect{Bemerkung}
Die multiplikative Gruppe von GF(q) enthält $q-1$ Elemente. $GF(q)$ enthält genau $\varphi(q-1)$ Elemente. Dabei bezeichnet $\varphi$ die Eulerische Phi Funktion
\ilmath{\varphi(n) := \mid \Set{x \in \bZ_n \mid ggT(x, n) = 1}\mid}
\ssect{Bemerkung}
Endliche Körper $GF(p)[x]/f(x)$ sind Vektorrräume öber GF(p).
Allgemein gilt:
\begin{itemize}
\item Der kleinste Teilkörper eines Körpers wird Primkörper genannt
  \item Für jeden endliche Körper ist der Primkörper ein Körper GF(p)
\end{itemize}
\sssect{Folgerung}
Endliche Körper K sind Vektorräume üuber GF(p). Daher haben endliche Körper $p^k$ Elemente.
\ssect{Satz}
Sind $K_1$ und $K_2$ endliche Körper mit $\mid K_1 \mid = \mid K_2 \mid$, dann sind $K_1$ und $K_2$ gleich.
\sect{Minimalpolynome}
\ssect{Definition: Minimalpolynom}
Sei $GF(q) = GF(p^k, p prim$ ein endlicher Körper mit dem Primkörper $GF(p)$
\ilmath{GF(q) = 0 = \cup \langle \alpha \rangle}
Sei $\alpha^i \in GF(q) \setminus \Set{0}$.

Das normierte Polynomkleinsten Grades in $GF(p)[x] \subseteq GF(q)[x]$, das $\alpha^i$ als Nullstelle hat, wird \underline{Minimalpolynom $m_{\alpha^i}(x)$} von $\alpha^i$ gennant.

\ssect{Bemerkung}
Normierte POlynome haben den höchsten Koeffizienten 1:
\ilmath{1 \cdot x^k + a_{k-1} x^{k-1} + \cdots + a_1 x + a_0}
\ssect{Bemerkung}
$a^i$ ist Nullstelle von $m_{\alpha^i}(x) \iff x- \alpha^i$ ist ein Teiler von $m_{\alpha^i} (x)$
\ssect{Bemerkung}
\ilmath{M_{\alpha^i} (x) \in GF(p)[x]}

\ssect{Eigenschaften von $m_{\alpha^i} (x)$}
\begin{enumerate}
\item $m_{\alpha^i} (x)$ ist irreduzibel über GF(p)
\item $m_{\alpha^i} (x) = \prod_{j \in Z_i} (x - \alpha^j)$ mit $Z_i = \Set{i, ip, ip^2, ip^3, \ldots, ip^l}$, wobei $l > 0$, die kleinste natürliche Zahl mit
  \ilmath{i\cdot p^{l-1} = i \mod p^k -1}
  ist.
  \item $m_{\alpha^i} (x) = m_{\alpha^{ipt}} (x)$ für alle $t \in \bN$
\end{enumerate}
\ssect{Beispiel}
\ilmath{GF(q) \cong GF(2)[x] / x^4 + x^3 + 1, GF(p) \cong GF(2)}
Bezeichnungn des Minimalpolynoms von $\alpha^3$:
\ilmath{Z_3 = \Set{3, 3 \* 2, 3 \* 2^2, 3 \* 2^3}}
denn $3 \* 2^4 \equiv 3 \mod 2^4 -1$
Rechnung modulo $2^3 -1$
\ssect{Bemerkung}
Primitive Element $\alpha^i$ in $GF(2)[x] / x^4 + x^3 +1$:

$\alpha^i$ mit $i \in \Set{1, 2, 4, 8, 7, 14, 13, 11}$ (d.h. $ggT(2^4 -1, i) = 1$)

Minimalpolynome $m_{\alpha^i}(x)$, die primitive Polynome sind: (1, 2, 4, 3) und (7, 14, 13, 11)

$m_{\alpha^o} (x)$ primitives Polynom $off \alpha^i$ primitives Element
\sect{Primitive n-=te Einnheitswurzel}
\ssect{Definition}
Die Nullstellen von $x^n -1 \in GF(q)[q] (n \in \bN, n >0)$ nennt man n-te Einheitswurzeln in $GF(q)$.

\ssect{Bemerkung}
$\tilde{\alpha}$ ist eine n-te Einheitswurzel in GF(q) $\iff \alpha \in GF(q) und \tilde{\alpha}^n = 1$
\ssect{Bemerkung}
Die n-ten Einheitswurzeln bilden eine Untergruppe der multiplikativen von GF(q).

Standard lame ass Beweis von Untergruppen...
\ssect{Bemerkung}
Die n-ten Einheitswurzeln bilden eine zyklische Untergruppe der multiplikativen Gruppe von GF(q).
(denn Untergruppen von zyklische Gruppen sind zyklisch.)
\ssect{Bemerkung}
ist a eine n-te Einheitswurzel in GF(q),
dann ist die Bedeutung von a in der multiplikativen Gruppe $GF(q) \setminus \Set{0}$ ein Teiler der ORdnung der multiplikativen Grippe, also filt
\ilmath{ord(a) / q-1}
wegen $ord(a)  = \mid \langle a \rangle \mid$ gilt auch
\ilmath{\mid \langle \alpha \rangle \mid / q-1}

\ssect{Definition}
Eine n-te Einheitswurel in GF(q) heisst primitiv, wenn gilt:
\ilmath{\mid \langle \alpha \rangle \mid = n}
\end{document}
