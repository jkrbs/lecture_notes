\documentclass{../../meta/tudscript}
\begin{document}

\sssect{Beweis}

\((a,b) = {{a}, {a,b}} \\ A \times B = {(a,b) \mid a \in A, b \in B}\)

\ssect{Lemma}

Seien A und B Mengen. So gilt für alle \(a,x \in A\) und \(b, y \in B\)

\((a,b)=(x,y) \iff a = x \wedge b = y\)

\sssect{Beweis}

Offentsichtlich: \((a,b) = (x,y)\), falls a=x und b=y

Zu zeigen ist \(\Rightarrow\)

Annahme: \((a,b) = (x,y)\)

Fallunterscheidung:
\begin{itemize}

\item
	Fall 1: \(a=b\), dann
	\({x,y} \in (x,y)= (a,b) = {{a}, {a,b}} = {{a}}\), also \{x,y\} = \{a\},
	und somit x=y=a, wegen Extensionalität.

\item
  Fall 2: \(a \neq b\), dann \({a} \in (a,b)=(x,y) = {{x}, {x,y}}\),
  also a=x, weiter. \({a,b} \in (a,b) = (x,y) = {{x}, {x,y}}\), daher
  \({a,b} = {x,y}\) und \(x \neq y\)
\end{itemize}

Deswegen \(a = x \wedge b = y\)

\ssect{Proposition}

Für beliebige Mengen \(A,B,X,Y\) gilt:

\begin{enumerate}
\def\labelenumi{\arabic{enumi}.}

\item
  \((A \cup B) \times X = (A \times X) \cup (B \times X)\)
\item
  \((A \cap B) \times (X \cap Y) = (A \times X) \cap (B \times Y)\)
\item
  \((A \times X = \emptyset \iff A = \emptyset \vee B = \emptyset\)
\end{enumerate}

\sssect{Beweis}

\begin{enumerate}
\def\labelenumi{\arabic{enumi}.}

\item
  Es gilt:
  \((u,v) \in (A \cup B) \times X \iff u \in A \cup B \wedge v \in X\)
  \(\iff (u \in A \vee u \in B) \wedge v \in X \iff (u \in A \wedge v \in X) \vee (u \in B \wedge v \in X)\)
  \(\iff (u,v) \in A \times x \vee (u,v) \in B \times X\)
\end{enumerate}

\ssect{Bemerkung: Russelsche Antinomie}

Eine Menge M heiße ``normal'', wenn \(M \notin M\). Betrachten wir die
Gesamtheit N aller Mengen, d.h.

\(N := {M \mid M \notin M}\)

Ist N eine Menge? Wenn ja, dann wäre N entweder normal oder nicht
normal.

\begin{enumerate}
\def\labelenumi{\arabic{enumi}.}

\item
  Ist N normal, dann ist \(N \in N\) und daher \(N \in N\) nach
  Definition von N.
\item
  ist N nicht normal, dann ist \(N \in N\) und daher \(N \notin N\) nach
  definition von N.
\end{enumerate}

Dies ist ein Widerspruch. Es folgt: N ist keine Menge.

\sect{Relationen}

\ssect{Definition: Relation}

Eine Relation zwischen zwei Mengen X und Y ist eine Teilmenge
\(R \subseteq X \times Y\), für \((x,y) \in R\) schreibt man auch
\(xRy\)

Eine (binäre)Relation auf einer Menge X ist eine Teilmenge von
\(X \times X\).

\ssect{Eigenschaften von Relationen}

Sei X eine Menge. Eine Relation \(R \subseteq X \times X\) heißt: 
\begin{itemize}
	\item
		reflexiv \(: \iff \forall x \in X: xRx\) 
	\item
		irreflexiv \(: \iff \forall x \in X: \neg (xRx)\) 
	\item	symmetrisch \(: \iff \forall x, y \in X: xRy \Rightarrow yRx\) 
	\item	antisymmetrisch \(: \iff \forall x, y \in X: xRy \wedge yRx \Rightarrow y = x\) 
	\item	transitiv \(: \iff \forall x, y, z \in X: xRy \wedge yRz \Rightarrow xRz\) 
	\item 	total \(: \iff \forall x, y \in X: xRy \vee yRx\)
\end{itemize}

\ssect{Bemerkung}

Sei X eine Menge. Dann ist die \emph{Diagonalrelation} (identische
Realtion)

\(\Delta X := {(x,x) \mid x \in X} = {(x,y) \in X \times X \mid x = y}\)

auf X ist reflexiv, symmetrisch, antisymmetrisch und transitiv.

Die \emph{Allrellation}:

\(\bigtriangledown x = X \times X\) auf X ist reflexiv, symmetrisch,
transitiv und total.

\ssect{Beispiele}
\begin{tabular}{l c c c c c r}
X & \(R \subsetq X \times X\) & refl. & irrefl. & sym. & antisym. & trans. & total \\
\(\mathbb{Q}\) & \(\{(x,y) \in \mathbb{Q} \times \mathbb{Q} \mid x \leq y\}\) & 1 & 0 &0&1&1&1\\
\(\mathbb{Q}\) & \(\{(x,y) \in \mathbb{Q} \times \mathbb{Q} \mid x < y\}\) &0&1&0&1&1&0\\
\(\mathbb{Q}\) & \(\{(x,y) \in \mathbb{Q} \times \mathbb{Q} \mid |x| = |y|\}\)&1&0&1&0&1&0\\
\(\mathbb{N}\) & \(\{(x,y) \in \mathbb{N} \times \mathbb{N} \mid x teilt y\}\)&1&0&0&1&1&0\\
\(\mathbb{Z}\) & \(\{(x,y) \in \mathbb{Z} \times \mathbb{Z} \mid x + y = 5\}\) &0&1&1&0&0&0\\
\(\mathbb{Z}\) &\(\{(x,y) \in \mathbb{N} \times \mathbb{N} \mid x \leq y \}\) &0&1&0&1&1&0\\
\end{tabular}

\ssect{Definition: Ordnungsrelation}

Sei X Menge. Eine Relation \(R \subseteq X \times X\) heißt
Ordnungsrelation auf X, falls R reflexiv, antisymmetrisch und transitiv
ist.

\ssect{Beispiele}

\begin{enumerate}
\def\labelenumi{\arabic{enumi}.}

\item
	\({(x,y) \in \mathbb{N} \times \mathbb{N} | x \leq y} \text{ist eine Ordnungsrelation auf} \mathbb{R}\)
\item
	\({(x,y) \in \mathbb{N} \times \mathbb{N} | x teilt y} \text{ist eine Ordnungsrelation auf} \mathbb{N}\)
\end{enumerate}

\ssect{Definition: Äquivalenzrelation}

Sei X Menge. Eine Relation \(R \subseteq X \times X\) heißt
Äquivalenzrelation (ÄR) auf X, falls R reflexiv, symmetrisch und
transitiv ist.

Sei R ÄR auf X. Für \(x \subseteq X\) heißt

\([x]_R := {y \subseteq X | xRy}\) die Äquivalenzklasse von x bzgl. R

Die Menge \(y_r := {[x]_r \mid x \in X} (\subseteq P(X))\)

aller Äquivalenzklassen bzgl. R wird auch mit ``X modulo R'' bezeichnet.

\ssect{Lemma: Eigenschaften von Relationen}

Sei R Äquivalenzrelation auf einer Menge X. Für alle \(x, y \in X\) sind
folgende Aussagen äquivalent:

\begin{enumerate}
\def\labelenumi{\arabic{enumi}.}

\item
  xRy
\item
  \([x]_R \cap [y]_R \neq \emptyset\)
\item
  \([x]_R = [y]_R\)
\end{enumerate}

\sssect{Beweis: Ringschluss}

\((1) \Rightarrow (2) \Rightarrow (3)\)

\paragraph{\texorpdfstring{\((1) \Rightarrow (2)\)}}

Annahme: xRy. Wegen Symmetrie: yRx Wir zeigen: \([x]_R = [y]_R\), d.h.

(*) \(\forall z \in X: xRz \iff yRz\)

Für jedes \(z \in X\) gilt tatsächlich

\(xRz \Rightarrow yRz\) und \(yRz \Rightarrow xRz\)

Somit ist (*) eine wahre Aussage.

\paragraph{\texorpdfstring{\((3) \Rightarrow (2)\)}}

Ist \([x]_R = [y]_R\), dann insbesondere

\(x \in [x]_R = [y]_R \Rightarrow x \in [x]_R \cap [y]_R\), also
\([x]_R \cap [y]_R \neq \emptyset\).

\paragraph{\texorpdfstring{\((2) \Rightarrow (1)\)}}

\([x]_R \cap [y]_R \neq \emptyset \Rightarrow xRy\)

Annahme: \([x]_R \wedge [y]_R = 0\) Dann existiert \(z \in X\) mit xry
und yRx. Wegen Symmetrie von R ist auch zRy und wegen Transitivität auch
xRy

qed.


\end{document}
