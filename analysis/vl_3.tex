\documentclass{../tudscript}
\author{Urs Kober}
\title{Mathe VL 3}

\begin{document}
	
	\ssect{Bemerkung}
		Oft muss man $x_n$ umformen bevor die GWS anwendbar sind.
		
		\sssect{Beispiel}
			\ilmath{
				\lim_{n \to \infty}& \underbrace{\frac{11+n}{9-n}}_{x_n} ? \\
				& \frac{11+n}{9-n} = \frac{n(\frac{11}{n}+1)}{n(\frac{9}{n}-1)}\\
				\lim_{n \to \infty}& (\frac{11}{n}+1) = 1 \\
				\lim_{n \to \infty}& (\frac{9}{n}-1) = -1 \\
				\lim_{n \to \infty}& x_n = \frac{1}{-1} = -1 \\
			}
			\ilmath{
				\underbrace{\infty}_{uneig. GW} &= \lim_{n \to \infty} (11+n) + \lim_{\underbrace{n \to \infty}_{ex. nicht}} n = \xcancel{11+\dots} \\
				\underbrace{- \infty}_{uneig. GW} &= \lim_{n \to \infty} (9-n)
			}
		
	\ssect{Lemma: Quetschlemma}
		Seien $(x_n), (y_n)$ Folgen mit
		\ilmath{\lim_{n \to \infty} x_n =  \lim_{n \to \infty} y_n = a}
		und es gelte $x_n \leq z_n \leq y_n$ für \emph{"fast alle"} $n \in \bN$\\
		D.h. es ex. ein $N \in \bN$, so dass für alle $n \geq N$ die Ungleichung gilt. (Endlich viele Ausnahmen erlaubt.) \\
		
		\ilmath{ \text{Dann gilt für die Folge $(z_n)$: } \boxed{ \lim_{n \to \infty} z_n = a } }
		
		\sssect{Beispiel}
			Ist die Folge $((-1)^n \frac{1}{n})$ konvergent?
			\ilmath{
				- \frac{1}{n} \leq (-1)^n \frac{1}{n} \leq + \frac{1}{n} \\
				& \lim_{n \to \infty} (- \frac{1}{n}) = -1 \cdot 0 = 0 \\
				& \lim_{n \to \infty} \frac{1}{n} = 0 \im \lim_{n \to \infty} (-1)^n \frac{1}{n} = 0
			}
			
		\sssect{Beispiel}
			$x_n = \frac{a^n}{n!}$ mit $a \in \bR$, $a$ fest, $a > 0$ \\
			
			\ilmath{\lim_{n \to \infty} \frac{a^n}{n!} = 0}
			denn \\
			\ilmath{
				 x_n &= 0 \leq \frac{a^n}{n!} \\
				 \lim_{n \to \infty} x_n &= 0 \leq \frac{a^n}{n!} \leq \underbrace{\overbrace{y_n}^{\text{gesucht}}}_{\lim_{n \to \infty} y_n = 0} \\
			} \\
			für hinreichend großes $n$.\\
			
			\ilmath{
				\frac{a^n}{n!} = \frac{a}{n} \cdot \frac{a^{n-1}}{(n-1)!} & \leq \frac{1}{2} \cdot \frac{a^{n-1}}{(n-1)!} \\
				& \frac{1}{2} \cdot \frac{a}{n-1} \cdot \frac{a^{n-2}}{(n-2)!} \leq \frac{1}{2} \cdot \frac{1}{2} \cdot \frac{a^{n-2}}{(n-2)!}  \\
				y_n = \left(\frac{1}{2}\right)^{\underline{n-k}} \cdot \frac{a^{\underline{k}}}{(\underline{k})!} \text{ $k$ fest} & \\
			}
		
		Es gilt: $\frac{a^n}{n!} \leq y_n$ für hinreichend großes n
		und
		\ilmath{
			\lim_{n \to \infty} y_n &= \lim_{n \to \infty} \underbrace{(\frac{1}{2})^{n-k}}_{(\frac{1}{2})^n \cdot (\frac{1}{2})^{-k}} \cdot \underbrace{\frac{a^k}{k!}}_{\text{konst.}} \\
			&= \lim_{n \to \infty} (\frac{1}{2})^n \cdot \lim_{n \to \infty} (\frac{1}{2})^{-k} \cdot \lim_{n \to \infty} \frac{a^k}{k!} \\
			&= 0 \cdot (\frac{1}{2})^{-k} \cdot \frac{a^k}{k!} \\
			&\underline{= 0}
		}
	
	\ssect{Grenzwerte Rekursiv definierter Folgen}
		Kann man oft durch Lösen einer "Fixpunktgleichung" berechnen. \\
		
		$x_0, x_{n+1} = f(x_n,\dots)$ \\
		
		Idee: Falls $(x_n)$ konvergent ist, muss gelten: \\
		\ilmath{\lim_{n \to \infty} x_n = \lim_{n \to \infty} x_{n-1} = \lim_{n \to \infty} x_{n-2} = \dots = a}
		
		\sssect{Beispiel}
			$(x_n) x_0 = \frac{7}{5}, x_{n+1} = \frac{1}{3} (x^2_n + 2)$ \\
			Ü: ist mon. fallend, beschränkt -> konvergent \\
			
			\ilmath{
				\lim_{n \to \infty} x_n = a, \lim_{n \to \infty} x_{n+1} = a& \\
				\lim_{n \to \infty} x_{n+1} = \lim_{n \to \infty} \frac{1}{3} (x^2_n + 2) &= \frac{1}{3} \lim_{n \to \infty} (x^2_n + 2) \\
				&= \frac{1}{3} (\underbrace{(\lim_{n \to \infty} x_n)^2}_{a} + 2)
			}
	
		\paragraph{Fixpunktgleichung}
			$a = \frac{1}{3} (a^2 + 2)$ ges: a \\
			$3a = a^2 + 2 \iff a^2 - 3a + 2 = 0 \iff a_{1,2} = \frac{2}{3} \pm \sqrt{\frac{9}{4} - \frac{8}{4}} = \frac{3}{2} \pm \frac{1}{2}$ \\
			Lösungen sind $a_1 = 2, a_2 = 1$ \\
			Nur eine Lösung kann Grezwert sein, also betrachte Startwert und monotonieverhalten.. \\
			
			$\boxed{\im \lim_{n \to \infty} x_n = 1}$
		
	\ssect{Beispiel}
		$(x_n)$ mit $x_0 := c$ ($c \in \bR_{+} \text{ und } c > 0$, $c$ fest), $x_{n+1} := \frac{1}{2} (x_n + \frac{c}{x_n})$ \\
		(1) $(x_n)$ beschränkt \\
		(2) $(x_n)$ monoton \\
		$\im$ also \underline{$(x_n)$ konvergent.} (wichtige Vorraussetzung!) \\
		
		Sei \ilmath{\lim_{n \to \infty} x_n = a} Dann
		
		\ilmath{
			\underbrace{\lim_{n \to \infty} x_{n+1}}_{a} &= \lim_{n \to \infty} \frac{1}{2} (x_n + \frac{c}{x_n}) \\
			&= \boxed{\frac{1}{2} (a + \frac{c}{a}) = a} \text{ Fixpunktgleichung} \\
			&\iff 2a = a + \frac{c}{a} \iff a = \frac{c}{a} \\
			&\iff a^2 = c \iff a \ \sqrt{c}
		}
		Wir haben \ilmath{\lim_{n \to \infty} x_n = \sqrt{c}} erhalten \\
		
	\ssect{Bemerkung}
		Der Nachweis der Konvergenz der rekursiven Folge darf \underline{NICHT} weggelassen werden, \\
		denn z.B.: $x_0 := 2$, $x_{n-1} := x^2_n$ ist divergent. \\
		\ilmath{
			\lim_{n \to \infty} x_n = \infty
		}
	
		Ausnahme:
		\ilmath{
			\lim_{n \to \infty} x_n = a \\
			\underbrace{\lim_{n \to \infty} x_{n+1}}_{a} = \underbrace{\lim_{n \to \infty} x^2_n}_{a^2} \\
			\im a \in \set{0, 1}
		}
	
	\ssect{Definition}
		Sei $(a_k)$ eine reellwertige/komplexweritge Folge
		\ilmath{
			\underline{s_n} = \sum_{k=0}^{n} a_k = a_0 + a_1 + \dots + a_n
		}
		heißt n-te Partialsumme\\
			
		$(s_n)$ heißte (undendliche) \underline{Reihe}
		
		Schreibweise: $(s_n)^{\infty}_{n=0}$ bzw.
		\ilmath{
			(\sum_{k=0}^{n} a_k)^{\dots}_{\dots} \text{ bzw. } \sum_{k=0}^{\infty} a_k
		}
		sind Symbole für die Reihe.
		
	\ssect{Bemerkung}
		Reihen sind spezielle Folgen, also konvergent oder divergent.
		
	\ssect{Definition}
		Für eine konvergent Reihe wird der Grenzwert auch \underline{Wert der Reihe} genannt.
		
		Schreibweise:
		\ilmath{
			\lim_{n \to \infty} s_n = \lim_{n \to \infty} \sum_{k=0}^{n} a_k \text{ bzw. } \sum_{k=0}^{\infty} a_k
		}
		Symbol für den Wert der Reihe, falls er existiert.
		
	\ssect{Beispiel}
		Teleskop-Reihe
		\ilmath{
			\sum_{k=1}^{\infty} (\frac{1}{k} - \frac{1}{k+1})
		}
		ist konvergent, Wert der Reihe ist
		\ilmath{
			\sum_{k=1}^{\infty} (\frac{1}{k} - \frac{1}{k+1}) = 1
		}
		\ilmath{
			&\lim_{n \to \infty} s_n = \lim_{n \to \infty} \sum_{k=1}^{\infty} (\frac{1}{k} - \frac{1}{k+1}) = \lim_{n \to \infty} \left((1-\frac{1}{2}) + (\frac{1}{2} - \frac{1}{3}) + (\frac{1}{3} - \frac{1}{4}) + \dots + (\frac{1}{n} - \frac{1}{n+1})\right) \\
			&\lim_{n \to \infty} (1 - \frac{1}{n+1}) = 1 - 0 = 1
		}
		(setze Klammern so um das viele Terme sich selbst "wegkürzen")
		
	\ssect{Beispiel}
		Geometrische Reihe\\
		\ilmath{
			\sum_{k=0}^{\infty} q^k \text{ ist für } |q| < 1 \text{ konvergent,}
		}
		Wert der Reihe für $|q| < 1$
		\ilmath{
			\sum_{k=0}^{n} q^k = \underline{\frac{1}{1-q}} \text{ für } |q| < 1
		}
		Berechnung
		\ilmath{
			s_n &= q^0 + q^1 + \dots + q^n | \cdot q \\
			q s_n &= q^1 + q^2 + \dots + q^{n+1} \\
			(1-q) s_n &= q^0 - q^{n+1} \\
		}
		Es folgt
		\ilmath{
			s_n &= \frac{1-q^{n+1}}{1-q} = \frac{1}{1-q} (1-q^{n+1}) \\
			\im \lim_{n \to \infty} s_n &= \frac{1}{1-q} \lim_{n \to \infty} (1-q^{n+1}) \\
			&= \frac{1}{1-q} (1- \lim_{n \to \infty} q^{n+1})
		}
	
	\ssect{Definition}
		Rechenregeln für Reihen: $\frac{1}{1-q}$\\
		\underline{Konvergente} Reihen kann man addieren, subtrahieren mit einem Skalar multiplizieren, wie endliche Summen. \\
		
		ABER: Das gilt im Allgemeinen nicht für das Multiplizieren.

\end{document}
