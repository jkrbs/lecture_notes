\documentclass{../../meta/tudscript}
\begin{document}

\ssect{Lemma: Gleichheit von Abbildungen}

Seien X, Y, Z, W Mengen und

\begin{itemize}

\item
  \(f: X \rightarrow Y\)
\item
  \(g: Y \rightarrow Z\)
\item
  \(h: Z \rightarrow W\)
\end{itemize}

Dann gilt:

\begin{enumerate}

\item
  \(id_y \circ f = f = f \circ id_x\)
\item
  \((h \circ g) \circ f = h \circ (g \circ f)\)
\end{enumerate}

\sssect{Beweis}

\paragraph{Vorbemerkung}

Zwei Abbildungen \(u,v: A \rightarrow B\) sind genau dann gleich, wenn

\(u(a) = v(a)\) für alle \(a \in A\)

\paragraph{1.}}
Für alle
\(x \in X: (id_y \circ f)(x) = f(id_y(x)) = f(x) = id_x(f(x)) = (f \circ id_x)(x)\)

\paragraph{2.}

Für alle
\(x \in X: ((h \circ g) \circ f)(x) = (h \circ g)(f(x)) = h(g(f(x))) = h((g \circ f)(x)) = (h \circ (g \circ f))(x)\)

\ssect{Definition: Invertierbarkeit von Abbildungen}

Eine Abbildung \(F: X \leadsto Y\) heißt

\begin{enumerate}
\def\labelenumi{\arabic{enumi}.}

\item
  linksinvertierbar
  :\(\iff \exists g: Y \rightarrow X: g \circ f = id_x\)
\item
  rechtsinvertierbar
  :\(\iff \exists g: Y \rightarrow X: f \circ g = id_y\)
\item
  invertierbar, genau dann, wenn links- und rechtsinvertierbar
\end{enumerate}

\ssect{Bemerkung: Invertierbarkeit und In-/Surjektivität}

Sei \(f: X \rightarrow Y\) eine Abbildung. Dann gilt

\begin{enumerate}
\def\labelenumi{\arabic{enumi}.}

\item
  Wenn f linksinvertierbar, dann ist f injektiv.
\item
  Wenn f rechtsinvertierbar, dann ist f surjektiv.
\item
  Wenn f injektiv und \(X \neq \emptyset\)
\end{enumerate}

\sssect{Beweis 1.}}

Sei \(g: Y \rightarrow X\) und \(g \circ f = id_x\). Sei nun
\(x,y \in X\) mit \(f(x) = f(y)\), dann folgt
\(x = id_x(x) = (g \circ f)(x) = g(f(x)) = g(f(y)) = id_x(y) = y\)

\sssect{Beweis 2.}

Sei \(g: Y \rightarrow X\) und \(f \circ g = id_y\). für alle
\(y \in Y\), betrachte x = g(y) und beobachte
\(y = id_y(y) =(f \circ g)(y) = f(g(y)) = f(x)\)

Also \(f(X) = Y\), d.h. f ist surjektiv.

\sssect{Beweis 3.}

Wähle \(x_0 \in X\). Dann ist

\(g := {(x,y) \in X \times Y | (x,y) \in f} \cup ((Y \ f(x)) \times {x_0}\)

eine Abbildung von \(Y \rightarrow X\), und es gilt \(g \circ f =id_x\)

\ssect{Proposition}

Sei \(f: X \rightarrow Y\). Dann sind folgende Aussagen äquivalent.

\begin{enumerate}
\item
  f ist bijektiv
\item
  f ist invertierbar
\item
  \(\exists g: Y \rightarrow X: g \circ f = id_x \wedge f \circ g = id_y\)
\end{enumerate}

Ist f bijektiv, so ist die Abblidung g eindeutig bestimmt. mit

\(g = {(x,y) \in Y \times X | (x,y) \in f}\)

\ssect{Definition}

Seien X, Y Mengen. Die Menge alle Abbildungen \(X \rightarrow Y\)

\(Y^X = {f \in \wp ((x,y)) | f \)Abbildung von X \( \rightarrow Y}\)

\ssect{Satz}

Sei X Menge. Dann ist die Abbildung

\(\Phi : {0,1}^X \rightarrow \wp (x), f \mapsto f^{-1} ({1})\)

\sssect{Beweis}

für jede Teilmenge \(A \subseteq X\) betrachten wir die
charakteristische Funktion.

(Indikatorfunktion) \(\chi_1 : x \mapsto {0,1}\) mit

\begin{equation}
  \chi_1 (x) =
  \begin{cases}
    1 falls X \in A
    0 falls X \notin A
  \end{cases}
\end{equation}

d.h. \(\chi_1 (x) = (A \times {1} \cup ((X \ A_x{0}\)

Definiere: \(\Psi : \wp (X) \rightarrow {0,1}^X, A \mapsto \chi_A\)

Für jede Teilmenge von \(A \subseteq X\) gilt

\(\Phi (\Psi (A)) = \Phi(\chi_A) = (\chi_A)^{-1} ({1}) = A\)

also \(\Phi \circ \Psi = id_{\wp (x)}\)

Für jedes \(f \in {0,1}^X\) ist
\(\Psi(\Phi(f)) = \chi_{\Phi(f)} = \chi_{f^{-1}(1)} = f\), und daher
\(\Psi \circ \Phi = id_{{0,1}^X}\)

Somit ist \(\Phi\) invertierbar, also bijektiv nach Proposition 4.12.

\ssect{Definition: Tupel}

Sei X Menge. Für \(n \in \mathbb{N}\) bezeichnet man die Elemente von

\(X^n = X^{{1,n}}\) als n-Tupel bzw. nX

Schreibweise für \(x \in X^n\)

d.h. \(x: {1,...,n} \rightarrow X: x = (x_1,...,x_n)\)

\(X_i = X(i)\) für \(i \in {1,...,n}\)

Für \(m,n \in \mathbb{N}\) heißen die Elemente von

\(X^{m \times n} = X^{{1,...,m} \times {1,...,m}}\)

auch \(m \times n\) Matrizen (Sigular: Matrix).

Schreibweise für \(A \in X^{m \times n}\)

\(A= \begin{bmatrix}  a_{11} & \dots & a_{1n} \\  \dots & \dots & \dots \\  a_{m1} & \dots & a_{mn}\\  \end{bmatrix}\)

wobei \(A_{ij} := A(i,j)\) für \(i \in {1,...,m}\) und \( j \in {1,...,n}\)

\ssect{Definition: Familien}

Sei I Menge. Ist X eine Menge, so nennt man \(x: I \rightarrow X\) auch
\emph{Familie} in X und dann schreibt man \(x = (X_i)_{i \in I}\), wobei
\(x_i := x(i)\) für \(i \in I\).

Eine Mengenfamilie ist eine Familie \((M_i)_{i \in I}\) in der
Potenzmenge \(\wp(Y)\) einer Menge Y, d.h. für \(i \in I\) ist
\(M_i \subseteq Y\).

Für die Mengenfamilie \((M_i)_{i \in I}\) definieren wir:

\begin{enumerate}

\item
  die Vereinigung


\(\cup_{i \in I} = \cup {M_i | i \in I} (= {y \in Y| \exists i \in I: y \in M_i})\)


\item
  die Schnittmenge, dalls \(I \neq 0\) \$
  \(\cap_{i \in I} = \cap {M_i | i \in I} (= {y \in Y | \forall i \in I: y \in M_i})\)
\item
  das kartesiche Produkt
\end{enumerate}

\(\prod_{i \in I} M_i = {(x_i)_{i \in I} | \forall i \in I: x_i \in M}\)

\(= {(x_i)_{i \in I} \in (\cup_{i \in I} M)^I | \forall i \in I: x \in M_i}\)

\sect{Natürliche Zahlen und vollständige Induktion}

\ssect{Definition: Natürliche Zahlen}

Der Nachfolger einer Menge X ist:

\(X^+ = x \cup \{X\}\)

Die natürlichen Zahlen sind nun rekursiv definiert:

\begin{itemize}

\item
  \(0 := \emptyset\)
\item
  \(1 := 0^+ = \emptyset \cup \{\emptyset\} = \{\emptyset\} = \{0\}\)
\item
  \(2 := 1^+ = 1 \cup \{1\} = \{0\} \cap \{1\} = \{0,1\}\)
\item
  \(3 := 2^+ = 2 \cup \{2\} = \{0,1,2\}\)
\end{itemize}

\end{document}
