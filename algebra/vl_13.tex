\documentclass{../tudscript}
\author{Jakob Krebs}
\title{Mathe VL 13}

\begin{document}
\ssect{Rückblick}
\begin{itemize}
\item Interpolationsbedingung zur Unterteilung $ x_0 < x_1 < \ldots < x_n$
\ilmath{(x_0, y_0), (x_1, y_1), \ldots}
\item Interpolationspolynom zu den Interpolationsbedingungn $(x_i, y_i), (i = 1, 2, \ldots, n)$
\ilmath{p(x) = a_0 + a_1 x + \ldots + a_n x^n}

\item Polynom gegeben, Werte gesucht \underline{Polynomauswertung} (Transformation T)
\item Werte gegeben, Polynom gesucht \underline{Polynominterpolation} (inverse Transformation $T^{-1}$)

\end{itemize}

\ssect{Vandermonde - Determinante}
\begin{itemize}
\item Vandermonde - Matrix über einem Körper K:
\ilmath{V(x_0, x_1, \ldots, x_n) = \m{1 & x_0 & x_{0}^2 & \ldots & x_{0}^n\\
                                      1 & x_1 & x_{1}^2& \ldots & x_{0}^n\\
                                      \vdots & \vdots  & & \vdots \\
                                      1 & x_n & x_{n}^2 & \ldots & x_{n}^n}}
\item Vandermonde - Determinante

\ilmath{det V(x_0, x_1, \ldots, x_n) = \prod_{i > j, i, j \in {0, 1, \ldots, n}} (x_i - x_j)}
\item \ilmath{|\Set{x_0, x_1, \ldots, x_n}| = n + 1 \implies det V(x_0, x_1, \ldots, x_n) \neq 0}

\end{itemize}

\ssect{Transformation}
Transformation:
\ilmath{T: K^{n+1} \rightarrow K^{n+1}: \m{a_0 \\ a_1 \\ \vdots \\ a_n} \mapsto \m{y_0 \\ y_1 \\ \vdots \\ y_n}}
Die Transformation beschriebbt die Polynomauswertung von $p(x) = a_0 + a_1 x + \ldots + a_n x^n$ an den Stützstellen $x_0, x_1, \ldots, x_n$

\ilmath{V(x_0, \ldots, x_n) \m{a_0 \\ a_1 \\ \vdots \\ a_n} = \m{y_0 \\ y_1 \\ \vdots \\ y_n}}

Wegen $det(V(x_0, \ldots, x_n)) \neq 0$ exisiter $V(x_0, \ldots, x_n)^{-1}$, also ist T bijektiv.

\ssect{Bemerkung}
\begin{itemize}
\item Zur Polynominterpolation mit $T^{-1}$ benötigt man $V(x_0, x_1, \ldots, x_n)^{-1}$.
\item Aufwand zur Berechnung von $V(x_0, \ldots, x_n)^{-1}$ reduzieren, indem man $x_0, \ldots, x_n$ geschickt auswählt.
\end{itemize}
\ssect{Definition}
Sei K ein Körper, $\omega \in K \setminus \Set{0}$. $\omega$ heißt \underline{primitive (n+1)te Einheitswurzel} in K, wenn $\omega$ in der multiplikativen Gruppe von K die Ordnung $n + 1$ hat.

\ssect{Bemerkung}
Für primitive (n+1)te einheitswurzel $\omega$ gilt:
\ilmath{|\rangle \omega \langle| = |\Set{\omega^0, \omega^1, \ldots, \omega^n}| = n+1}
Man kann im Exponenten modulo n+1 rechnen.

\ssect{Satz}
Ist $\omega$ eine primitive (n+1)te Einheitswurzel im Körper K, dann gilt:
\ilmath{\sum_{k = 0}^n (\omega^\alpha)^k = \begin{cases} n + 1, \alpha = 0 \\ 0, \alpha \in \Set{1, 2, \ldots, n}\end{cases}}

\% hier Beweis einfügen

\ssect{Bemerkung}
\ilmath{\omega^\alpha -1 \neq 0}, denn $\alpha \neq 0$
\ssect{Bemerkung}
Im Körper $\bC$ der komplexen Zahlen gibt es für jedes $n > 0$ primitive (n+1)te Einheitswurzeln:
\ilmath{\omega = e^{\frac{2 \pi}{n + 1} i}}
ist eine primitive (n + 1)te Einheitswurzel.
$\omega^j$ mit $ggT(j, n + 1) = 1$ sind primitive (n+1)te Einheitswurzeln.

\ssect{Bemerkung}
Bei $ggT(p, n) = 1$, p Prim.

\ilmath{GF(p^k)} mit $n+1\ p^k -1$ ($k > 0$, k minimal) ist der Körper mit kleinstem k, sodass $(GF(p^k)$ prmitive (n+1)te Einheitswurzeln enthält.

\ssect{DFT}
Polynomauswertung an den Stützstellen $\omega^0, \omega^1, \ldots, \omega^n$ für eine primitive (n+1)te Einheitswurzel $\omega$.

\underline{DFT Matrix $D_{n+1, \omega}$}

\ilmath{D_{n+1 , \omega} + V(x_0, x_1, \ldots, x_n) = \m{1 &  \omega^0 & (\omega^0)^2 & \ldots \\
                                                         1 &  \omega^1 & (\omega^1)^2 & \ldots \\
                                                         \vdots & \vdots & \vdots & \vdots}\\
        \implies D_{n+1 \omega} = (\omega^{i \cdot j})_{i, j = 0, 1, \ldots, n}}

\ssect{Beispiel}

$K = \bC, p(x) = 1 + 2x + x^3 \rightarrow \boxed{n+1 = 4, \omega = i}$

\ilmath{D_{4, \omega} = \m{1 & 1 & 1 & 1 \\
                           1 & i & -1 & -i \\
                           1 & -1 & 1 & -1 \\
                           1 & -i & -1 & i}}

\ssect{Bemerkung}
\ilmath{D_{n+1, \omega}^{-1} = \frac{1}{n+1} D_{n+1, \omega^{-1}}}
\sssect{Beweis}
Man kann zeigen, dass
\ilmath{D_{n+1, \omega} \cdot (\frac{1}{n+1} D_{n+1}, \omega^{-1}) \overset{=}{\star} E_n}
gilt.

\% too lazy to tex
\ssect{inverse DFT Transformation}
Polynominterpolation mit den INterpolationsbedingungen
$(\omega_0, y_0), \ldots, (\omega_n, y_n)$ für eine primitive (n+1)te Einheitswurzel $\omega$:

Inverse der DFT Matrix $D_{n+1, \omega^{-1}}$
\ilmath{D_{n+1, \omega}^{-1} = \frac{1}{n+1} D_{n+1, \omega^{-1}}}
\end{document}
