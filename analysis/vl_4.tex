\documentclass{../tudscript}
\author{Jakob Krebs}
\title{Mathe VL 4}

\begin{document}
    \ssect{Beispiel: Geometrische Reihe}
    %%TIKZ

    \ilmath{2 A = 1^2 + (\frac{1}{2})^2 +(\frac{1}{4})^2 + \ldots + (\frac{1}{2^k})^2}
    \ilmath{q = \frac{1}{4} = \frac{1}{1 - \frac{1}{4}} = \frac{1}{\frac{3}{4}} = \frac{4}{3} = 2A}

    Beispiel Perioden
    \ilmath{0, 4\bar{3} &= \frac{4}{10} + \frac{3}{100} + \frac{3}{1000} + \ldots \\
                        &= \frac{4}{10} + \frac{3}{10^2} + \frac{3}{(10^2)^2}\\
                        &= \frac{4}{10} + \frac{1}{30}  = \frac{12 + 1}{30} = \frac{13}{30}}
    Beispiel: unmögliche Periode
    \ilmath{0,4\bra{9} &= \frac{4}{10} + \frac{9}{100} + \frac{10}{9} \\
                       &= \frac{4}{10} + \frac{1}{10} = \frac{5}{10} = \frac{1}{2} = 0,5}

    \ssect{Konvergenz der geometrischen Reihe}
    %TODO missing notes
    \ilmath{\sum^{\infty}_{k = 1} \frac{1}{k}}
    ist divergent
    \ilmath{\lim_{n \to \infty} \sum^{n}_{k = 1} \frac{1}{k}}
    existiert nicht.
    \ilmath{s_n &= \frac{1}{1} + \frac{1}{2} + \frac{1}{3} + \frac{1}{4} + \ldots + \frac{1}{n} \\
                &> 1 + \frac{1}{2} + \frac{2}{4} + \frac{4}{8} + \ldots
                &\implies \lim_{n \to \infty} s_n = \infty}
    \sect{Allgemeine harmonische Reihe}
    \ilmath{\sum_{k = 1}^\infty \frac{1}{k^\alpha} (\alpha \text{ fest})}
    \ilmath{\begin{cases} \alpha > 1 \implies \text{ R konv.}\\
                          \alpha \leq 1 \implies \text{ R div.}\end{cases}}
    \ssect{Beispiel}
        \ilmath{\sum_{k = 1}^\infty \frac{1}{k^2}}
        ist konvergent.
        Beweis mit Monotoniekriterium für Folgen:
        \begin{enumerate}
        \item \ilmath{\sum_{k = 1}^n \frac{1}{k^2}} ist monoton wachsend
        \item \ilmath{\sum_{k = 1}^n \frac{1}{k^2}} ist beschränkt
        \end{enumerate}
        $\implies$ Folge ist konvergent.
    \sect{Exponentioalreihe}
    Die Exponentialreihe 
    \ilmath{\sum_{k = 1}^\infty \frac{1}{k!} = \lim_{n \to \infty} (1 + \frac{1}{n})^n =: e}
    ist konvergent.
    \sect{Konvergenzkriterien für Reihen}
    \begin{enumerate}
        \item Kriterien für Folgen gelten auch für Reihen
        \item Hauptkriterium/Nullfolgenkriterium
        \item Leibnizkriterium (für alternierende Reihen)
        \item Kriterien für absolute Konverenz von Reihen
    \end{enumerate}

    \ssect{Hauptkriterium}
        \ilmath{\sum_{k = 0}^\infty a_k \text{ konvergent} \implies (a_k) \text{ Nullfolge} \\
                \lim_{k \to \infty} a_k \neq 0 \implies \sum_{k = 1}^\infty a_k \text{ nicht konvergent}}

        Beispiel:
        \ilmath{\sum_{i = 0}^\infty \frac{3 k^2 + 1}{4 k^2 -1}}
        divergiert, aber
        \ilmath{\sum \frac{1}{k}}
        divergiert und 
        \ilmath{\frac{1}{k}}
        Nullfolge.

    \ssect{Behauptung}
        \ilmath{\sum a_k \text{ konvergiert } \implies (a_k) \text{ Nullfolge}}

        \ilmath{s_n = \sum_{k = 0}^n a_k, s_{n+1} = \sum_{k = 0}^{n+1} a_k, s_{n+1} = s_n + a_{n+1}}
        \ilmath{s = \lim_{n \to \infty} s_n = \lim_{n \to \infty} s_{n+1}}
        \ilmath{\lim_{n \to \infty} a_{n+1} = \lim_{n \to \infty} s_{n + 1} - \lim_{n \to \infty} s_n = s - s = 0}
    \ssect{Kriterium für alternierende Reihen}
    z.B.:
    \ilmath{\sum (-1)^k}

    \ilmath{\sum (-1)^k (a_k) \implies \text{ Reihe ist konvergent}}
    wobei $(a_k)$ eine Streng monoton fallende Nullfolge ist. mit $(a_k) \geq 0$
    
    \ssect{Definition: absolute Konvergenz}
        Reihe
        \ilmath{\sum_{k = 0}^\infty }
        heißt absolut konvergent, wenn 
        \ilmath{\sum |a_k|}
        konvergent ist.
        
        Beispiel:
        \ilmath{\sum (-1)^k \frac{1}{k}}
        ist konvergent, aber nicht absolut konvergent.

        \ilmath{\sum (-1)^k \frac{1}{k^2}}
        ist konvergent und absolut konvergent.

    \ssect{Satz: Folge aus absoluter Konvergenz}
        Wenn eine Reihe absolut konvergent ist, ist sie auch konvergent.

    \ssect{Bemerkung: multiplikation konvergenter Reihen}
        Absolut konvergente Reihen kann man multiplizieren, wie endliche Summen.
        (konvergente nicht.)

    \ssect{Quotienten und Wurzelkriterium}
        Quotientenkriterium (QK):
        \ilmath{\lim_{n \to \infty} |\frac{a_{k+1}}{a_{k}}| < 1&\implies \sum a_k \text{ist (absolut) konvergent}\\
        \lim_{n \to \infty} |\frac{a_{k+1}}{a_{k}}| > 1&\implies \sum a_k \text{ist divergent}\\
        \lim_{n \to \infty} |\frac{a_{k+1}}{a_{k}}| = 1&\implies \text{keine Aussage}}
        
        Wurzelkriterium (WK):
        \ilmath{\lim_{n \to \infty} \sqrt[k]{|a_k|} < 1 &\implies \sum a_k \text{ ist (absoult) konvergent}\\
        \lim_{n \to \infty} \sqrt[k]{|a_k|} > 1 &\implies \sum a_k \text{ ist divergent}\\
        \lim_{n \to \infty} \sqrt[k]{|a_k|} = 1 &\implies \sum a_k \text{ keine Aussage}}

\end{document}
