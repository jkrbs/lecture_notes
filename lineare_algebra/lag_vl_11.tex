\documentclass{../tudscript}
\begin{document}
   \sect{Matrizen}
        \ssect{Determinanten}
            $det: K^{n \times n} \rightarrow K, A \mapsto \underbrace{det(A=}_{Determinante der n \times n Matrix}$
            Det (A) gibt den Faktor an, um den sich dads "Volumen" einer Figur ändert, wenn auf die Figur die linare Abbildung: $x \rightarrow Ax$ anwendet.
        \ssect{Beispiele}
            
          \begin{enumerate}
          \item \begin{equation*}
            det(  \begin{pmatrix}
                a & b \\
                c & d \\        
                \end{pmatrix} ) = ad -bc
            \end{equation*}            
            %%here some TIKZ will appear TODO
        
         \end{enumerate}            
    \ssect{Bermerkung}
      $det( \begin{pmatrix} 
        v_1 & w_1 \\
        v_2 & w_2 \\
    \end{pmatrix} = 0 \iff v_1 w_2 - v_2 w_1 = 0 \iff v_1 v_2 = v_2 v_1 \iff \frac{v_1}{w_1} = \frac{v_2}{w_2} = k \iff \ve{v_1}{v_2} = \ve{w_1 k}{w_2 k} = k \ve{w_1}{w_2} \iff \vec{v}, \vec{w}$ sind linear abhängig.

    $det (\begin{pmatrix} v_1 & w_1 \\ v_2 && w_2 \end{pmatrix}) \neq 0 \iff \vec{v}, \vew{w} lin unab.$

    \subsectiobn{Beispiel}
        $n = 3$
        \begin{equation*}
        det( \begin{pmatrix}
            a & b & c \\
            d & e & f \\
            g & i & h \\
        \end{pmatrix} ) = a \cdot e \cdot i + b \cdot f \cdot g + c \cdot d \cdot h - g \cdot e \cdot c - h \cdot f \cdot a - i \cdot d \cdot b  = a(ei -hf) + d (ib -ch) + g (bf- ec) = a det()\begin{pmatrix} e & f \\ h & i \\ \end{pmatrix} - d det(\begin{pmatrix} b & c \\ h & i \\ \end{pmatrix}) + g det (\begin{pmatrix} b & c \\ e 6 f \\ \end{pmatrix})
        \end{equation*}
        Für Matrizen mit n > 3 gibt es \underline{KEINE!!11!elf} derartige Brechnungsmethode.
\ssect{Definition}
    Sei $A = (a_{ij}) \in K^{n \times n}. Dann gilt:
\begin{enumerate}
    \item $n = 1 \implies det((a_{11})) := a _{11}$
    \item $n > 1 \implies det(A) = \sum_{i = 1}^n (-1) a_{i1} det(A_{i1})
    wobei $A_{i1}$ aus A durch Streichen der i-ten Zeile und 1-ten Spalte entsteht.
\end{enumerate}
\ssect{Bemerkung: Vorzeichenregel}
erste Zeile +. Dann alternierend.
\ssect{Bemerkung}
Die Berechnungsvorschrift aus der Definition nennt man Entwicklung nach der 1-ten Spalte 

\ssect{Satz: es ist egal nach welcher Spalte wir entwickeln}
Bezeichnung $a \in A^{n \times n} A_{ij}  \in K^{(n-1) \times (n-1)}$ entsteht aus A durch Streichen der i-ten Zeile und j-ten Spalte.

Satz: Sei $A = (a_{ij})_{n \times n}  (n>1)$, dann gilt.

\begin{enumerate}
\item $det (A) = \sum_{i = 1}^n (-1)^{i+j} a_{ij} \cdot det (A_{ij})$ Einteilung nach der j-ten Spalte.
\item $det (A) = \sum_{j = 1}^n (-1)^{i+j} a_{ij} \cdot det (A_{ij})$ Einteilung nach der i-ten Zeile.
\end{enumerate}
Beiweis möglich mittel Vollständiger Induktion: Klar! üb!^\trademark

\ssect{Folgerung}
Matrizen, die eine Nullzeile oder eine Nullspalte haben, haben die Determinante 0.

\ssect{Bermerkung}
\ilmath{det(E^{n \times n}) = 1}
\ssect{Umformung von Determinanten}
\begin{enumerate}
    \item Es gilt $det A = det A^T$

    \item A' entstehe aus A durch Multiplikation einer Zeile mit $k \in K$. Dann: $det A' = k det A$
    
    \item $A' = kA \implies det (A') = k^n det A$

    \item A' entstehe aus A durch Vertauschen zweier Zeilen. Dann $det A' = - det A$
    
    \item A enthält zwei gleiche Zeilen $\implies det A = 0$

    \item A' entsteht aus A, indem man zu einer Zeile das k-fache einer bel. \underline{anderen} Zeile addiert, dann:
    \ilmath{det A' = det A}
    
    \item 2 bis 6 gilt auch für Spalten statt zeilen
\end{enumerate}
\ssect{Multiplikationssatz}
Sei $A, B \in K^{n \times n} \implies det(AB) = det A \cdot det(B)$.
\ssect{Folgerung}
$A \in K^{n \times n}: \boxed{A^{-1} ex \iff det (A) \neq 0}$

Beweis: $\implies A A^{-1} = e \implies det (A) det (A^{-1}) = 1$
\end{document}
