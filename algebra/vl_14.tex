\documentclass{../tudscript}
\author{Jakob Krebs}
\title{Mathe VL 14}

\begin{document}
\ssect{Beispiel}
$K = \bC, n+1 =4, \omega = i, p(x) = 1+2x+x^3 \leftrightarrow (a_0, a_1, a_2, a_3) = (1, 2, 0, 1)$

Transformation T mit der DFT Matrix $D_{4,i}$
\ilmath{\m{1 & 1 & 1 & 1 \\ 1 & i & -1 & -i \\ 1 & -1 & 1 & -1 \\ 1 & -i & -1 & i} \cdot \m{1 \\2 \\ 0 \\ 1} = \m{4\\1+i\\-2 \\ 1-i}=\m{p(1) \\ p(2) \\ p(0) \\ p(-i)}}

Transformation $T^{-1}$ mit der i(nverse)DFT Matrix $\frac{1}{4} D_{4, i^{-i}}$
...
\ilmath{\frac{1}{4}D^{-1} \* \m{4 \\1+i\\-2\\1-i} = \m{1\\2\\0\\1} = \m{a_0 \\ a_1 \\ a_2 \\a_4}}

Bemerkung
\ilmath{(D_{4,i})^{-1} = \frac{1}{4} D_{4,i^{-1}}}

\ssect{Beispiel in GF(5)}
$K = GF(5), n+1 = 4, \omega = 2, p(x) = 1 + 2x+3x^2 + 4x^3 \leftrightarrow (1,2,3,4)$

Transformation $T$ mit der DFT Matrix $D_{4,2}$
\ilmath{\m{1 & 1 & 1 & 1\\ 1 & 2 & 4 &3 \\ 1 & 4 & 1 & 4\\1 & 3 & 4 & 2} \cdot \m{1 \\ 2 \\ 3 \\ 4} = \m{0 \\ 4\\3\\2} = \m{p(1) \\ p(2) \\ p(4) \\ p(3)}}

Transformation $T^{-1}$ mit der DFT Matrix $\frac{1}{4} \* D_{4,3}$
\ilmath{\frac{1}{4}\m{1 & 1 & 1 & 1\\ 1 & 3 & 4 &2 \\ 1 & 4 & 1 & 4\\1 & 2 & 4 & 3} \cdot \m{0 \\ 4 \\ 3 \\ 2} = \m{1 \\ 2\\3\\4} = \m{a_0 \\ a_1 \\ a_2 \\ a_3}}

Bemerkung
\ilmath{(D_{4,i})^{-1} = \frac{1}{4} D_{4,i^{-1}}}

\sect{Polynommultiplikation mit DFT}

Gegeben: $a(x), b(x)$, Gesucht: $a(x)\cdot b(x)$


Polynomauswertung mit DFT. Die Stützstellen sind frei wählbar $\implies (a(x_0), \ldots), (b(x_0), \ldots) \implies$ Koordinatenweise Multipliation
$(a(x_0) \* b(x_0), a(x_1) \* b(x_1), \ldots) \implies$ iDFT Polynominterpolation $\implies a(x) \* b(x)$

\sect{schnelle Multiplikation großer natürlicher Zahlen}

Gegeben: $a, b$, Gesucht: $a\*b$

\ilmath{a &= a_n 2^n + \ldots a_1 2 + a_0 \implies p_1(x) \\
  b &= b_n 2^n + \ldots b_1 2 + b_0 \implies p_2(x)\\
p(x) = p_1(x) \* p_2(x) \implies p(2) = a \cdot b}

\ssect{Bemerkung}
Noch schneller geht es für $n + 1 = 2^r$. Dann kann man mit FFT, iFFT und einen Algorithmus von Cooley-Tukey anwenden.

\ssect{Beispiel}
\ilmath{p(x) &= x^3 + x^2 + 2x + 1 = (x^2 + 1) + x (x^2 + 2)\\
  &= (y+1) + y \* (y+2), y = x^2}
$\omega = 1, \omega^1 = i, \omega^2 = -1, \omega^i = -i$ auswerten:
\ilmath{p(1) &= (1^2 + 1) + 1 (1^2 +2)\\
  p(i) &= (i^2 + 1) + i (i^2 +2)\\
  p(-1) &= ((-1)^2 + 1) -1 ((-1)^2 +2) (1^2 =(-1)^2)\\
  p(-i) &= ((-i)^2 + (-i)) + (-i) ((-i)^2 +2)}


Allgemein:


\% musste semps fixen. falls jemand Lust hat zu ergänzen
\ssect{Bemerkung zum Rechenaufwand}
Es gilt $A(r) = \frac{3}{2} 2^r * r = \frac{3}{2} (n+1) \* log_2(n+1)$

Beweis via Induktion

\ssect{Bemerkung}
Aufwand der FFT
\ilmath{\mathcal{O} (n \* log(n))}

Aufwand der iFFT
\ilmath{\mathcal{O} (n \* log(n))}

\end{document}
