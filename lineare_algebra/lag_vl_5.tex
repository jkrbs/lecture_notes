\documentclass{../tudscript}
\begin{document}
\newcommand{\ilmath}[1]{\begin{flalign*}#1\end{flalign*}}
\newcommand{\vect}[1]{\begin{pmatrix}#1\end{pmatrix}}

\#3. Vektorräume z.B.
\(\left(\left\{\begin{pmatrix}1\\2\end{pmatrix} \mid a,b \in \mathbb{R}\right\}, \boxed{+}, \boxed{\cdot}\right)\)

Vekturräume werden im folgenden mit VR abgekürzt.

\#\#Definition Sei K ein Körper. Ein K-VR (VR über K) ist eine Struktur
bestehend aus

\begin{itemize}
\tightlist
\item
  einer Menge \(V \neq \emptyset\)
\item
  einer Addition
  \(\boxed{+}: V \times V \rightarrow V, (v_1, v_2) \mapsto v_1 \boxed{+} v_2 = v_1 + v_2\)
\item
  einer Skalarmultiplikation
  \(\boxed{\cdot}: K \times V \rightarrow V, (k, v) \mapsto k \boxed{\cdot} v = k \cdot v = kv\)
\end{itemize}

wobei die Vektorraumaxiome (V1) bis (V10) erfüllt sind.

\begin{itemize}
\tightlist
\item
  (V1) Zu je zwei \(v_1, v_2 \in V\) existiert ein eindeutig bestimmter
  \(v_1 \boxed{+} v_2 \in V\)
\item
  (V2) Für alle \(v_1,v_2,v_3 \in V\) gilt
  \((v_1 \boxed{+} v_2) \boxed{+} v_3 = v_1 \boxed{+} (v_2 \boxed{+} v_3)\)\\
\item
  (V3) Für alle \(v_1,v_2 \in V\) gilt (v\_1 \boxed{+} v\_2) = (v\_2
  \boxed{+} v\_1)
\item
  (V4) Es exisitert ein Vektor \(0 = \in V\) mit
  \(v \boxed{+} 0 = v = 0 \boxed{+} v\)
\item
  (V5) Zu jedem \(v \in V\) existiert ein \(\boxed{-v} \in V\) mit
  \(v \boxed{+} \boxed{-v} = 0 = \boxed{-v} \boxed{+} 0\)
\item
  (V6) Zu jedem \(k \in K\) und \(v \in V\) existiert ein einduetig
  bestimmtes \(k \boxed{\cdot} v \in V\)
\item
  (V7) \(1 \boxed{\cdot} v = V\) für alle \(v \in V\) (wobei 1 das
  Einselement des Körpers K ist)
\item
  (V8)
  \((k_1 \cdot k2) \boxed{\cdot} v = k_1 \boxed{\cdot} (k_2 \boxed{\cdot} v)\)
  für alle \(k_1, k_2 \in K, v \in V\)
\item
  (V9) (k\_1 + k\_2) \boxed{\cdot} v = (k\_1 \boxed{\cdot} v) \boxed{+}
  (k\_2 \boxed{\cdot} v)\$ für alle \(v_1, v_2 \in V, k_1, k_2 \in K\)
\item
  (V10)
  \(k \boxed{\cdot} (v_1 \boxed{+} v_2) = (k \boxed{\cdot} v_1) \boxed{+} (k \boxed{\cdot} v_2)\)
  für alle \(v_1, v_2 \in V, k \in K\)
\end{itemize}

\hypertarget{beispiele}{%
\ssect{Beispiele}\label{beispiele}}

\begin{enumerate}
\def\labelenumi{\arabic{enumi}.}
\item
  VR \(K^{m \times n}\) mit Matritzenaddition und Skalarmultiplikation
  \(0 =\) Nullmatrix \(0_{m \times n}\)
\item
  K Körper, \(n \in \mathbb{N}\),
  \(K^n := \left\{\begin{pmatrix}k_1 \\ \vdots \\ k_n\end{pmatrix} \mid k_1, ..., k_n \in K\right\}\)
  \hfill\quad\linebreak  z.B \(\mathbb{R}^2\)
\item
  K Körper, z.B \(\mathbb{R}, \mathbb{C}, GF(2)\)
\item
  Sei \(A \neq 0\), \(\wp(A)\) für \(X,Y \in A\). Sei Nullvektor =
  \(\emptyset\)
\item
  VR von Abbildungen \(f: A \rightarrow K\), Nullvektor
  \(f : A \rightarrow K, a \mapsto 0\)
\end{enumerate}

\hypertarget{bezeichnungen}{%
\ssect{Bezeichnungen}\label{bezeichnungen}}

\begin{itemize}
\tightlist
\item
  \(\mathbb{R}\)-VR: reeller Vektorraum
\item
  \(\mathbb{C}\)-VR: komplexer Vektorraum
\end{itemize}

\hypertarget{bemerkung}{%
\ssect{Bemerkung}\label{bemerkung}}

\begin{itemize}
\item
  \((V, \boxed{+})\): abelsche Gruppe \quad \(\boxed{+}\) 2-stellige
  Operation
\item
  \((V, \boxed{+}, \underbrace{(k \mid k \in K)}_{\mathrlap{  \text{Skalarmultiplikation}  \begin{cases}  |k| \text{ 1-stellige Operation} \\  k \in K, k \text{ fest } V \rightarrow V: v \mapsto k  \end{cases}  }})\)
\item
  (V7) kann nicht weggelassen werden. Ein Gegenbeispiel ist eine
  abelsche Gruppe mit einer Skalarmultiplikation als Operation bei der
  alle Elemente auf \(0\) abgebildet werden.
\item
  (V3) kann weggelassen werden (kann aus den anderen Axiomen hergeleitet
  werden)
\item
  da \(0 \in K\) und \(0 \in V\) wird oft \(0_K\) und \(0_V\)
  geschrieben
\end{itemize}

\hypertarget{rechengesetze-fuxfcr-vektoren}{%
\ssect{Rechengesetze für
Vektoren}\label{rechengesetze-fuxfcr-vektoren}}

\begin{itemize}
\item
  (R1) \(0_K \in K, \> V \in V \implies 0_K \boxed{\cdot} v = 0_V\)
\item
  (R2) \(k \in K, \> 0_V \in V \implies k \boxed{\cdot} 0_V = 0_V\)
\item
  (R3)
  \(k \in K, \> v \in V, \> k \boxed{\cdot} v = 0_V \implies k = 0_K \vee v=0_V\)
\item
  (R4)
  \(k \in K, \> v \in V \implies -k \boxed{\cdot} v = -(k \boxed{\cdot} v)\)
  \hfill\quad\linebreak z.B.
  \((-1) \boxed{\cdot} v = - (1 \boxed{\cdot} v) = -v\)
\end{itemize}

\hypertarget{beweis-r1}{%
\sssect{Beweis (R1)}\label{beweis-r1}}

\begin{flalign*}
0

Substitution: \(\boxed{\quad} := (0_K \boxed{\cdot} v)\)

Wegen (V5) existiert \(-\boxed{\quad}\) mit
\(\boxed{\quad} \boxed{+} -\boxed{\quad} = 0_V\) nach (V3)

\begin{flalign*}
\boxed{\quad} &= \boxed{\quad} \boxed{+} \boxed{\quad}&&\\
\boxed{\quad} \boxed{+} -\boxed{\quad} &= \boxed{\quad} \boxed{+} \boxed{\quad} \boxed{+} -\boxed{\quad}&&\\
0

Rücksubstitution \(\boxed{\quad} = 0_K \boxed{\cdot} v\)

\(\implies 0_K \boxed{\cdot} v = 0_V \hfill \square\)

\hypertarget{bemerkung-1}{%
\ssect{Bemerkung}\label{bemerkung-1}}

Der Nullvektor ist in jedem VR eindeutig bestimmt, denn:

Annahme: \$0\_1 0\_2 seien Nullvektoren mit \(0_1 \neq 0_2\)

\(0_1 \boxed{+} 0_2 = 0_2 \boxed{+} 0_1 = 0_1 = 0_2 \implies\)
Widerspruch \(\implies\) Nullvektor eindeutig bestimmt

\hypertarget{definition}{%
\ssect{Definition}\label{definition}}

Sei V ein K-VR mit \(U \in V\), so heißt U heißt Untervektorraum von V,
wenn gilt:

\begin{enumerate}
\def\labelenumi{\arabic{enumi}.}
\setcounter{enumi}{-1}
\tightlist
\item
  \(0_V \in U\) oder äquivalent \(\{\} \neq U\)\\
\item
  \(U_1, U_2 \in U \implies u_1 \boxed{+} u_2\) für alle
  \(u_1, u_2 \in U\) (Abgeschlossenheit von \(\boxed{+}\))
\item
  \(k \in K, \> u \in U \implies k \boxed{\cdot} u \in U\) für alle
  \(k \in K, u \in U\) (Abgeschlossenheit von \(\boxed{\cdot}\))
\end{enumerate}

Untervektorraum wird im folgenden im UVR abgekürzt.

\hypertarget{bemerkung-2}{%
\ssect{Bemerkung}\label{bemerkung-2}}

\begin{itemize}
\tightlist
\item
  UVR sind ebenfalls VR
\item
  Sei \(V\) ein K-VR, so sind insbesondere folgende VR UVR von V:

  \begin{itemize}
  \tightlist
  \item
    (V, \boxed{+}, \boxed{\cdot}) {[}V ist UVR von sich selbst{]}
  \item
    (\{0\_V\}, \boxed{+}, \boxed{\cdot}) {[}Nullraum{]}
  \end{itemize}
\end{itemize}

\hypertarget{definition-1}{%
\ssect{Definition}\label{definition-1}}

Sei V ein K-VR, \(T \subseteq V\) der kleinste UVR von \(V\) der
sämltiche Elemente von \(T\) enthält heißt Spannraum von \(T\) oder von
\(T\) aufgespannter UVR und wird mit \(Span(T)\) bezeichnet.
\end{document}
