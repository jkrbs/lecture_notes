\documentclass{../tudscript}

\title{Mathe VL 15}
\author{Jakob Krebs}

\begin{document}
\sect{Wahrscheinlichkeitstheorie}
\ssect{Beispiel}
Stochastischer Test eines Algorithmus A auf Korrektheit

Eingabe: Polynom $f(x)$ vom Grad $\leq d$ über einem Körper K.

Ausgabe: Normalform von $f(x)$

\ilmath{f (x) \overset{\rightarrow}{A} g (x)}

Wähle:
\ilmath{r \in\Set{0, 1, \ldots, 100d-1}}
\underline{zufällig aus.}

\begin{itemize}

\item $f(r) \neq g(r) \implies f(x) \neq g(x) \implies$ A nicht korrekt
\item $f(r) = g(r) \implies f(x) = g(x)$ oder $f(x) \neq g(x)$ mit Wahrscheinlichkeit $\leq \frac{1}{100} = \frac{d}{100d}$

\end{itemize}

Denn $f(r) = g(r) \implies$ r ist Nullstelle von $f(x) - g(x)$. Gilt $f(x) \neq g(x)$, denn $f(x) - g(x)$ hat höchstens d Nullstellen über K.

\ssect{Definition: Ereignisfeld}
Sei $\Omega$ eine Menge.

$\sE$ heißt Ereignisfeld, wenn $\sE$ eine Menge von Teilmengen von $\omega$ ist und gilt:

\begin{enumerate}
\item $\Omega \in \sE$
\item $A \in \sE \implies \bar{A} := \Omega \setminus A \in \sE$
\item $A_1, A_2, \ldots \in \sE \implies \bigcup_{i = 1}^{\infty} A_i = A_1 \cup A_2 \cup, \ldots \in \sE$
\end{enumerate}

\ssect{Bemerkung}
Ein Ereignisfeld wird auch E-Algebra genannt.

\begin{itemize}
\item $\sE \subseteq P(\Omega)$
\item Man kann auf Ereignisfeldern rechnen mit $\bar{}, \cup, \cap$
\end{itemize}

\ssect{Bezeichnungen}
\begin{itemize}
\item $A \in \sE$: (zufällige) Ereignis
\item $\Omega \in \sE$: sicheres Eregnis
\item $\bar{\Omega} = \emptyset \in \sE$: unmögliches Ereignis
\end{itemize}

\ssect{Bemerkung}
$A_1, A_2, \ldots$ sind abzählbar viele Ereignisse. Endliche viele oder abzählbar undendlich viele.

endlich: $A_1, A_2, \emptyset, \emptyset, \ldots$

\ssect{Beispiel}
Würfeln: $A_i, \ldots$ der i-te Wurd ist eine 6. $\bar{A_i}$: der i-te Wurf ist keine 6.

\begin{enumerate}
\item $\bigcap_{i = 1}^{4} A_i = A_1 \cap A_2 \cap A_3 \cap A_4$ In den ersten vier Würfen wird jeweils eine 6 gewürfelt.
\item $\bigcup_{i = 1}^4 A_i = A_1 \cup A_2 \cup A_3 \cup A_4$ unter den ersten vier Würfen ist mindestens eine 6.
\item $\bigcap_{n = 1}^{\infty} \bigcup_{i =n}^{\infty} A_i = (A_1 \cup A_2 \cup A_3 \cup \ldots) \cap (A_2 \cup A_3 \cup \ldots) \cap (A_3 \cup \ldots) \cap \ldots$
Für jedes $n_0$ gibt es ein $n \geq n_0$, sodass im n-ten Wurf 6 gewürfelt wird. Die 6 wird \underline{unendlich oft} gewürfelt.
\item $bigcup_{n = 1}^{\infty} \bigcap_{i = n}^{\infty} A_i = (A_1 \cap A_2 \cap \ldots) \cup (A_2 \cap \ldots) \cup \ldots$

Es gibt ein $n_0$ mit $A_{n_0} \cap A_{n_0 + 1} \cap \ldots$ Ab dem $n_0$-ten Wurf wird immer 6 gewürfelt.

Die 6 wird \enquote{fast immer} gewürfelt (es gibt endlich viele Ausnahmen.)

\end{enumerate}

\ssect{Defeinition Wahrscheinlichkeitsmodell}

Sei $\sE$ ein Ereignisfeld über $\Omega$.

Eine Abbildung
\ilmath{p: \sE \rightarrow \bR: A \mapsto p(A)}
heißt \underline{uWahrscheinlichkeitsmaß} und $p(A)$ die \underline{Wahrschinlichkeit} von A, wenn gilt:

\begin{enumerate}
\item $A \in \sE \implies 0 \leq p(A) \leq 1$
\item $p(\Omega) = 1$
\item Sind die Ergebnisse $A_1, A_2, \ldots$ \underline{paarweise unvereinbar} d.h. gilt
\ilmath{\forall i,j mit i \neq j:A_i \cap A_j = \emptyset}
dann gilt:
\ilmath{p(\bigcup_{i = 1}^{\infty} A_i) = \sum_{i =1}^{\infty} p(A_i)}
\end{enumerate}

\ssect{Bemerkung}
$p(\emptyset) = 0$, denn $p(\Omega \cup \emptyset) = \underbrace{p(\Omega)}_{=1} + p(\emptyset) = 1$

Aus $p(A) = 0$ folgt nicht, dass $A = \emptyset$ gilt. A heißt \underline{fast unmögliches Ereignis}

Aus $p(A) = 1$ folgt nicht, dass $A = \Omega$ gilt. A heißt \underline{fast sicheres Ereignis}

\ssect{Satz: Siebformel}
Falls paarweise Unvereinbarkeit nicht erfüllt ist, kann amn bei endlich vielen Ergebnissen die Siebformal anwenden.

Für Ergebnisse $A_1, A_2, \ldots, A_n, n \geq 2$ gilt:

\ilmath{p(\bigcup_{i =1}^{n} A_i) &= \sum_{i = 1}^n p(A_1) - \sum_{i < j} p(A_i \cap A_j) + \sum_{i < j < k} p(A_i \cap A_j \cap A_k) - \ldots + \ldots \\
&= \sum_{k =1}^n (-1)^{k+1} \sum_{i_1 < \ldots < i_k} p(A_{i_1} \cap \ldots \cap A_{i_k})}

\ssect{diskrete Wahrscheinlichkeitsräume $(\Omega, p)$}
$\Omega = \Set{\omega_1, \omega_2, \ldots}$ (Menge von Elementarereignisse)

$p: \Omega  \rightarrow [0,1]$ mit $\sum_{i =1}^\infty p(\omega_i) = 1$

Bezeichnung:

\ilmath{\m{\omega_1 & \omega_2 & \ldots \\ p(\omega_1) & p(\omega_2) & \ldots}}

\ilmath{A= \Set{\omega_{i_1}, \omega_{i_2}, \ldots, \omega_{i_k}} \implies p(A) = \sum_{j = 1}^k p(w_{i_j})}

\ssect{klassicher (kombinatorische) Wahrscheinlichkeitsbegriff}
\begin{itemize}
\item Annahme: Das Ereignisfeld ist ein Laplace-Feld, d..:
\begin{itemize}
\item nur endliche viele Vorsuchsdurchgänge/Ereignisse möglich
\item die Versuchsdurchgänge beeinflussen sich nicht gegenseitig
\item alle Versuchsduchgänge sind gleichmöglich
\item bei jeder Versuchsdurchführung tritt genau ein Ereignis ein.
\end{itemize}
\item Definition: Sei $A \in \sE$ und $\sE$ ein Ereignisfeld.
\ilmath{p(A) := \frac{\text{Anzahl der Versuchsdurchgänge, bei denen A eintritt}}{\text{Gesamtzahl der Versuchsdurchgänge}}}
\end{itemize}

\sssect{Beispiel: Geburtstagsproblem}
siehe Wikipedia

\ssect{Geometrischer Wahrscheinlichkeitsbegriff}
\begin{itemize}
\item Annahme:
\begin{itemize}
\item unendlich viele Vorsuchsdurchgänge/Ereignisse möglich
\item die Versuchsdurchgänge beeinflussen sich nicht gegenseitig
\item alle Versuchsduchgänge sind gleichmöglich
\item bei jeder Versuchsdurchführung tritt genau ein Ereignis ein.
\end{itemize}
\item Definition: Bezeichnet $F(\Omega)$ den Flächeninhalt eines endlichen Gebiets G, das dem sicheren Ereignis $\Omega$ entspricht. und $F(x)$ dem Flächeninhalt eines Teilgebiets von G, das dem Ereignis A entspricht,
dann heißt
\ilmath{p(A) = \frac{F(A)}{F(\Omega)}}
die \underline{geometrische Wahrscheinlichkeit} von A.
\end{itemize}
\end{document}
