\documentclass{../tudscript}
\begin{document}
\newcommand{\ilmath}[1]{\begin{flalign*}#1\end{flalign*}}
\newcommand{\vect}[1]{\begin{pmatrix}#1\end{pmatrix}}
\makeatletter

\hypertarget{beispiele-fuxfcr-vektorruxe4ume-aus-der-informatik}{%
\ssect{Beispiele für Vektorräume aus der
Informatik}\label{beispiele-fuxfcr-vektorruxe4ume-aus-der-informatik}}

\hypertarget{section}{%
\sssect{1.}\label{section}}

UVR der Splinefunktion (H 36)

\hypertarget{section-1}{%
\sssect{2.}\label{section-1}}

\(V = GF(2)^n\), Matrix \(H \in GC(2)^{m \times n}\)
\(\mathscr{L} = \Set{x \in GF(2)^n | H_{m \times n} \cdot X_{n \times 1} = 0_{m \times 1}}\)
(Linearcodes)

UVR von \(GF(2)^n\), z.B.

\(H = \begin{pmatrix}1&1&0\\1&0&1 \end{pmatrix} \> , \> \mathscr{C}\Set{\begin{pmatrix}0\\0\\0\end{pmatrix}, \begin{pmatrix}1\\1\\1\end{pmatrix}}\)

\(\mathscr{C}\) ist ein UVR von \(GF(2)^3\), denn

\begin{enumerate}
\def\labelenumi{\arabic{enumi}.}
\tightlist
\item
  \(\begin{pmatrix}0\\0\\0\end{pmatrix} = 0_{GF(2)^3} \in \mathscr{C}\)
\item
  \(\begin{pmatrix}0\\0\\0\end{pmatrix} + \begin{pmatrix}0\\0\\0\end{pmatrix},  \begin{pmatrix}0\\0\\0\end{pmatrix} + \begin{pmatrix}1\\1\\1\end{pmatrix},  \begin{pmatrix}1\\1\\1\end{pmatrix} + \begin{pmatrix}0\\0\\0\end{pmatrix},  \begin{pmatrix}1\\1\\1\end{pmatrix} + \begin{pmatrix}1\\1\\1\end{pmatrix}  \in \mathscr{C}\)
\item
  \(0 \cdot \begin{pmatrix}0\\0\\0\end{pmatrix},  1 \cdot \begin{pmatrix}0\\0\\0\end{pmatrix},  0 \cdot \begin{pmatrix}1\\1\\1\end{pmatrix},  1 \cdot \begin{pmatrix}1\\1\\1\end{pmatrix} \in \mathscr{C}\)
\end{enumerate}

\hypertarget{beispiel-vr-mathscrr3}{%
\ssect{\texorpdfstring{Beispiel: VR
\(\mathscr{R}^3\)}{Beispiel: VR \textbackslash{}mathscr\{R\}\^{}3}}\label{beispiel-vr-mathscrr3}}

UVR sind unter anderem:

\begin{itemize}
\tightlist
\item
  \(\mathscr{R}^3\)
\item
  \(\Set{\begin{pmatrix}0\\0\\0\end{pmatrix}^\prime} = \{0_{\mathscr{R}^3}\}\)
  Nullraum
\item
  UVRe der Geraden durch\hfill\quad\linebreak 
  \(0_{\mathscr{R}^3}\Set{t\begin{pmatrix}a\\b\\c\end{pmatrix} | t \in \mathscr{R}}, \begin{pmatrix}a\\b\\c\end{pmatrix} \in \mathscr{R}^3\)
\item
  UVR Ebene durch \(0_{\mathscr{R}^3}\)\\
  \(\Set{t_1 \begin{pmatrix}a_1\\b_1\\c_1\end{pmatrix} + t_2 \begin{pmatrix}a_2\\b_2\\c_2\end{pmatrix} | t_1,t_2 \in \mathscr{R}}\)\\
  vorausgesetzt, \(\begin{pmatrix}a_2\\b_1\\c_1\end{pmatrix}\) befindet
  sich nicht auf einer Geraden mit
  \(\begin{pmatrix}a_2\\b_2\\c_2\end{pmatrix}\), und beide sind ungleich
  dem Nullvektor \(0_{\mathbb{R}^3}\)
\end{itemize}

\hypertarget{allgemein}{%
\ssect{Allgemein:}\label{allgemein}}

Sei \(V\) ein K-VR, mit \(T \subseteq V\)

\begin{center}
\begin{tikzpicture}[scale=0.25]
\draw (0,0) circle (4);
\draw (2,0) node [text=black, right] {$V$};
\fill [pattern=north west lines, pattern color=blue](-0.5,0) circle (3);
\draw (-0.5,0) circle (3);
\draw (-1.5,1.5) circle (1)node [text=black] {$T$};
\draw (-0.5, -0.9) node [text=black] {$Span(T)$};
\end{tikzpicture}
\end{center}

\begin{itemize}
\tightlist
\item
  \(Span(T)\): kleinster UVR, der als Teilmenge \(T\) enthält.
\item
  \(Span(V) = V\)
\item
  \(Span(\emptyset) = \{0_V\}\)
\item
  \(Span(\{\underbrace{v_1, v_2, ..., v_n}_{\in V}\}) = \Set{k_1v_1 + k2v_2 + ... + k_nv_n | k_1,k_2,...,k_n \in K}\)\\
  \(\Rightarrow\) Linearkombination der Vektoren \(v_1, v_2, ..., v_n\)
  und den Koeffizienten \(k_1, k_2, ..., k_n\)
\end{itemize}

\hypertarget{beispiel}{%
\ssect{Beispiel}\label{beispiel}}

Ist
\(\Set{\begin{pmatrix}3a + 5b\\2c\\4\end{pmatrix}| a,b,c \in mathbb{R}}\)
UVR von \(\mathbb{R}^3\) ?\\
\(\Rightarrow\) Nein, da kein Nullelement.

Ist
\(\Set{\begin{pmatrix}3a + 5b\\2c\\a+b+c\end{pmatrix}| a,b,c \in mathbb{R}}\)
UVR von \(\mathbb{R}^3\) ?

\(=\Set{a \begin{pmatrix}3\\0\\1\end{pmatrix} + b \begin{pmatrix}5\\0\\1\end{pmatrix} + c \begin{pmatrix}0\\2\\1\end{pmatrix} |  a,b,c \in \mathscr{R}}\)

\(= Span\left(\left\{\begin{pmatrix}3\\0\\1\end{pmatrix},\begin{pmatrix}5\\0\\1\end{pmatrix},\begin{pmatrix}0\\2\\1\end{pmatrix}  \right\}\right)\)
ist UVR von \(\mathbb{R}^3\)

\hypertarget{definition}{%
\ssect{Definition}\label{definition}}

Sei \(V\) ein K-VR, \(T \subseteq V\). Gilt \(Span(T) = V\), so nennt
man \(T\) ein Erzeugungssystem vom UVR

\hypertarget{beispiel-1}{%
\sssect{Beispiel}\label{beispiel-1}}

gesucht: ein minimales Erzeugungssystem für den VR V

\hypertarget{definition-1}{%
\ssect{Definition}\label{definition-1}}

Sei \(V\) ein K-VR, \(v_1, ..., v_n \in V\) \(v_1, ..., v_n\) bzw.
\(\{v_1, ..., v_n\}\) bzw \((v_1, ..., v_n)\) heißt linear unabhängig,
wenn gilt:

Andernfalls heißten die Vektoren linear abhängig.

\hypertarget{beispiel-2}{%
\sssect{Beispiel}\label{beispiel-2}}

\begin{enumerate}
\def\labelenumi{\arabic{enumi}.}
\item
  \(\begin{pmatrix}1\\0\end{pmatrix}, \begin{pmatrix}1\\1\end{pmatrix}, \begin{pmatrix}2\\1\end{pmatrix} \in \mathbb{R}\)
  sind linear abhängig, denn 
\item
  \(\begin{pmatrix}1\\0\end{pmatrix}, \begin{pmatrix}1\\1\end{pmatrix} \in \mathbb{R}\)
  sind linear unabhängig, denn Das LGS hat genau eine Lösung,
  \(\begin{pmatrix}0\\0\end{pmatrix}\)
\end{enumerate}

\hypertarget{allgemein-1}{%
\ssect{Allgemein}\label{allgemein-1}}

Ist
\(\begin{pmatrix}\bigcirc\\ \vdots \\ \bigcirc\end{pmatrix}, \begin{pmatrix}\bigcirc\\ \vdots \\ \bigcirc\end{pmatrix}, \dotsm, \begin{pmatrix}\bigcirc\\ \vdots \\ \bigcirc\end{pmatrix} \in K^n\)
linear abhängig/unabhängig?

\(\Set{\begin{pmatrix}\bigcirc\\ \vdots \\ \bigcirc\end{pmatrix},  \begin{pmatrix}\bigcirc\\ \vdots \\ \bigcirc\end{pmatrix}, \dotsm,  \begin{pmatrix}\bigcirc\\ \vdots \\ \bigcirc\end{pmatrix}|  \begin{pmatrix}0\\ \vdots \\0\end{pmatrix} } \rightsquigarrow\)
homogenes LGS \(\rightsquigarrow\) ZSF

\newcommand\x{\bigcirc}
\newcommand\xp{\hphantom{\bigcirc}}
\newcommand*{\bord}{\multicolumn{1}{c|}{}}
\begin{flalign*}
    \left(
        \begin{array}{cccccc}
            \bigcirc& \bigcirc& \bigcirc& \bigcirc& \dotsm & 
            \multirow{4}{*}{
                $\left.\begin{aligned}\\\\\\\\\end{aligned}\right\}r$ 
            }\\ 
            \cline{1-1}
            \multicolumn{1}{c|}{}& \bigcirc& \bigcirc& \bigcirc& \dotsm \\ 
            \cline{2-2}
                          & \multicolumn{1}{c|}{}& \bigcirc& \bigcirc& \dotsm \\ 
            \cline{3-3}
            \mbox{\huge 0} &        & \multicolumn{1}{c|}{}& \bigcirc& \dotsm \\ \cline{4-5}
            \multicolumn{5}{c}{
                \underbrace{\begin{array}{ccccc}
                \hphantom{\bigcirc}& \hphantom{\bigcirc}& \hphantom{\bigcirc}& \hphantom{\bigcirc}& \hphantom{\bigcirc}
                \end{array}}_n
            }&\\
        \end{array}
    \right)
    \text{\qquad}
    \begin{array}{ll}
    r=n \implies\ &\text{die Vektoren sind linear unabhängig}\\
    r<n \implies\ &\text{die Vektoren sind linear abhängig}\\
    \end{array}
\end{flalign*}

\hypertarget{beispiel-3}{%
\sssect{Beispiel}\label{beispiel-3}}

\begin{itemize}
\tightlist
\item
  \(\emptyset\) ist linear unabhängig
\item
  \(\{0_V\}\) ist linear unabhängig
\end{itemize}

\hypertarget{definition-2}{%
\ssect{Definition}\label{definition-2}}

Sei \(V\) ein K-VR, mit \(B \subseteq V\).\\
B heißt eine Basis von V, wenn gilt:

\begin{enumerate}
\def\labelenumi{\arabic{enumi}.}
\tightlist
\item
  \(Span(B) = V\)
\item
  B ist linear unabhängig
\end{enumerate}

Bemerkung: Eine Basis ist ein linear unabhängiges Erzeugungssystem

\hypertarget{beispiel-4}{%
\sssect{Beispiel}\label{beispiel-4}}

\begin{enumerate}
\def\labelenumi{\arabic{enumi}.}
\item
  \begin{itemize}
  \tightlist
  \item
    \(\begin{pmatrix}1\\0\end{pmatrix}, \begin{pmatrix}1\\1\end{pmatrix}\)
    Basis von \(\mathbb{R}^2\)
  \item
    \(\begin{pmatrix}1\\0\end{pmatrix}, \begin{pmatrix}0\\1\end{pmatrix}\)
    Standardbasis von \(\mathbb{R}^2\)
  \item
    \(\{e_1, ..., e_n\}\) mit
    \(e_i = \begin{pmatrix}0 \\ \vdots \\ 1 \\ 0\end{pmatrix} \begin{array}{l}  \vphantom{0}\\  \vphantom{\vdots}\\  \leftarrow \text{i-te Zeile}\\  \hphantom{0} \end{array} \qquad\)
    Standardbasis von \(K^n\)
  \end{itemize}
\item
  \begin{itemize}
  \tightlist
  \item
    \(1\) ist eine Basis für \(\mathbb{R}\) (Standardbasis)
  \item
    \(1 + i\) ist eine Basis für \(\mathbb{C}\)
  \end{itemize}
\item
  \(\begin{pmatrix}1 & 0 \\ 0 & 0\end{pmatrix}, \begin{pmatrix}0 & 1 \\ 0 & 0\end{pmatrix},  \begin{pmatrix}0 & 0 \\ 1 & 0\end{pmatrix}, \begin{pmatrix}0 & 0 \\ 0 & 1\end{pmatrix}\)
  ist eine Basis von \(\mathbb{R}^{2 \times 2}\)
\item
  Der Nullraum \(\{0_V\}\) hat die Basis \(\emptyset\)
\end{enumerate}

\hypertarget{bemerkung}{%
\sssect{Bemerkung}\label{bemerkung}}

\begin{itemize}
\tightlist
\item
  Eine Basis eines VR ist ein minimales Erzeugungssystem dieses VR
\item
  Jedes minimale Erzeugungssystem eines VR ist eine Basis dieses VR\\
  (der Beweis ist aktuell noch zu schwer)\texttrademark
\end{itemize}

\hypertarget{bemerkung-1}{%
\sssect{Bemerkung}\label{bemerkung-1}}

\begin{itemize}
\tightlist
\item
  Eine Basis eines VR ist eine maximale linear unabhängige Menge dieses
  VR
\item
  Jede maximale linear unabhängige Menge ist eine Basis des VR
\end{itemize}

\hypertarget{satz}{%
\ssect{Satz}\label{satz}}

Sei \(V\) ein K-VR, \(B_1, B_2\) seien Basen.\\
Dann gilt \(|B_2| = |B_2|\)

\hypertarget{definition-3}{%
\ssect{Definition}\label{definition-3}}

Sei \(V\) ein K-VR, \(B_1\) Basis von V.\\
\(dim\>V := |B|\quad\) heißt Dimension von V

\hypertarget{beispiel-5}{%
\sssect{Beispiel}\label{beispiel-5}}

\begin{itemize}
\tightlist
\item
  \(dim\>\mathbb{R}= |\{1\}| = 1\)
\item
  \(dim\>\mathbb{C} = \left|\left\{\begin{pmatrix}1\\0\end{pmatrix}, \begin{pmatrix}0\\i\end{pmatrix}\right\}\right| = 2\)
\item
  \(dim\>K^n = n\)
\item
  \(dim\>\{0_v\} = |\emptyset| = 0\)
\end{itemize}

\hypertarget{definition-4}{%
\ssect{Definition}\label{definition-4}}

Sei \(V\) ein K-VR, \(B=(b_1, b_2, ..., b_n)\) eine angeordnete Basis
von \(V, v \in V\)\\
Dann nennt man den eindeutig bestimmten Vektor
\(\begin{pmatrix}k_1\\\vdots\\k_n\end{pmatrix} \in K^n\) mit
\(k_1b_1 + k_2b_2+...+k_nb_n = v\) den Koordinatenvektor von v,
Bezeichnung: \(v_B\)
\end{document}
