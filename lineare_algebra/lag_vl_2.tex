\documentclass{../tudscript}
\begin{document}
\hypertarget{einfuxfchrung-in-die-mathematik-fuxfcr-informatikerinnen}{%
\sect{Einführung in die Mathematik für
InformatikerInnen}\label{einfuxfchrung-in-die-mathematik-fuxfcr-informatikerinnen}}

\hypertarget{vl}{%
\sect{2. VL}\label{vl}}

\hypertarget{beispiel-der-komplexen-uhr}{%
\ssect{Beispiel der komplexen
Uhr}\label{beispiel-der-komplexen-uhr}}

\begin{itemize}
\tightlist
\item
  kleiner Zeiger: \(Z_{k} = e^{i \phi}\)
\item
  großer Zeiger: \(Z_{g} = e^{i \phi \cdot 12}\)
\end{itemize}

Gesucht sind die Winkel an denen der kleine auf dem großen Zeiger liegt.

\(z_{k} = z_{g} \iff e^{i \phi} = e^{i \phi \cdot 12} \\ 1 = \frac{e^{i 12 \phi}}{e^{i \phi}}\\ = e^{i 11 \phi} \\ \iff e^{0 i} = e^{i 11 \phi} \\ \iff 1 \phi = {0 \pm 2 \pi k \mid k \in \mathbb{N}}\\ \iff \phi \in {\pm \frac{2 \pi}{11}\cdot k \mid k \in \mathbb{N}}\)

Für \(0 \leq \phi < 2 \pi\)

\(L = {0, \frac{2 \pi}{11}\cdot 1, \frac{2 \pi}{11}\cdot 2}, \frac{2 \pi}{11}\cdot 3, \dotsc, \frac{2 \pi}{11}\cdot 10\)

\(k \in {0, 1, \dotsc, 10}\)

\(e^{i \cdot \frac{2\pi}{11} (k \pm 11x)} = e^{i \frac{2 \pi}{11} \cdot k} \cdot e^{i \cdot \frac{2 \pi}{11} \cdot 11 x} = e^{i \frac{2 \pi}{11} \cdot k}\)

\hypertarget{satz-von-moirre}{%
\ssect{Satz von Moirre}\label{satz-von-moirre}}

Sei \(Z = r \cdot e^{i \phi} \in \mathbb{C}\) und \(n \in \mathbb{N}\).
Dann gilt:

\(z^n = r^n \cdot e^{i n \phi}\)

Beweis mittels vollständiger Induktion möglich.

\hypertarget{folgerung-formel-von-moirre}{%
\ssect{Folgerung: Formel von
Moirre}\label{folgerung-formel-von-moirre}}

Sei \(Z = e^{i \phi} \in \mathbb{C}\) und \(n \in \mathbb{N}\). Dann
gilt:

\(z^n = e^{i n \phi}\)

\hypertarget{bemerkung}{%
\ssect{Bemerkung}\label{bemerkung}}

Sei \(Z \in \mathbb{C}\) und \(n \in \mathbb{N}\)

\(| z^n | = r^n = | z |^n\)

Potenzieren in \(\mathbb{C}\) ist Äquvivalent zum Potenzieren in
\(\mathbb{R}\)

\hypertarget{beispiel}{%
\ssect{Beispiel}\label{beispiel}}

\(z = e^{i \frac{\pi}{4}}\), gesucht \({z^n \mid n \in \mathbb{N} }\)

Alle Potenzen \(z^n\) liegen auf dem Einheitskreis (Kreis mit r = 1 und
Mittelpunkt 0).

\hypertarget{wurzelziehen-von-komplexen-zahlen}{%
\ssect{Wurzelziehen von komplexen
Zahlen}\label{wurzelziehen-von-komplexen-zahlen}}

\begin{enumerate}
\def\labelenumi{\arabic{enumi}.}
\tightlist
\item
  \(z^n = 1\)
\item
  \(z^n = z_{0} \in \mathbb{C}\)
\end{enumerate}

\hypertarget{definition-einheitswurzel}{%
\sssect{Definition:
Einheitswurzel}\label{definition-einheitswurzel}}

Die n-te Einheitswurzel ist eine komplexe Zahl z mit \(z^n = 1\), wobei
\(n\in \mathbb{N}, n \geq 1\) gilt.

\hypertarget{beispiele}{%
\sssect{Beispiele}\label{beispiele}}

\begin{enumerate}
\def\labelenumi{\arabic{enumi}.}
\item
\end{enumerate}

\begin{itemize}
\tightlist
\item
  i ist eine 4-te Einheitswurzel, denn \(i^4 = 1\)
\item
  -i ist eine 4-te Einheitswurzel, denn \(-i^4 = 1\)
\item
  1 ist eine 4-te Einheitswurzel, denn \(1^4 = 1\)
\end{itemize}

\begin{enumerate}
\def\labelenumi{\arabic{enumi}.}
\setcounter{enumi}{1}
\tightlist
\item
  1 ist eine n-te Einheitswurzel für \(n \in \mathbb{N}\), denn
  \(1^n = 1\)
\item
  \(e^{i \cdot \frac{2 \pi}{3}}\) ist eine 3-te Einheitswurzel, denn
  \(e^{i \cdot \frac{2 \pi}{3} \cdot 3} = 1\)
\item
  \(e^{i \cdot \frac{2 \pi}{4}}\) ist eine 8-te Einheitswurzel, denn
  \(e^{i \cdot \frac{2 \pi}{4} \cdot 8} = 1\)
\end{enumerate}

\hypertarget{bemerkung-1}{%
\sssect{Bemerkung}\label{bemerkung-1}}

\(z^n = 1 (n \in \mathbb{N}, n \geq 1)\) hat genau n Lösungen.

\(z_k = 1 \cdot e^{i \frac{2 \pi}{n} \cdot k}\)
\((k = 0, 1, \dotsc, n-1)\)

\hypertarget{probe}{%
\sssect{Probe}\label{probe}}

\((e^{i \frac{2 \pi}{n} \cdot k })^n = e^{i 2 \pi k} = 1\)

\hypertarget{luxf6sen-von-gleichungen-in-mathbbc}{%
\ssect{\texorpdfstring{Lösen von Gleichungen in
\(\mathbb{C}\)}{Lösen von Gleichungen in \textbackslash{}mathbb\{C\}}}\label{luxf6sen-von-gleichungen-in-mathbbc}}

\hypertarget{gesucht}{%
\sssect{Gesucht}\label{gesucht}}

Alle Lösungen von \(z^n = z_0 = e^{i \phi_{0}} \in \mathbb{C}\)

(\(z_0 \in \mathbb{C}\), beliebig, fest)

\hypertarget{ansatz}{%
\sssect{Ansatz}\label{ansatz}}

\(z = r \cdot e^{i \phi}\) (ges. r, \(\phi\))

\hypertarget{einsetzen}{%
\sssect{Einsetzen}\label{einsetzen}}

\((r \cdot e^{i \phi})^n = r \cdot e^{i \phi_0} \\ \iff r^n e^{i n \phi} = r_0 e^{i \phi_0} \\ \iff r^n = r_0 \wedge e^{i n \phi} = e^{i \phi_0} \\ \iff r = \sqrt[n]{r_0} \wedge n \phi = \phi_0 \pm 2 \pi k \mid k \in \mathbb{N}\)

\(\phi \in {\frac{\phi_0 \pm 2 \pi k}{n} \mid k \in \mathbb{N}}\)

\hypertarget{luxf6sungen-von-zn-z_0}{%
\sssect{\texorpdfstring{Lösungen von
\(z^n = z_0\)}{Lösungen von z\^{}n = z\_0}}\label{luxf6sungen-von-zn-z_0}}

\(z_k = \sqrt[n]{r_o} \cdot e^{i \cdot (\frac{\phi_0}{n} + \frac{2 \pi}{n}k)}\)

\(k = {0,1, \dotsc, n-1}\)

\hypertarget{bemerkung-2}{%
\sssect{Bemerkung}\label{bemerkung-2}}

Lösungen von \(z^n = z_0\)

\(z_k = \sqrt[n]{r_0} \cdot e^{i \cdot \frac{\phi_0}{n}} \cdot e^{i \cdot \frac{2 \pi}{n} k}\)

\(k = {0,1, \dotsc, n-1}\)

\(e^{i \cdot \frac{2 \pi}{n} k}\) ist die n-te Einheitswurzel

\hypertarget{beispiel-1}{%
\sssect{Beispiel}\label{beispiel-1}}

\(z^2 = -16 i = 16 \cdot (-i) = 16 \cdot e^{i \frac{3}{2} \pi}\)

\hypertarget{weg}{%
\paragraph{1. Weg}\label{weg}}

\(z_k = \sqrt{16} e^{i (\frac{3}{4} \pi + \frac{2 \pi}{2}k)}\)

\(k = {0,1}\)

\(z_0 = 4 e^{i \frac{3 \pi}{4}} = 4 (- \frac{1}{2} \sqrt{2} + \frac{1}{2} \sqrt{2} = -2 \sqrt{2} - 2 \sqrt{2} i)\)

\(z_1 = 4 e^{i \frac{7 \pi}{4}} = 2 \sqrt{2} - 2 \sqrt{2} i)\)

\hypertarget{weg-1}{%
\paragraph{2. Weg}\label{weg-1}}

2-te Einheitswurzel \{1, -1\}

Eine spezielle Lösung: \(4 \cdot e^{i \frac{3 \pi}{4}}\)

\(L = \pm 4 \cdot e^{i \frac{3 \pi}{4}} = \pm 2 \sqrt{2} - 2 \sqrt{2} i\)

\hypertarget{allgemein}{%
\sssect{Allgemein}\label{allgemein}}

\(az^2 + bz + c = 0\) mit \(a, b, c \in \mathbb{C}, a \neq 0\)

\(\iff z^2 + \frac{b}{a} z + \frac{c}{a} = 0\)

\((p = \frac{b}{a}, q = \frac{c}{a})\)

\(\iff z^2 + pz + q = 0\)

\(\iff (z + \frac{p}{2})^2 - \frac{p^2}{4} + q = 0\)

Substituation mit \(w := z - \frac{p^2}{4}\)

\((z +\frac{p}{2})^2 = \frac{p^2}{4} - q \iff w^2 = \frac{p^2}{4} - q\)

daraus folgt:

\(z_1 = - \frac{p}{2} + w_0\) \(z_2 = \frac{p}{2} + w_1\)
\end{document}
