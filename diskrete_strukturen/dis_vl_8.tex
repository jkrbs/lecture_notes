\documentclass{../../meta/tudscript}
\begin{document}


\paragraph{2)}

``\(\implies\)'': Klar. ``\(\impliedby\)'' Nach 1) folgt sowohl
\(m \leq n\) als auch \(n \leq m\) daher \(n = m\)A.


\ssect{Definition}

Eine Menge \(X\) heißt endlich, falls es \(n \in \mathbb{N}\) und eine
Bijektion \(f:[n] \rightarrow X\) gibt. In diesem Fall ist \(n\)
eindeutig bestimmt (nach FOlgerung 6.3, 2) und wir nennen \(|X|\) := n
die Mächtigkeit (Kardinalität) von \(X\).


\ssect{Bemerkung}

Seien X,Y,Z Mengen, so folgt

\begin{enumerate}
\def\labelenumi{\arabic{enumi})}
\item
  Sind \(f: X \rightarrow Y\) und \(g: Y \rightarrow Z\) injektiv
  (surjektiv, bijektiv) dann ist auch \(g \circ f: X \rightarrow Z\)
  injektiv (surjektiv, bijektiv).
\item
  Ist \(f: X \rightarrow Y\) bijektiv, dann ist auch
  \(f^{-1}: Y \rightarrow X\) mit
  \(f^{-1} = \{(y,x) \in Y \times X \mid (x,y) \in f\}\) eine Bijektion.
\end{enumerate}


\ssect{Satz}

Sei X endliche Menge. EIne Abbildung \(f: X \rightarrow X\) ist injektiv
genau dann, wenn \(f\) surjektiv ist.


\sssect{Beweis}

``\(\implies\)'': Sei \(f: X \rightarrow X\) injektiv. Sei \(n \in sN\)
und \(g: [n] \rightarrow X\) bijektiv, dann ist
\(h:= g^{-1} \circ f \circ g: [n]\) injektiv nach Bemerkung 6.5. Also
\(h\) surjektiv nach Lemma 6.2. Somit \(f \circ h \circ g^{-1}\)
surjektiv nach Bemerkung 6.5. \(X \xrightarrow{f} X\)


\sssect{Beweis:}

eine Surjektive Abbildung ist immer Injektiv bei einer endlichen Meneg
ein sich selbst.


\paragraph{Vorbemerkung}

Sind X,Y endliche Mengen und ist \(f: X \rightarrow Y\) surjektiv, dann
ist diese auch rechtsinvertierbar, d.h es exyistiert eine Abbildung
\(g: Y \rightarrow X\) mit \(\exists g = id_y\). (Beweis durch
vollständoige Induktion über \textbar{}X\textbar{})

In unserem Fall brachten wir surjektivität und rechtsinvertierbarkeit
als äquivalent, genau, wie Injektivität und linksinvertierbarkeit.


\paragraph{\texorpdfstring{Beweis
\(\Rightarrow\)}}

Sei \(f: X \rightarrow X\) surjektiv. Dann gibt es (Vorbemerkung)
\(g: X \rightarrow X\) mit \(f \circ g =id_x\)

Nach Bemerkung 4.11,1 ist g injektiv, also Surjektiv nach
\(\Rightarrow\).

Damit folgt Inkektivität von f: Sind\(x,y \in X\) mit \(f(x) = f(y)\),
dann gibt es (wegen Surjektivität von g) \(a,b \in X\) mit \(g(a) = x\)
und \(g(b) = y\) und es folgt

\begin{flalign*}x = g(a) = g(f((g(a))) = g(f(x)) = g(f(y)) = g(f(g(b))) = g(b) = y\end{flalign*}
\begin{flalign*}q.e.d\end{flalign*}


\ssect{Proposition:}

Seien X,Y endliche Mengen.

\begin{enumerate}
\def\labelenumi{\arabic{enumi}.}

\item
  \(|X| \leq |y| \iff \exists f: x \rightarrow Y\) injektiv
\item
  \(|x| = |Y| \iff \exists f: X \rightarrow Y\) bijektiv
\end{enumerate}


\sssect{Bemerkung}

Es folgt das \emph{Schubfachprinzip} (engl. pigionhole principal). Sind

X,Y endliche Mengen mit \(|X| > |Y|\), dann ist jede Abbildung
\(f: X \rightarrow Y\) nicht-injektiv.


\ssect{Proposition}

Seien X,Y endliche Mengen. Dann gelten folgende Aussagen

\begin{enumerate}
\def\labelenumi{\arabic{enumi}.}

\item
  \(X \cap Y = \emptyset \iff |X \cup Y| = |X| + |Y|\)
\item
  Ist \(\mathscr{C}= \{A_1, ..., A_n\}\) eine Partition von X mit
		\(| \mathscr{C} | = n\), dann ist \(|X| = \sum_{i=1}^{n} |A_i|\)
\item
  \(|X \times Y| = |X| \cdot |Y|\)
\item
  \(|Y^X| = |Y|^{|X|}\)
\end{enumerate}


\sssect{Beweis 1}

Übung


\sssect{Beweis 2}

Folgt aus 1. mittels vollständiger Induktion über n. (1. ist Spezialfall
von 2. Betrachte die Partition \(\mathscr{C}= \{X,Y\}\) der Menge
\(X \cup Y\).)


\sssect{Beweis 3}

für jedes \(x \in X\) ist die Abbildung

bijektiv. (Lemma 2.8) Also \(|\{x\} \times Y| = |Y|\). (nach Proposition
6.7.2)

Auch:

ist Partition von \(X \times Y \text{mit} |\mathscr{C}| = |X|\)

(Lemma 2.8, Propostition 6.7.2), Hinweis
\(x \rightarrow \mathscr{C}, x \mapsto \{x\} \times Y\) bijektiv

Ist nun \(n := |X|\) und \textbar{}X = \{x\_1, \ldots{}, x\_n\},
dann folgt mit 2.

\(|X \times Y| = \sum_{i=1}^{n} |\{ x_i\} \times Y| = n \cdot |Y| = |X| \cdot |Y|\)


\sssect{Beweis 4}

Vollständige Induktion über \textbar{}X\textbar{}.

IA: \(|X| = 0\), d.h \(X = \emptyset\). Dann ist die Aussage wahr, denn
\(|Y^{\emptyset} | = 1 = |Y|^{0}\)

IS: Angenommen \(n \in \mathbb{N}\) und für alle endlichen Mengen X,Y
mit \(|X| = n\) gilt:

\begin{flalign*}IV: | Y^{X} | = | Y |^{n}\end{flalign*}

Sei nun \(\tilde{X}\) endliche Menge mit \(| \tilde{X} | = n + 1\).
Wähle \(x_0 \in \tilde{X}\).

Definiere \(X = \tilde{X} \\ \{x_0\}\). Dann \textbar{}X\textbar{} = n.

Für jedes \(y \in Y\) erhalten wir eine Bijektion:

\(\varphi_{y} : Y^X \rightarrow A_y, f \mapsto \varphi_{y} (f)\)

mit \(A_y = \{g \in Y^X | y(x_0) = y\}\) und

\begin{equation}
  \varphi_g(f) :=
  \begin{cases}
	  f(x) \text{falls} x \in X
	  y \text{falls} x = x_0
  \end{cases}
\end{equation}

und daher ist \(|A_y| = |Y^X| = |Y|^ni\).

wieder: \(\mathscr{C}:= \{ A_y | y \in Y \}\) Partition von
\(Y^{\tilde{X}}\) mit
\(|\mathscr{C}| = |Y| ( Abb. Y \rightarrow \mathscr{C}, y \mapsto A_y\)
ist bijektiv nach Proposition 6.7.2.

Ist nun m = \textbar{}Y\textbar{} und Y = \{y\_1, \ldots{}, y\_m\},
dann folgt mit 2.

\(|Y^{\tilde{X}}| = \sum_{i = 1}^{n} A_{y} = m \cdot |Y|^n = |Y| \cdot |Y|^n= |Y|^{n + 1} = |Y|^{|\tilde{X} |}\)

\begin{flalign*}q.e.d\end{flalign*}


\ssect{Folgerung}

Ist X endliche Menge., dann gilt: \( | \mathscr(P) (X) | = 2 \)


\sssect{Beweis}

\begin{flalign*}|\wp(X)| = |\{0,1\}^X = 2^{|X|}\end{flalign*}
\begin{flalign*}&&q.e.d&\end{flalign*}


\sect{§7 Algebraische Strukturen}


\ssect{Definition: Halbgruppe}

Eine Halbgruppe ist ein Paar (M, *) bestehend aus einer Menge M, sowie
einer Abbildung
\( * : M \times M \rightarrow M, (x,y) \mapsto x*y ,  \)
Sodass gilt:
\( \forall x,y,z \in M : x*(y*z) + (x*y)*z \) (Assoziativität)

Eine Halbgruppe (M, *) heißt \emph{Monoit}, falls (Existens eines
neutralen Elements)


\ssect{Bemerkung:}

Sei (M, *) Monoid. Dann besitzt (M, *) genau ein neutrales Element:

Sind \(e, \tilde{e} \in M\) mit

\begin{flalign*}\forall \in M: x \* e =x \wedge x \* e^{\tilde{i}} = \tilde{e} \* x = x\end{flalign*}

dann folgt \(e = \tilde{e} \* e = \tilde{e}\).


\ssect{Definition: Gruppen}

Ein Monoid (M, *) mit neutralem Element \(e \in M\) heißt \emph{Gruppe},
falls

\begin{flalign*}\forall x \in M: \exists y \in M: x \* y = y \* x = e\end{flalign*}

Existens \emph{inverser Elemente}.


\ssect{Bemerkung}

Sei (M, *) Gruppe, \(x \in M\). Dann besitzt (M, *) genau ein zu x
inverses Element. Sind \(y, \tilde{y} \in M\) mit
\(x \* y = y \* x = e\), dann folgt
\(y = y \* e = y \* (x \* \tilde{y}) = (y \* x) \* \tilde{y} = e \* \tilde{y} = \tilde{y}\).

Das zu x inverse Elementfindet wird häufig \(x^{-1}\) bezeichnet.


\end{document}
