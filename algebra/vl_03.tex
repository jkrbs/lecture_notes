\sect{Faktorhalbgruppen und Faktorgruppen}
\ssect{Definition: Kongruenzrelation}
Sei $(H, \circ)$ eine Halbgruppe und R eine Aequivalenzrelation auf H. R
heisst \underline{Kongruenzrelation} in $(H, \circ)$, wenn gilt:
\ilmath{\forall a, b, c, d \in H: (a,b) \in R \land (c, d) \in R \implies (a \circ c, b \circ d) \in R}
\sssect{Beispiel}
\ilmath{(\bN,+), (a, b) \in R :\iff a \mod n = b \mod n}
\ilmath{a\equiv b (\mod n), c \equiv d (\mod n) \iff a + c = b + d \mod n}
R ist eine Aequivalenzrelation in $\bN$
\sssect{Beispiel}
Halbgruppe $(\bZ_4, \cdot), R_1 = \Set{(0,0), (1,1), (2,2), (2,3), (3,2), (3,3)}$
\begin{center}
  \begin{tabular}{lllll}
    $\cdot$ & 0 & 1 & 2 & 3 \\
    0     & 0 & 0 & 0 & 0 \\
    1     & 0 & 1 & 2 & 3 \\
    2     & 0 & 2 & 0 & 2 \\
    3     & 0 & 3 & 2 & 1
  \end{tabular}
   \end{center}
  $R_1$ ist keine Kongruenzrelation, denn:
  \ilmath{(2,2) \in R \land (2,3) \in R, \text{ aber } (2\cdot2, 2\cdot3) = (0,2) \notin R}
  $(0,2)$ ist nicht in der selben Aequivalenzklassen

  $R_2 = \Set{(0,0), (2,2), (1,1), (1,3), (3,1), (3,3)}$
  ist Kongruenzrelation.
  \ssect{Bemerkung}
  Sei $(H, \circ)$ eine Halbgruppe, R eine Aequivalenzrelation in H, $a \in H$
  \ilmath{[a]_R := \Set{b \in H \mid (a,b) \in R}}
  bezeichnet die Aequivalenzklasse.

  Die Menge
  \ilmath{H/R := \Set{[a]_R \mid a \in H}}
  aller Aequivalenzklassen heisst \underline{Faktormenge}
  \sssect{Beispiel}
  \ilmath{\bZ_4 / R_1 = \Set{\Set{0},\Set{1},\Set{2,3}}\\
    \bZ_4 / R_2 = \Set{\Set{0},\Set{2},\Set{1,3}}}
  \sssect{Beispiel}
  $\bZ_4,\cdot$, Kongruenzrelation $R_2$ mit den Aequivalenzklassen
  $\Set{0},\Set{2},\Set{1,3}$
  % todo tables
\underline{Aber:}\\
$ \bZ_4, \cdot$ Aequivalenzrelation $R_1$ mit den
Aequivalenzklassen  $\Set{0},\Set{1},\Set{2,3}$ keine Kongruenzrelation.
\ilmath{\Set{2,3}\circ_{R_1}\Set{2,3} = ?}
ist nicht eindeutig bestimmt, also ist $\circ_{R_1}$ keine Operation.

\ssect{Bemerkung}
Ist $H,\circ$ eine Halbgruppe und R eine Kongruenzrelation, dann ist durch
\ilmath{\forall [a]_R, [b]_R \in H/R | [a]_R \circ_R [b]_R := [a \circ b]_R} eine
Operation $\circ_R$ auf der Faktormenge H/R definiert.
D.h. fuer alle $[a]_R, [b]_R \in H/R$ gilt: \\
\begin{enumerate}
  \item$[a]_R \circ_R [b]_R$ existiert
  \item$[a]_R \circ_R [b]_R$ ist eindeutig bestimmt
  \item$[a]_R \circ_R [b]_R \in H/R$
\end{enumerate}
\sssect{Beweis von 2.}
\ilmath{[a]_R = [a']_R \text{ und } [b]_R = [b']_R \\
\implies (a,a'),(b,b') \in R\\
  \implies (a \circ b, a' \circ b') \in R\\
\implies [a \circ b]_R = [a' \circ b']_R}

\ssect{Bemerkung}
Ist $(H, \circ)$ eine Halbgruppe und R eine Kongruenzrelation, dann ist $[a]_R
\circ_R [b]_R $ die eindeutig bestimmte Aequivalenzklasse, die $a \circ b$
enthaelt (naemlich $[a \circ b]_R$)\\
Es gilt also: \ilmath{[a]_R = [a']_R, b]_R = [b']_R \implies [a \circ b]_R = [a'
\circ b']_R}
Diese Eigenschaft nennt man \underline{Repraesentantenunabhaengigkeit}.\\
Ist die Repraesentantenunabhaengigkeit erfuellt, so ist $[a]_R \circ_R [b]_R$
eindeutig bestimmt.

\ssect{Satz}
Sei $(H,\circ)$ eine Halbgruppe und R eine Kongruenzrelation. Dann ist
$(H/R,\circ_R)$ eine Halbgruppe.
\sssect{Beweis}
$H/R = {[a]_R | a \in H \neq \emptyset}$\\
$\circ_R$ ist eine Operation in H/R.\\
$\circ_R$ ist assoziativ, denn:
\ilmath{\forall a,b,c \in H &| [a] \circ_R ([b] \circ_R [c])\\
  &= [a] \circ_R [b \circ c]\\
  &= [a \circ (b \circ c)]\\
  &= [(a \circ b) \circ c]\\
  &= [a \circ b] \circ_R [c]\\
  &= ([a] \circ_R [b]) \circ_R [c]
}
\ssect{Definition}
Sei $(h, \circ)$ eine Halbgruppe und R eine Kongruenzrelation. $(H/R,\circ_R)$
heisst Faktorhalbgruppe von $(H, \circ)$ nach R
\sssect{Beispiel}
$(\bZ_4,\cdot)$ Halbgruppe\\ 
$R = \{(a,b \in \bZ_4 \times \bZ_4 | a \cong b \mod 2)\}$ Kongruenzrelation\\
$\bZ_4/R = \Set{\Set{0,2},\Set {1,3}}$
Faktorhalbgruppe:
%todo table
\ssect{Definition}
$(H_1,\circ_1), (H_2,\circ_2)$ seien Halbgruppen und  $h: H_1 \rightarrow H_2$
eine Abbildung.\\
h heisst Homomorphismus, wenn gilt:
\ilmath{\forall a,b \in H_1 | h(a \circ b) = h(a) \circ_2 h(b)}
\sssect{Beispiel}
\begin{enumerate}
	\item $h: \bZ_4 \rightarrow \bZ_2 | x \mapsto x\mod 2$ ist ein Homomorphismus
   von $(\bZ_4,\circ)$ auf $(\bZ_2,\circ)$
   \item $h: R^{n \times n} \rightarrow \bR | A \mapsto det(A)$ ist ein
     Homomorphismus von $(\bR^{n \times n, \cdot})$ auf $(\bR,\cdot)$
   \end{enumerate}
   \ssect{Bemerkung}
   Sei $h: H_1 \rightarrow H_2$ ein Homomorphismus der Halbgruppe
   $(H_1,\circ_1)$ in die Halbgruppe $(H_2,\circ_2)$. Dann gilt:
   U Untergruppe von $(H_2, \circ_2)$
   \ssect{Bemerkung}
   Homomorphismen sind strukturvertraegliche Abbildungen.
   \ssect{Spezielle Homomorphismen}
   %to-do table spezielle Homomorphismen
