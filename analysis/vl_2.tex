\documentclass{../tudscript}
\author{Jakob Krebs}
\title{Mathe VL 2}

\begin{document}
    \url{https://geogebra.org/m/HPjMS7c7} 
    \ssect{Satz: eindeutigkeit des Grenzwerts}
    $(x_n)$ konvergent $\implies$ Der Grenzwert ist eindeutig bestimmt.
    \sssect{Beweis}
    Sei a Grenzwert der Folge $(x_n)$, b Grenzwert von $(x_n)$
    D.h. Sei $\epsilon > 0$, beliebig, fest.
    \ilmath{&\exists N_a: \forall n \geq N_a : |x_n - a| < \epsilon \\
            &\exists N_b: \forall n \geq N_b : |x_n - ab| < \epsilon}

    Sei $max (\Set{N_a, N_b}) =: N$. Dann gilt: 
    \ilmath{n \geq N \implies |x_n - a| < \epsilon \land |x_n - b| < \epsilon \\
            \implies |x_n - a| + |x_n - b| < 2 \epsilon}
    \\

    Annahme: $A \neq b$, d.h. $|a-b| \neq 0$
    
    \ilmath{|a-b| = |a + 0 + b| = |(a - x_n) + (x_n -b)| \\
    \underbrace{\leq}_{Dreiecksungleichung} |x_n - a| + |x_n -b| < 2 \epsilon}
    \ilmath{\boxed{|a-b| < 2 \epsilon}}

    Wähle z.B. $\epsilon = \frac{|a - b|}{3}$, dann gilt:
    \ilmath{|a-b| < 2 \frac{|a-b|}{3} \implies 1 < \frac{2}{3}}
    Widerspruch, also ist die Annahme falsch:
    \ilmath{\implies a = b}

    \ssect{Beispiele: Grenzwerte von Folgen}
    \sssect{harmonische Folge}
    \ilmath{(x_n): x_n = \frac{1}{n}, GW = 0}
    \paragraph{Beweis}
    Sei $\epsilon < 0$, bel, fest.
    gesucht: N mit 
    \ilmath{n \geq N \implies |x_n - a| = |\frac{1}{n} - 0| = \frac{1}{n} < \epsilon}
    Wähle
    \ilmath{N := \left\lceil\frac{1}{\epsilon}\right\rceil + 1}
    
    \ssect{Schreibweise: Limes}
    
    \ilmath{\lim_{n \to \infty} x_n = a}
    bzw.
    \ilmath{x_n \overset{n \to \infty}{\longrightarrow} a}

    \ssect{Definition: Nullfolge}
    $(x_n)$ heißt Nullfolge, wenn gilt:
    \ilmath{\lim_{n \to \infty} x_n = 0}
    
    \sssect{Bemerkung: schwierigkeit der Berechnung der GW von Folgen}
    Es ist leichter die Konvergenz einer Folge zu beweisen, als den Grenzwert 
    zu berechnen.
    \paragraph{einfaches Beispiel}
    \ilmath{x_n = \frac{1}{3} + (\frac{11-n}{9+n})^n}
    
    \ilmath{\lim_{n \to \infty} (x_n + y_n) = (\lim_{n \to \infty} x_n) \underset{\div}{\overset{\cdot}{\pm}} (lim_{n \to \infty} y_n)}
    Voraussetzung: beide Grenzwerte existieren!
    \ilmath{GW &= \lim_{n \to \infty} (\frac{1}{3} \oplus (\frac{11-n}{9+n})^9)\\
     &= \lim_{n \to \infty} \frac{1}{3} + \lim_{n \to \infty} (\frac{11-n}{9+n})^n \\
     &= \frac{1}{3} + (\lim_{n \to \infty} \frac{11-n}{9+n})^n \\
     &= \frac{1}{3} + \lim_{n \to \infty} (\frac{n \cdot (\frac{11}{n} - 1)}{n \cdot (\frac{9}{n} + 1)})^n \\
     &= \frac{1}{3} + (\frac{\lim_{n \to \infty} (\frac{11}{n} -1)}{\lim_{n \to \infty} (\frac{9}{n} +1})^9 \\
     &= \frac{1}{3} + (\frac{11 \cdot 0 - 1}{9 \cdot 0 + 1})^9 \\
     &= - \frac{2}{3}}

    \ssect{Definition: uneigentlicher Grenzwert}
    Eine Folge $(x_n)$ hat den uneigntlichen Grenzwert $+ \infty$, wenn gilt:
    \ilmath{\forall r \in \bR \exists N \in \bN \forall n \geq N: x_n > r}
    $\infty$ ist kein Grenzwert!
    Die Grenzwertsätze gelten nicht für uneigentliche Grenzwerte!
     
    \sssect{Schreibweise}
    \ilmath{\lim_{n \to \infty} x_n = \infty }

    \sssect{Beispiel: geometrische Folge}
    \ilmath{(x_n): x_n = q^n, q \in \bR, q \text{ fest}}
    
    \ilmath{\lim_{n \to \infty} q^n = \begin{cases} 0, |q| < 1 \\ 
                                                    1, q = 1\\
                                                    \infty, q > 1 \\
                                                    \text{ex. nicht},q \leq -1 \\
    \end{cases}}
    
    \ssect{Konevergenzkriterien}
    (zum Beweis eines Grenzwerts, nicht zum Berechnen eines Grenzwerts)
    \begin{enumerate}
        \item $(x_n)$ konvergent $\implies (x_n)$ beschränkt, bzw \\
              $(x_n)$ nicht beschränkt $\implies (x_n)$ nicht konvergent
        \item Monotoniekriterium:\\ 
        $(x_n)$ beschränkt und Monoton $\implies (x_n)$ konvergent
    \end{enumerate}
\end{document}
