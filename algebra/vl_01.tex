\sect{Organisatorisches}
\ssect{Inhalt der Vorlesung}
\begin{itemize}
    \item Halbgruppen und Gruppen
    \item Ringe, Ingritaetsringe, Koerper
    \item Faktorstrukturen, Chinesischer Restsatz und Anwendungen
    \item Modularpolynome, BCH Kode
    \item Konstruktion aller endlichen koerper
    \item Wahtsceinlichkeitstheorie
    \item Polynomauswertung
    \item Splineinterpolation
    \item Diskrete Fourierinformationen (DFT, FFT)
\end{itemize}
\sect{Halbgruppen und Gruppen}
\ssect{Definition: Halbgruppe}
Sei $H \neq \emptyset$ eine Menge und $\circ$ eine assoziative \underline{Operation auf H},
d.h.:
\ilmath{\forall a, b,c \in H: a \circ (b \circ c) = (a \circ b) \circ c}
Dann nennt man $(H; \circ)$ eine \underline{Halbgruppe}
(kurz: Halbgruppe H)

$\mid H \mid$ heisst \underline{Ordnung} der Halbgruppe H.

Eine Halbgruppe $(H; \circ)$ heisst \underline{kommutativ}, wenn gilt:
\ilmath{\forall a, b, \in H: a \circ b = b \circ a}

\ssect{Bemerkung: Operation auf Halbgruppen}
$\circ$ ist eine Operation auf H $\iff \circ$ ist eine Abbildung von$H \times H$ in H
$\iff$
\ilmath{\forall a,b \in H: &a \circ b \text{ ex.} \\
                           &a \circ b \text{ eindeutig bestimmt} \\
                           &a \circ b \in H}
\ssect{Beispiel}
\begin{itemize}
    \item $(\bN \setminus \Set{0}, +)$ ist eine kommutative Halbgruppe
    \item $(\bZ, -)$ ist keine Halbgruppe, denn
        \ilmath{0 - (8-5) + (0 - 9) = 15}
    \item $(\bR^{n \times n},\underbrace{\cdot}_{Matrixmultiplikation})$ Halbgruppe (nicht kommutativ)
    \item $(T_A, \circ)$ Halbgruppe (nicht kommutativ)\\
    (Transformationshalbgruppe)
\end{itemize}

\ssect{Definition: neutrales Element}
Sei $(H, \circ)$ eine Halbgruppe und $e \in H$. e heisst \underline{neutrales Element},
wenn gilt:
\ilmath{\forall a \in H: e \circ a = a \circ e = a}
\sssect{Bemerkung}
Enthaellt eine Halbgruppe ein neutrales Element, dann ist es eindeutig bestimmt, denn
\ilmath{e_1 \circ e_2 = e_2, e_1 \circ e_2 = e_1 \implies e_1 = e_2}
Halbgruppe entgaelt hoechstens ein neutrales Element.

\ssect{Definition: Monoid}
Sei $(H, \circ)$ eine Halbgruppe mir einem neutralem Element, dann ist $(H, \circ)$ ein Monoid
\sssect{Beispiel}
\begin{itemize}
\item $(\bN, +), e = 0$
\item $(\bZ, \cdot) , e = 1$
\item Freies Monoid ueber dem Alphabet $\Sigma$ (Worthalbgruppe)
\ilmath{(\underbrace{\Sigma^\star}_{\text{Menge aller Woerter ueber ueber $\Sigma$}}), \underbrace{\circ}_{\text{Konkatination}}, e = \epsilon}
\end{itemize}

\ssect{Definition: Unterhalbgruppe}
Sei $(H, \circ)$ eine Halbgruppe, $\emptyset \neq U \subseteq H$.
U iheisst Unterhalbgruppe von H, wenn gilt:
\ilmath{\forall a, b \in U \implies a \circ b \in U}
\sssect{Bemerkung}
Unterhalbgruppen sind Halbgruppen (Uebung)
\sssect{Beispiel}
$(2\bZ, \cdot)$ ist eine Unterhalbgruppe von $(\bZ, \cdot)$

\ssect{Definition: Inverse Elemente}
Sei $(H, \circ)$ ein Monoid mit dem neutralem Element e.
$a \in H$ heisst \underline{invertierbar}, wenn es $b \in H$ mit
\ilmath{a \circ b = b \circ a = e}
existiert.
\sssect{Bemerkung}
Sei $a \circ b_1 = b_1 \circ a = e$ und $a \circ b_2 = b_2 \circ a =e$. Dann gilt
$b_1 = b_1 \circ e = b_1 \circ (a \circ b_2) = (b_1 \circ a) \circ b_2 = e \circ b_2 = b_2$
\sssect{Bemerkung}
Ist $a \in H$ invertierbar, dann nennt man $b \in H$ mit 
\ilmath{b \circ a = a \circ b = e}
\underline{das Inverse} von $a$.
\sssect{Schreibweise}
\ilmath{b = a^{-1}}

\ssect{Definition}
Sei $(H, \circ)$ ein Monoid mit dem neutralen Element e.

\ilmath{H^* = \Set{a \in H \mid a^{-1} \in H}}
nennt man die Menge der invertierbaren Elemente.

\sssect{Bemerkung}
\begin{itemize}

\item $(a^{-1})^{-1} = a$
\item $e^{-1} = e$
\item $(a \circ b)^{-1} = b^{-1} \circ a^{-1}$

\end{itemize}

\sssect{Satz}
Sei $(H, \circ)$ ein Monoid, Dann gilt:
$H^*$ ist eine Unterhalbgruppe von $(H, \circ)$


\ssect{Satz: Gruppe}

Sei $(H, \circ)$ ein Monoid. $(H, \circ)$ ist eine Gruppe $\iff H = h^*$

\sssect{Beispiel}
\begin{itemize}
\item $(\bZ, +), (\bR \setminus \Set{0}, \cdot)$
\item $(\bZ_n, +)$
\item $(\bZ_p \setminus \Set{0}, \cdot) (p prim)$
\end{itemize}
sind abelsche Gruppen\\

Gruppoid $\overset{\circ\; assoz.}{\rightarrow}$ Halbgruppe $\overset{e}{\rightarrow}$ Monoid $\overset{a^{-1}}{\rightarrow}$ Gruppe $\overset{\circ\; komm.}{\rightarrow}$ abelsche Gruppe

\ssect{Satz: Kuerzungsregeln}
Sei $(G, \circ)$ eine Gruppe. Dann gelten die \underline{Kuerzungsregeln}
\begin{enumerate}
\item $\forall a, x_1, x_2 \in G: a \circ x_1 = a \circ x_2 \implies x_1 = x_2$
\item $\forall a, x_1, x_2 \in G: x_1 \circ a = x_2 \circ a \implies x_1 = x_2$
\end{enumerate}
\sssect{Beweis}
\begin{enumerate}
\item Sei $a, x_1, x_2 \in G$ und $a \circ x_1 = a \circ x_2$\\
Wegen $G^* = G \text{ ex. } a^{-1} \in G$\\
Also gilt $a^{-1} \circ (a \circ x_1) = a^{-1} \circ (a \circ x_2)$ sowie\\
$(a^{-1} \circ a) \circ x_1 = (a^{-1} \circ a) \circ x_2$\\
Daraus folgt:\\
$e \circ x_1 = e \circ x_2$ und $x_1 = x_2$
\item Analog
\end{enumerate}
\ssect{satz}
Sei $(G, \circ)$ eine Gruppe. Dann sind alle Gleichungen $a \circ x = b$, $z \circ a = b$ mit $a, b \in G$
in G eindeutig loesbar.
\ssect{Satz}
Sei $(H, \circ)$ ein Halbgruppe. $(H, \circ)$ ist genau dann eine Gruppe, wenn alle Gleichungen
$ax = b, ya =b$ mit $a, b \in H$ in H eindeutig loesbar.
\ssect{Satz}
Sei $(H, \circ)$ eine endliche Halbgruppe. Gelten in $(H, \circ)$ die Kuerzungsregeln, dann ist $(H, \circ) eine Gruppe$

\ssect{Bemerkung}
In $(\bN, +)$ gelten die Kuerzungsregeln, aber $(\bN, +)$ ist keine Gruppe.

\ssect{Beispiel: bijektive Abbildungen sind eine Gruppe}
Menge $A \neq \emptyset$ A endlich
\ilmath{T_{A}^\star = \Set{f: A \rightarrow A \mid f \text{ bijektiv}}}
$(T_{A}^{\star}, \circ)$ ist eine Gruppe.
\sssect{Bezeichnung}

$\mid A\mid = n \implies s_n := T_{A}^\star$
\sssect{Bemerkung}
$\mid s_n\mid = n!$, $(s_n, \circ)$ist die \underline{symmetrische Gruppe von grad n}
\ssect{Definition: Permutationsgruppe}
Die Untergruppen von $(S_n, \circ)$ heissen \underline{Permutationsgruppen}
\ssect{Beispiel: Permutationsgruppen vom Grad 3}
$n = 3$\\
$$ s_3 = \left\{ \m{1 & 2 & 3 \\ 1 & 2 & 3}, \m{1 & 2 & 3 \\ 1 & 3 & 2}, \m{1 & 2 & 3 \\ 3 & 2 & 1},\\
		\m{1 & 2 & 3 \\ 2 & 1 & 3},	\m{1 & 2 & 3 \\ 2 & 3 & 1},	\m{1 & 2 & 3 \\ 3 & 1 & 2} \right \} $$

$(s_3, \circ)$ ist eine nicht abelsche Gruppe.

\ssect{Bemerkung}
$(S_n, \circ)$ ist fuer $n \geq 3$ eine nicht abelsche Gruppe

\ssect{Zyklenschreibweise fuer Permutationen}
Jede Permutation auf einer endlichen Menge laesst sich als Konkatination elementfremder Zyklen schreiben.

z.b.
\ilmath{\alpha = \m{1 & 2 & 3 & 4 & 5 & 6 & 7 & 8 & 9\\ 8 &2 & 4 & 7 & 6 & 3 & 5 & 1 & 9} \in s_9}
in Zyklenschreibweise:
\ilmath{\alpha = (1 8) (2) (3 4 7 5 6 ) (9)}
\ilmath{\alpha = (1 8) (3 4 7 5 6) = (3 4 7 5 6) (1 8)}

Bei elementfremden zyklen ist die Reihenfolge der Zyklen beliebig.

