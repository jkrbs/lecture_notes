\documentclass{../tudscript}
\author{Jakob Krebs}
\title{Mathe VL 18}

\begin{document}
    \ssect{(Totale) Diffenezierbarkeit von Funktionen mit zwei veränderlichen}
        \ilmath{y = f(x_n, \ldots, x_1), patielele Ableitung f_{_i} = \frac{delta f}{\delta x_i}}
        wobei $x_j$ mit $j^+$ const. \enquote{del f nach del x}
    
        ex. partielle Abl. $\implies f$ ist partiell Diffenzierbar,\\
                DEF        $\implies f$ ist \underline{total} differenzierbar\\
        lineare Approximierbarkeit $\implies$ totale Diffenrenzierbarkeit\\

   \ssect{Definition}
        Sei
        \ilmath{f: X \rightarrow \bR, X \subseteq \bR^n, x_0 \in X (\vec{x_i} = (x_0, \ldots, x_n))}     
        f heißt an der Stelle $x_0$ total differenzierbar, wenn gilt:
        \ilmath{\exists a \in \bR^n: f(\vec{x}) = f(\vec{x_0}) + a \circ (\vec{x} - \vec{x_a})^1 + R(\vec{x}), \vec{c} \in X}
        mit 
        \ilmath{\lim_{x \to x_0} \frac{| R_n (\vec{x})|}{|| \vec{x} - \vec{x_0} ||}= 0}
        \ilmath{n = 1 \hfill f(x) = f(x_0) ] f(x_-)(x-x_0) + R(x)}
    \ssect{Bemerkung}
        R(x) geht scheinbar gegen Null, also $x \to x_0$ gilt
    \ssect{Bemerkung}
        Ist f in $\vec{x_0}$ total differenzierbar, dann gilt:
        \ilmath{f(x) \approx f(\vec{x_0}) ] a \circ (x - x_0)^1}
        (lineare Approximation von f)
    \ssect{Bemerkung: Bestimmung von a}
        Ist f an $\vec{x_0}$ linear approximierbar, dann kann $\vec{a}$ wie folgt bestimmt werden.
        \ilmath{a = \m{f_{x_1} (x_0)\\ 
                       f_{x_2} (x_0)\\
                       f_{x_3} (x_0)\\ 
                       f_{x_4} (x_0)\\
                       f_{x_5} (x_0)\\
                       \ldots} \* \m{\frac{\delta f}{\delta x_1}\\
                                     \frac{\delta f}{\delta x_2}\\
                                     \frac{\delta f}{\delta x_3}\\
                                     \ldots}_{|x_0}}
    \ssect{Bemerkung}
        f (total) diffbar $\overset{\implies}{\not\Leftarrow}$ f partiell diffbar\\
        f (total) diffbar $\overset{\Leftarrow}{\not\implies}$ f stetig diffbarr

    \ssect{Verallgemeinerung Kettenregel}
        \ilmath{f(x,y) = z, x = x(t) = y = y(t)\\
               f(x(t), y(t)) = \varphi (t)}
        Beispiel:
        \ilmath{z = f(x,y) = x^2 y, x(t) = t^2, y(t) = t \\
                \varphi (t) = (t^2 )^2 t = t^5\\
                f(x()t, y(t)) = \varphi (t)}
        Ableitung nach t
        \ilmath{f_{x}( x(t), y(t)) \frac{dx}{dt} + f_y (x(t), y(t)) \frac{dy}{dt}\frac{d \varphi}{dt} = \varphi'}

        \ilmath{\boxed{f_x \frac{dx}{dt} + f_y \frac{dy}{dt} = \frac{d \varphi}{dt}}}
    \sssect{Bemerkung}
        Für die Gültigkeit der Formel benötigt man stetige Diffbarkeit (für f,x,y und $\varphi$ und t)
    
    \sssect{Beispiel weiterführung}
        Ableitung nach t:
        \ilmath{f_x \frac{dx}{dt} + f_y \frac{dy}{dt} = 2xy 2t +x^2 = 4t^2 + 4t^4 = 5t^4}
    \ssect{Taylorentwicklung}
        für 
        \ilmath{z=f(x,y)}
        an der Stelle $x_0, y_0$
        \ilmath{f(x,y) &= f(x_0, y_0) + \ldots\\
                       &= f(x_0 + 1 \* (x - x_0)), y_0 + 1 \* (y - y_0)\\
                       &= f(\underbrace{x_0 + t \* \underbrace{(x - x_0)}_{h}}_{x(t) = x_0 = th}), \underbrace{y_0 + t \* \underbrace{(y - y_0)}_{k}}_{y(t) = y_0 = lt}\\
                       &- \varphi (t)}
        Taylorformel für $\varphi (t)$ an der Stelle $t_0 = 0$
        \ilmath{\varphi (t) = \varphi (0) + \varphi' (0) t + 0.5 \varphi^{(2)} (0) t^2 + \ldots + R_n (t)}
        für t = 1
        \ilmath{\varphi (t) = \varphi (0) + \varphi' (0) + 0.5 \varphi'' (0) + \ldots + \frac{1}{n!} \varphi^{(n)} t^n + R_n (1)}
        $\varphi (t) = f(x(t), y(t))$ nach t ableiten und dabei die verallg. KEttenregel verwenden.
        \ilmath{\varphi' (t) = f_x h + f_y k}
        \ilmath{\varphi'' (t) = (f_{xx} h + f_{yx} k)h + (f_{yx} h + f_{yy} k)k = f_{xx} h^2 ] 2 f_{xy} hk + f_{yy} k^2}
        \ilmath{\varphi''' (t) = f_{xxx} h^3 + 3 f_{xxy} h^2 k + 3 f_{yyx} k^2 h+ f_{yyy} k^3}
        usw. siehe pascalsches Dreieck

     \sssect{Lineare Näherungformel}
        für $f(x,y)$ an $(x_0, y_0)$
        \ilmath{\varphi (0) + \varphi' (0) &= \underbrace{f(x_0 + 0 \* (x - x_0), y_0 + 0 \* (x-x_0))}_{f(x_0, y_0)}\\ 
                                                    &+ f_x (x_0, y_0) \* (x-x_0) + f_y (x_0, y_0) \* (y-y_0)}
        beschreibt die Tangentialebene
   \sssect{quadrtische Näherungformel}
        für $f(x,y)$ an $(x_0, y_0)$
        \ilmath{\boxed{tangentialebene} &+ \frac{1}{2} (f_{xx} (x_0, y_0) (x-x_0)^2\\ 
                                        &+ 2 f_{xy} (x-x_0)(y-y_0) + f_{yy} (y-y_0)^2)}
\end{document}
