\documentclass{../tudscript}
\author{Jeroen Trzaska}
\title{Mathe VL 5}

\begin{document}
$(H_1,\circ_1)$ und $(H_2,\circ_2)$ seine Halbgruppen.
Sei $h: H_1 \rightarrow H_2$ ein surjektiver Homomorphismus, d.h. :
\begin{itemize}
\item $\forall a,b \in H_1 | h(a \circ_1 b) = h(a) \circ_2 h(b)$
\item $\{h(a)|a \in H_1\} = H_2$
\end{itemize}
Dann gilt:
\begin{enumerate}
  \item ker(h) $:= \{(a,b) \in H_1 \times H_1 | h(a)=h(b)\}$ ist eine
    Kongruenzrelation in $(H_1, \circ_1)$
  \item Die Abbildung $nat_{ker(h)}: H_1 \rightarrow H_1/_{ker(h)} : a \mapsto
      [a_{Ker(h)}]$ ist ein Homomorphismus (natuerlicher Homomorphismus).
      $nat_{ker(h)}$ surjektiv.
\end{enumerate}
\sssect{Beispiel}
Halbgruppen $(\bN,+)$ und $(\bZ_4,+_{\mod 4})$
\ilmath{h:\bN \rightarrow \bZ_4 : a \mapsto a \mod 4}
ist ein surjektiver Homomorphismus
\begin{itemize}
\item $ker(h) = \{(a,b) \in \bN \times \bN | a \mod 4 \}$ ist eine
  Kongruenzrelation in $(\bN,+)$
 \item $nat_{ker(h)}: \bN \rightarrow \bN/_{ker(h)}: a \mapsto [a]_{ker(h)}$ ist
   ein surjektiver Homomorphismus
 \end{itemize}
 \sssect{Beweis von 1)}
 \ilmath{ ker(h) := \{(a,b) \in H_1 \times H_1 | h(a) = h(b)\}}
 \begin{itemize}
 \item ker(h) ist eine Aequivalenzrelation in $H_1$
 \item ker(h) ist strukturvertraeglich, da \\
   \ilmath{(a,b),(c,d) &\in ker(h)\\
     &\implies h(a) = h(b), h(c) = h(d)\\
     &\implies h(a) \circ_2 h(c) = h(b) \circ_2 h(d)\\
     &\implies h(a \circ_1 c) = h(b \circ_1 d)\\
     &\implies (a \circ_1 c, b \circ_1 d) \in ker(h)}
   
   Also ist ker(h) eine Kongruenzrelation
 \end{itemize}
 \sssect{Beweis von 2)}
 $nat_{ker(h)} : H_1 \rightarrow H_1/_{Ker(h)}: a \mapsto [a]_{ker(h)}$
 \begin{itemize}
   \item $nat_{ker(h)}$ ist surjektiv (denn fuer jeder $[a]_{ker(h)} \in H_1/_{ker(h)}$
   existiert ein $a \in H_1$ mit $nat_{ker(h)} [a] = [a]_{ker(h)}$)
 \item $nat_{ker(h)}$ ist strukturvertraeglich, denn:
   \ilmath{nat_{ker(h)}(a \circ_1 b) &= [a \circ_1 b]_{ker(h)}\\
     &= [a]_{ker(h)} \circ_{ker(h)} [b]_{ker(h)}\\
     &= nat_{ker(h)} \circ_{ker(h)} [b]_{ker(h)}\\
   &= nat_{ker(h)} (a) \circ_{ker(h)} nat_{ker(h)} (b) \text{ fuer alle } a,b
   \in H_1}
 Also ist $nat_{ker(h)}$ ein surjektiver Homomorphismus
\end{itemize}
\ssect{Homomorphiesatz fuer Halbgruppen}
$(H_1, \circ_1), (H_2,\circ_2)$ seien Halbgruppen\\
$h:H_! \rightarrow H_2$ sei ein surjektiver Homomorphismus.\\
$ker(h) : \{(a,b) \in H_1 \times H_1 | h(a) = h(b)\}$ ist Kongruenzrelation\\
Die Faktorhalbgruppe von $(H_1, \circ_1)$ nach $ker(h)$ ist zu $(H_2,\circ_2)$
isomorph und\\
$f: H_1/_{ker(h)} \rightarrow H_2 : [a]_{ker(h)} \rightarrow h(a)$ ist ein
Isomorphismus:
\ilmath{H_1/_{ker(h)},\circ_{ker(h)} \underset{f}{\cong} (H_2,\circ_2)}
\sssect{Beispiel}
Halbgruppen $\bN,+$ und $\bZ_4, +_{\mod 4}$\\
surjektiver Homomorphismus $h:\bN \rightarrow \bZ_4 : a \mapsto a \mod 4$\\
$ker(h) = \{(a,b) \in \bN \times \bN | a \mod 4 = b \mod 4\}$\\
Faktorhalbgruppe
\ilmath{\bN/_{ker(n)} = \{[2]_{ker(h)} | a \in \bN\}=
  \{[0]_{ker(h)},[1]_{ker(h)},[2]_{ker(h)},[3]_{ker(h)}\}\\
  [a]_{ker(h)} +_{ker(h)} [b]_{ker(h)} = [a+b]_{ker(h)}\\
  (\bN/_{ker(h)},+_{ker(h)})\\
(\bN/_{ker(h)} \underset{f}{\cong} (\bZ_4,+_{\mod 4}))}
Isomorphismus $f: \bN/_{ker(h)} \rightarrow \bZ_4:[a]_{ker(h)}\mapsto a \mod 4$
$(G_1, \circ_1)$ und $(G_2,\circ_2)$ seien Gruppen.
Sei $h:G_1 \rightarrow G_2$ eine surjektiver Homomorphismus, d.h.:
\begin{itemize}
\item $\forall a,b \in G_1 : h(a \circ_1 b) = h(a) \circ_2 h(b)$
\item $\{h(a) | a \in G_1\}$
\end{itemize}
Dann gilt:
\begin{enumerate}
  \item $ker(h):=\{a \in G_1 | h(a) =  e_2\}$ ist ein Normalteiler in
    $(G_1,\circ_1)$
    \item Die Abbildung $nat_{ker(h)}: G_1 \rightarrow G_1/_{ker(h)}:g \mapsto
      ker(h)$ ist ein Homomorphismus (natuerlicher Homomorphismus),
      $nat_{ker(h)}$ ist surjektiv
\end{enumerate}
\sssect{Beispiel}
Gruppen $(\bZ,+)$ und $\bZ_4, +_{\mod 4}$\\
$h: \bZ \rightarrow \bZ_4 | a \mapsto a \mod 4$ ist ein surjektiver
Homomorphismus.
\begin{enumerate}
\item $ker(h) = \{a \in \bZ | a \mod 4 = 0\} = 4 \bZ = \{4z|z \in \bZ\}$ ist ein
  Normalteiler in $(\bZ,+)$
  \item $nat_{ker(h)}:\bZ \rightarrow \bZ/_{4_\bZ}:a \mapsto a + 4 \bZ$ ist ein
    surjektiver Homomorphismus.
  \end{enumerate}

  \sssect{Beweis}
  $ker(h) := \{a \in G_1 | h(a) = e:2\}$
  \begin{itemize}
  \item $ker(h) \neq \emptyset$, denn $h(e_1) = e_2$ und $e_1 \in ker(h)$
    \item $a,b \in ker(h) \implies h(a \circ_1 b) = h(a) \circ_2 h(b) =
        e_2 \circ e_2 = e_2 \implies a \circ_1 b \in ker(h)$
      \item $a \in ker(h) \implies h(a) = e_2 \implies h(a)^{-1} =
          e_2^{-1} \implies h(a^{-1})= e_2 \implies a^{-1} \in ker(h)$

\item $a ker(h) = \{a \circ_1 g | g \in ker(h)\} \overset{\ast}{=}\{b \in G_1 |
  h(a) = h(b)\} \\
  ker(h) a = \{g \circ_1 a | g \in ker(h)\} \overset{\ast}{=} \{b \in G_1 | h(a)
  = h(b)\}$\\
$\implies a ker(h) = ker(h)a$ fuer alle $a \in G_1$
Also ist $ker(h)$ ein Normalteiler in $(G_1, \circ_1)$
\end{itemize}
$\ast$\\
\ilmath{h(a \circ_1 g) = h(a) \circ_2 h(g) = h(a) \circ_2 e_2 = h(a)}
und
\ilmath{h(b)=h(a) &\implies h(a)^{-1} \circ_2 h(b) = h(a)^{-1} \circ_2 h(a)\\
  &\implies h(a^{-1}) \circ_2 h(b) = e_2\\
  &\implies h(a^{-1} \circ_1 b) = e_2\\
  &\implies a^{-1} \circ_1 b \in ker(h)\\
  &\implies b \in a ker(h)}
$\ast \ast$ analog
\sssect{Beweis von 2)}
$nat_{ker(h)}: G_1 \rightarrow G_1/_{ker(h)}:g \mapsto g ker(h), g\in g ker(h)$
\begin{itemize}
\item $nat_{ker(h)}$ ist surjektiv, denn $\forall a ker(h) \in G_1/_{ker(h)}:
  \exists a\in G_1 mit  nat_{ker(h)} (a) = a ker(h)$
\item $\forall a,b \in G$ gilt: \\
  \ilmath{nat_{ker(h)}(a \circ_1 b) &= (a \circ_1 b) ker(h)\\
    &= a ker(h) \circ_{ker(h)} ker(h)\\
    &= nat_{ker(h)} (a) \circ_{ker(h)} nat_{ker(h)} (b)}
  Also ist $nat_{ker(h)}$ ein surjektiver Homomorphismus.
\end{itemize}
\ssect{Homomorphiesatz fuer Gruppen}
$G_1, \circ_1$ und $(G_2,\circ_2)$ seien Gruppen.\\
$h:G_1 \rightarrow G_2$ sei ein surjektiver Homomorphismus.\\
$ker(h) := \{a \in G_1 | h(a) = e_2\}$ ist ein Normalteiler.\\
Die Faktorgruppe von $(G_1, \circ_1)$ nach $ker(h)$ ist zu $(G_2,\circ_2)$
isomorph und $f:G_2/_{ker(h)} \rightarrow G_2 : a ker(h) \rightarrow h(a)$ ist
ein Isomorphismus.
\ilmath{(G_1/_{ker(h)},\circ_{ker(h)}) \underset{f}{\cong} (H_2, \circ_2)}
\end{document}
