\documentclass{../tudscript}
\title{VL 06}
\author{Jakob Krebs}

\begin{document}
\sect{Ringe und Koerper}
\ssect{Definition: kommutativer Ring}
$(R, +, \cdot)$ hiesst Ring \underline{Ring}, wenn gilt:
\begin{enumerate}
\item $(R, +)$ ist eine abelsche Gruppe
\item $(R, \cdot)$ ist eine Halbgruppe
\item $a \cdot (b + c) = (a \cdot b) + (a \cdot c)$ und $(b +c) \cdot a = (b \cdot a) + (c \cdot a)$ fuer alle $a, b, , c \in R$ ($\cdot$ ist distributiv bzgl +)
  \item $a \cdot b = b \cdot a$ fuer alle $a, b \in R$
\end{enumerate}
, dann wird $(R, +, \cdot)$ ein \underline{kommuttiver Ring} genannt.
\ssect{Bemerkung}
Punktrechnung vor Strichrechnung
\ssect{Beispiele}
kommutative Ringe:
\begin{itemize}
\item $\bZ$
\item $\bR$
\item $\bC$
\item $\bZ_n$
\end{itemize}
Ringe
\begin{itemize}
\item $\bR^{n \times n}$
\end{itemize}
\ssect{Bemerkung: Nullelement}
Jeder Ring $(R, +, \cdot)$ enthaelt ein neutrales Element bzgl $+$, das \underline{Nullelement 0}
Das Nullelement ist eindeutig bestimmt.
\ssect{Bemerkung}
Sei $a \in R$. Dann ex. ein eindeutig bestimmtes Element von a bzgl $+$, Bezeichnung $-a$
\ssect{Bemerkung}
Nicht jedes $a \in R$ hat ein inverses Element bzgl $\cdot$, Bezeichnung $a^{-1}$
\ssect{Unterring}
Sei $(R, +, \cdot)$ ein Ring. $(U, +, \cdot)$ mit $U \subseteq R$ ist ein \underline{Unterring} von $(R, +, \cdot)$, wenn gilt:
\begin{enumerate}
\item $0 \in U$
\item $a, b \in U \implies a +b \in U$ fuer alle $a, b \in U$
\item $a \in U \implies -a \in U$ fuer alle $a \in U$
\item $a, b \in  U\implies a \cdot b \in U$ fuer alle $a, b \in U$
\end{enumerate}
\ssect{Bemerkung}
$0 \notin U implies (U, +, \cdot)$ ist kein Unterring
\ssect{Definition}
Sei $(R, +, \cdot)$ ein Ring.
Ex. ein Element $1 \in R \setminus \Set{0}$ mit $a \cdot 1 = 1 \cdot a = a$ fuer alle $a \in R$, dann wird 1 \underline{Einselement} im Ring R genannt.
\sect{Bemerkung}
1 ist das neutrale Element bzgl $\cdot$. Ex. ein Einselement, dann ist es eindeutig bestimmt. Ringe mit Einselement haben mindestens zwei Elemente.
\ssect{Beispiel}
Sei M eine Menge, $M \neq \emptyset$
$(P(M), \Delta, \cap)$ ist ein Ring mit Nullelement $\emptyset$, Einselement $M$.

$(\bZ, \oplus, \odot)$ mir $a \oplus b = a + b -1$ und $a \odot b = a + b - ab$
ist ein Ring mit Nullelement $1$ und Einselement $0$.
\ssect{Rechnen im Ring $(R, +, \cdot)$}
Sei $a \in R$. Dann gilt:
\begin{enumerate}
\item $a \cdot 0 = 0$
\item $0 \cdot a = 0$
\end{enumerate}
\ssect{Definition: Nullteiler im Ring}
Sei $(R, +, \cdot)$ ein kommutativer Ring. Sei $a, b \in R \setminus \Set{0}$ und $a \cdot b = 0$.
Dann werden $a, b$ \underline{Nullteiler} genannt.
\ssect{Bemerkung}
Ein Ring heisst nullteilerfrei, wenn er keinen Nullteiler enthaellt.
\ssect{Definition}
Ein Integritaetsring ist ein kommutativer nullteilerfreier Ring mit Einselement.
\sssect{Beispiele}
\begin{itemize}
\item $\bZ$ integer
\item $2\bZ$ ist kein Integritaetsring
\item $\bZ_p p Prim$ ist ein Integritaetsring
\item $\bZ_p p kein Prim$ ist kein Integritaetsring
\end{itemize}
  \sssect{Beweis}
  $\bZ_n$ mit $n = a \cdot b, 1 < a \leq b < n$.(in $\bZ$) 
  $0 = a \cdot b, a, b \in \bZ_n$
  $\implies a, b $ sind Nullteiler in $\bZ_n$

  \ssect{Definition}
  Sei $(R, +, \cdot)$ ein Ring mit Einselement. $a \in R$ heisst Einheit, wenn $a^{-1}$ existiert.
  \ssect{Bemerkung}
  $a^{-1}$ bezeichnet das Inverse von a bzgl $\cdot$
  \sect{Definition: Koerper}
  Sei $(R, +, \cdot)$ ein kommutativer Ring mit Einselement. Ist $(R \setminus \Set{0}, \cdot)$ eine Gruppe, dann wird $(R, +, \cdot)$ ein \underline{Koerper} genannt.
  \ssect{Bemerkung}
  Fuer jeden Koerper ist die multiplikative Gruppe $(R \setminus \Set{0}, \cdot)$ abelsch.

  $(R, \cdot)$ ist keine Gruppe.
  \ssect{Bemerkung}
  Ist der Ring $(R, +, \cdot)$ ein Koerper, dann sind alle von 0 verschiedenen Elemente Einheiten.
  \ssect{Satz}
  Jeder Koerper ist ein Integritaetsring.
  Beweis:\\
  Sei K ein koerper,
  \ilmath{a, b \in K, a \cdot b = 0 &\implies a = 0 \lor a^{-1} ex. \\
    &\implies a = 0 \lor a^{-1} \cdot (a \cdot b) = a^{-1} = 0 \\
  &\implies a = 0 \lor b = 0}
  Also gibt es in K keine Nullteiler, also is K ein Integritaetsring$\hfill\square$
  \ssect{Bemerkung}
  Es gibt INtegritaetsringe, die keine Koerper sind.
  \ssect{Satz}
  Jeder endliche Integritaetsring ist ein Koerper

  Beweis:\\
  Sei $a \in R, a \neq 0$. Z.z?: $a^{-1}$ ex in R
  [\ldots]
  
\end{document}
