\documentclass{../tudscript}
\author{Jakob Krebs}
\title{Mathe VL 11}

\begin{document}
\part{Numerische Mathematik}
\sect{Auswertung von Polynomen}
\ssect{Definition}
Sei $p(x):- a_n x^n + \ldots + a_1 x + a_0 \in K[x]$ K Körper $\alpha \in K$
Man nennt
\ilmath{p(\alpha) = a_n \alpha^n + \ldots + \alpha_0}
die Auswertung von p(x) an der Stelle $\alpha$. Die Polynomfunktion zu p(x) ist die Abbildung $K \rightarrow K: \alpha \mapsto p(\alpha)$
\ssect{Bemerkung}
Zur Aswertung von $p(x) = a_n x^n + \ldots + a_0$ in der Form
\ilmath{p(x) = \alpha_0 + x ( a_1 + x (a_2 + x (\ldots)))}
benötigt man n Multiplikationen und n Additionen.
\ssect{Horner Schema}
Die Form: siehe oben. Geht mit Fancy Tabelle

\sssect{Bemerkung}
Das Horner Schema kann man auch zur Polynomdivision von p(x) durch $x -alpha$
verstehen:
\ilmath{p(x) = q(x) (x-\alpha) + r}
Insebesondere gilt
\ilmath{p(\alpha) = q(\alpha) \cdot \underbrace{(x- \alpha)}_{0} + r = r}
\ssect{Satz: D.E. Knuth}
Jedes reelle Polynom vom Grad $\geq 3$ lässt sich mit $\lfloor \frac{n}{2} \rfloor + 2$ Multiplikationen und $n+1$ Additionen ausrechnen 
\sect{Interpolation}
\underline{Interpolationsaufgabe}\\

Gegeben: $n+1$ Punkte $(x_i, g_i) (i = 0, 1, \ldots, n)$ mit paarweise verschiedenen Stützstellen

Gesucht: Polynom p(x) mit
\ilmath{p(x_i) = g_i (i = 0, 1, \ldots, n)}
als Interpolationsbedingung.
\ilmath{g = p(x) = \sum_{k = 0}^n a_k x^k}
erfüllt die Interpolationsbedingung.

Lösen der Interpolationsaufgabe durch Aufstellen und lösen des LGS ist möglich, aber sehr rechenaufwändig.
\sssect{Bemerkung: lösbarkeit}
Das LGS ist lösbar, weil die Koeffizientenmatrix eine von Null verschiedene Determinate hat. (mit der VANDERHONDE Determinatnte zeigen).

\ssect{Allgemein}
Gegeben: $n+1$ Punkte mit paarweise verschiedenen Stützpunkten.

Gesucht: Gunktion F(x) aus einer bestimmten Funktionenklasse, die die INterpolationsbedingungen
\ilmath{F(x_i) = g_i (i - 0, \ldots, n)}
erfüllt.

\sssect{Bemerkung}
F(x) wird als Linearkombination von Basisfunktionen $\varphi_k (x)$ dagestellt:
\ilmath{F(x) = \sum_{k = 0}^n a_k \varphi_k (x)}
Gesucht sind die Koeffizienten $a_0, \ldots, a_n$

\begin{itemize}
\item $p(x) = \sum_{k = 0}^n a_k x^k$ Basisfunktion $1, x, x^2, \ldots, x^k$ (Berechnugn von $a_0, \ldots, a_k$ nicht effizient möglich)
\item $p(x) \sum_{k = 0}^n a_k L_k (x)$ Basisfunktion $L_0(x), \ldots, L_n(x)$ Lagrange Basispolynome: Gesucht: $a_0, \ldots, a_k$
\item $p(x) = \sum_{k = 0}^n c_k N_k (x)$ Basisfunktion $N_0 (x), \ldots, N_1 (x)$ Newton Basispolynome
\item $p(x) = \sum_{k = 0}^n a_k \varphi_k (x)$ Basisfunktion $\varphi_0 (x), \ldots, \varphi_n (x)$ B Splines
\end{itemize}

\ssect{Lagrange Interpolationsmethode}
Definition der Lagrange POlynome $L_k (x) (k = 0, \ldots n)$
\ilmath{L_k (x) = \frac{\prod_{k =0, j \neq k}^n (x-x_j)}{\prod_{j = 0, j \neq k}^n (x_k -x_j)}}

Eigenschaften der Lagrange Polynome $L_k (x)$
\ilmath{K_k (x_i) = \begin{cases} 1; i = k \\ 0, i \in \Set{0, 1, \ldots, n} \setminus \Set{k}\end{cases}}
Die POlynome $L_0 (x), L_1(x), \ldots, L_n (x)$ sind linear unabhängig.

\ssect{Lagrange Interpolation}
zu den Stützstellen
\begin{enumerate}
\item Lagrange Polynome $L_0, \ldots$ berechnen
\item Ansatz: $p(x) a_0 L_0 (x) + \ldots + a_n L_n (x)$ Gesucht $a_1, \ldots a_n$
\item Berechnung von $a_0, \ldots, a_n$ mit der Interpolationsbedingung $p(x_i) = g_i$
  $y_i = a_i$.
\item Ergebnis:
  \ilmath{p(x) = \sum_{k = o}^n y_k \cdot L_k(x)}
  Lagrangepolynom der Ordnung n
\end{enumerate}
\ssect{Newton Interpolationmethode}
\begin{itemize}
\item Definition der Newton POlynome $N_k(x) (k = 0, \ldots, n)$
  $N_0 (x) := 1$\\
  $N_k (x) := (x -x_0) (x - x_1) \cdot \ldots \cdot (x - x_{k-1})$ für $(k = 1, \ldots, n)$
\item Eigenschaften der Newoton POllynome %N_k (x) (k = 0, \ldots, n$
  \ilmath{N_k(x_i) = 0 \text{ für alle } i \in \Set{0, 1, \ldots, k-1}}
  Die Polynome $N_0 (x), N_1 (x), \ldots, N_n(x)$ sind linear unabhängig.
\end{itemize}

\underline{Berechnung von $c_0, \ldots, c_n$}
\begin{itemize}
  \item $p(x_0) = c_0 N_0 (x_0) + c_1 N_1 (x_0) + \ldots = z_0\implies c_0 = y_0$
  \item $p(x_1) = c_0 N_0 (x_1) + c_1 N_1 (x_1) + \ldots  c_n \underbrace{N_n(x_0)}_{0} = y_1$
    \ilmath{\implies c_1 = \frac{y_1 - y_0}{x_1 - x_0} =: [x_1;x_0]}
    Steigung (1. Ordnung)
    \ilmath{\implies c_2 = \frac{[x_2 x_1] - [x_1 - x_0}{x_2 - x_0}}
    Steigung (2. Ordnung)
\end{itemize}
\ssect{Newton Interpolation}
\begin{enumerate}
\item Newton POlynom $N_0 (x), \ldots$ breechne
\item Ansatz $p(x) = x_0 N_0 (x) + \ldots$ Gesucht $c_0, \ldots$
\item Berechnung von $c_0 , \ldots$
  Es ist ein gestaffeltes lineares Gelichungssystem zu lösen
\item Ergebnis $p(x) = \sum_{k = 0}^n c_k N_k (x)$ Newton Interpolationspolynom der Ordnung n
\end{enumerate}
\enquote{Der Steigunsspiegel heißt Spiegel, weil er kein Spiegel ist. Zudem ist es der einzige Spiegel, den ich kenne, bei dem rechts was völlig anderes steht, als auf der linken Seite.}
\end{document}
